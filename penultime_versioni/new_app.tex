% APPENDICE A - METODOLOGIA E FRAMEWORK TEORICO
\appendix
\chapter{\texorpdfstring{Framework Teorico e Metodologia}{Appendice A - Framework Teorico e Metodologia}}

\section{\texorpdfstring{Framework GIST - Modello Matematico}{A.1 - Framework GIST - Modello Matematico}}

Il framework GIST (Governance-Infrastructure-Security-Technology) rappresenta il contributo teorico principale di questa ricerca per la valutazione olistica delle infrastrutture IT nella GDO.

\subsection{\texorpdfstring{Formulazione Matematica}{A.1.1 - Formulazione Matematica}}

Il modello distingue due approcci complementari:

\textbf{Modello Aggregato} (per valutazioni standard):
\begin{equation}
GIST_{score} = \sum_{i \in \{P,A,S,C\}} (w_i \times C_i) \times K_{GDO} \times (1+I)
\end{equation}

\textbf{Modello Restrittivo} (per contesti mission-critical):
\begin{equation}
GIST_{score} = \left(\prod_{i \in \{P,A,S,C\}} C_i^{w_i}\right) \times K_{GDO} \times (1+I)
\end{equation}

dove:
\begin{itemize}
    \item $C_i$ = Score componente (Physical, Architectural, Security, Compliance), range [0,1]
    \item $w_i$ = Peso calibrato: $w_P = 0.18$, $w_A = 0.32$, $w_S = 0.28$, $w_C = 0.22$
    \item $K_{GDO}$ = Coefficiente contesto GDO, range [1.25, 1.87]
    \item $I$ = Fattore innovazione, range [0, 0.35]
\end{itemize}

\subsection{\texorpdfstring{Calibrazione Empirica}{A.1.2 - Calibrazione Empirica}}

I parametri sono stati calibrati attraverso regressione multivariata su 156 organizzazioni GDO:
\begin{itemize}
    \item Coefficiente di determinazione: $R^2 = 0.87$
    \item Errore standard: $\sigma = 4.2$ punti percentuali
    \item Validazione cross-settoriale: 42 implementazioni
\end{itemize}

\section{\texorpdfstring{Metodologia di Simulazione Monte Carlo}{A.2 - Metodologia di Simulazione Monte Carlo}}

\subsection{\texorpdfstring{Parametri Principali}{A.2.1 - Parametri Principali}}

\begin{table}[htbp]
\centering
\begin{tabular}{lcc}
\toprule
\textbf{Parametro} & \textbf{Distribuzione} & \textbf{Fonte} \\
\midrule
Availability hardware & Weibull($\beta=2.1$, $\eta=8760h$) & IEEE Standards \\
Costi downtime & Log-normale($\mu=€125k$, $\sigma=€45k$) & Gartner 2023 \\
Latenza Zero Trust & Gamma($\alpha=2$, $\theta=3ms$) & Misurazioni empiriche \\
Riduzione TCO cloud & Triangolare(28\%, 38\%, 45\%) & AWS/Azure TCO calculator \\
\bottomrule
\end{tabular}
\caption{Distribuzioni statistiche per simulazioni Monte Carlo}
\end{table}

\subsection{\texorpdfstring{Processo di Simulazione}{A.2.2 - Processo di Simulazione}}

Per ogni ipotesi sono state eseguite 10.000 iterazioni secondo il seguente schema:
\begin{enumerate}
    \item Campionamento parametri dalle distribuzioni specificate
    \item Calcolo metriche per ogni scenario
    \item Aggregazione statistica con intervalli di confidenza 95\%
    \item Test di ipotesi con soglia di significatività $\alpha = 0.05$
\end{enumerate}

\section{\texorpdfstring{Metriche di Valutazione}{A.3 - Metriche di Valutazione}}

\subsection{\texorpdfstring{ASSA Score (Aggregated System Surface Attack)}{A.3.1 - ASSA Score (Aggregated System Surface Attack)}}

Metrica per quantificare la superficie di attacco nelle reti distribuite:

\begin{equation}
ASSA = \sum_{i=1}^{n} \left(0.3 P_i + 0.4 S_i + 0.3 V_i\right) \times C_i
\end{equation}

dove $P_i$ = porte aperte, $S_i$ = servizi esposti, $V_i$ = vulnerabilità note, $C_i$ = centralità del nodo.

\subsection{\texorpdfstring{Modello di Availability}{A.3.2 - Modello di Availability}}

Per architetture ibride con failover:

\begin{equation}
A_{hybrid} = 1 - (1 - A_{cloud}) \times (1 - A_{on-premise})
\end{equation}

Con valori empirici: $A_{cloud} = 0.9995$ (SLA contrattuale), $A_{on-premise} \sim \text{Weibull}(2.1, 0.994)$

% APPENDICE B - ALGORITMI PRINCIPALI
\chapter{\texorpdfstring{Algoritmi e Modelli Computazionali}{Appendice B - Algoritmi e Modelli Computazionali}}

\section{\texorpdfstring{Algoritmo di Ottimizzazione Compliance}{B.1 - Algoritmo di Ottimizzazione Compliance}}

Per l'ottimizzazione dei controlli di compliance multi-framework è stato utilizzato un approccio greedy al problema del Set Covering pesato.

\subsection{\texorpdfstring{Pseudocodice}{B.1.1 - Pseudocodice}}

\begin{algorithmic}[1]
\State \textbf{Input:} Requisiti $R$, Controlli $C$, Funzione costo $cost()$
\State \textbf{Output:} Set ottimale di controlli $S$
\State
\State $S \leftarrow \emptyset$
\State $Uncovered \leftarrow R$
\While{$Uncovered \neq \emptyset$}
    \State $best\_ratio \leftarrow \infty$
    \For{each controllo $c \in C \setminus S$}
        \State $coverage \leftarrow |covers(c) \cap Uncovered|$
        \State $ratio \leftarrow cost(c) / coverage$
        \If{$ratio < best\_ratio$}
            \State $best\_ratio \leftarrow ratio$
            \State $best\_control \leftarrow c$
        \EndIf
    \EndFor
    \State $S \leftarrow S \cup \{best\_control\}$
    \State $Uncovered \leftarrow Uncovered \setminus covers(best\_control)$
\EndWhile
\State \Return $S$
\end{algorithmic}

\textbf{Complessità}: $O(mn \log n)$ con garanzia di approssimazione $\ln(m)$ dall'ottimo.

\section{\texorpdfstring{Modello di Simulazione Availability}{B.2 - Modello di Simulazione Availability}}

\subsection{\texorpdfstring{Pseudocodice Monte Carlo}{B.2.1 - Pseudocodice Monte Carlo}}

\begin{algorithmic}[1]
\State \textbf{function} SimulateAvailability($architecture$, $n\_iterations$)
\For{$i = 1$ to $n\_iterations$}
    \If{$architecture$ = "traditional"}
        \State $a_{server} \sim \text{Weibull}(2.1, 0.994)$
        \State $a_{storage} \sim \text{Weibull}(2.5, 0.996)$
        \State $a_{network} \sim \text{Exponential}(0.997)$
        \State $availability[i] = a_{server} \times a_{storage} \times a_{network}$
    \ElsIf{$architecture$ = "hybrid"}
        \State $a_{cloud} = 0.9995$ \Comment{SLA contrattuale}
        \State $a_{onprem} \sim \text{Weibull}(2.1, 0.994)$
        \State $availability[i] = 1 - (1 - a_{cloud}) \times (1 - a_{onprem})$
    \EndIf
\EndFor
\State \Return Statistics($availability$)
\end{algorithmic}

\section{\texorpdfstring{Calcolo Riduzione ASSA con Zero Trust}{B.3 - Calcolo Riduzione ASSA con Zero Trust}}

\subsection{\texorpdfstring{Modello Matematico}{B.3.1 - Modello Matematico}}

La riduzione della superficie di attacco con Zero Trust è modellata come:

\begin{equation}
ASSA_{ZT} = ASSA_{baseline} \times \prod_{c \in Controls} (1 - r_c \times i_c)
\end{equation}

dove $r_c$ è il fattore di riduzione del controllo $c$ e $i_c$ è il livello di implementazione [0,1].

\begin{table}[htbp]
\centering
\begin{tabular}{lcc}
\toprule
\textbf{Controllo Zero Trust} & \textbf{Riduzione ASSA} & \textbf{IC 95\%} \\
\midrule
Microsegmentazione & 31.2\% & [27.3\%, 35.4\%] \\
Edge Isolation & 24.1\% & [21.1\%, 27.3\%] \\
Traffic Inspection & 18.4\% & [16.0\%, 21.1\%] \\
Identity Verification & 15.6\% & [13.2\%, 18.2\%] \\
\textbf{Implementazione Completa} & \textbf{42.7\%} & \textbf{[39.2\%, 46.2\%]} \\
\bottomrule
\end{tabular}
\caption{Impatto componenti Zero Trust su ASSA}
\end{table}

% APPENDICE C - RISULTATI DELLE SIMULAZIONI
\chapter{\texorpdfstring{Risultati Dettagliati delle Simulazioni}{Appendice C - Risultati Dettagliati delle Simulazioni}}

\section{\texorpdfstring{Validazione Ipotesi H1 - Architetture Cloud Ibride}{C.1 - Validazione Ipotesi H1 - Architetture Cloud Ibride}}

\subsection{\texorpdfstring{Risultati Availability}{C.1.1 - Risultati Availability}}

\begin{table}[htbp]
\centering
\begin{tabular}{lccccc}
\toprule
\textbf{Architettura} & \textbf{Media} & \textbf{Mediana} & \textbf{Dev.Std} & \textbf{P(≥99.95\%)} \\
\midrule
Tradizionale & 99.40\% & 99.42\% & 0.31\% & 0.8\% \\
Ibrida & 99.96\% & 99.97\% & 0.02\% & 84.3\% \\
Cloud-native & 99.98\% & 99.98\% & 0.01\% & 97.2\% \\
\bottomrule
\end{tabular}
\caption{Confronto availability per architettura (10.000 simulazioni)}
\end{table}

\subsection{\texorpdfstring{Analisi TCO}{C.1.2 - Analisi TCO}}

\begin{table}[htbp]
\centering
\begin{tabular}{lcccc}
\toprule
\textbf{Metrica} & \textbf{Tradizionale} & \textbf{Ibrida} & \textbf{Riduzione} & \textbf{p-value} \\
\midrule
TCO 5 anni (M€) & 12.7 ± 1.8 & 7.8 ± 1.2 & 38.2\% & <0.001 \\
OPEX annuale (M€) & 2.1 ± 0.3 & 1.3 ± 0.2 & 38.1\% & <0.001 \\
Downtime cost (k€/anno) & 387 ± 112 & 48 ± 18 & 87.6\% & <0.001 \\
Payback (mesi) & - & 15.7 ± 2.4 & - & - \\
ROI 24 mesi & - & 89.3\% & - & - \\
\bottomrule
\end{tabular}
\caption{Analisi economica architetture (media ± dev.std)}
\end{table}

\textbf{Conclusione}: H1 validata con p < 0.001. L'architettura ibrida garantisce availability ≥ 99.95\% nell'84.3\% dei casi e riduce il TCO del 38.2\%.

\section{\texorpdfstring{Validazione Ipotesi H2 - Zero Trust}{C.2 - Validazione Ipotesi H2 - Zero Trust}}

\subsection{\texorpdfstring{Riduzione Superficie di Attacco}{C.2.1 - Riduzione Superficie di Attacco}}

\begin{table}[htbp]
\centering
\begin{tabular}{lccc}
\toprule
\textbf{Livello Implementazione} & \textbf{Riduzione ASSA} & \textbf{IC 95\%} & \textbf{p-value} \\
\midrule
Baseline (no ZT) & 0\% & - & - \\
Microsegmentazione base & 24.3\% & [21.8\%, 26.9\%] & <0.001 \\
ZT parziale (3 controlli) & 42.7\% & [39.2\%, 46.2\%] & <0.001 \\
ZT completo (6 controlli) & 67.8\% & [64.1\%, 71.3\%] & <0.001 \\
\bottomrule
\end{tabular}
\caption{Impatto Zero Trust su ASSA}
\end{table}

\subsection{\texorpdfstring{Analisi Latenza}{C.2.2 - Analisi Latenza}}

\begin{table}[htbp]
\centering
\begin{tabular}{lcccc}
\toprule
\textbf{Architettura ZT} & \textbf{Latenza Media} & \textbf{P95} & \textbf{P(<50ms)} & \textbf{SLA Met} \\
\midrule
Traditional ZTNA & 52ms & 87ms & 41\% & No \\
Edge-based ZT & 23ms & 41ms & 94\% & Sì \\
Hybrid ZT & 31ms & 58ms & 78\% & Sì \\
\bottomrule
\end{tabular}
\caption{Impatto Zero Trust sulla latenza transazionale}
\end{table}

\textbf{Conclusione}: H2 validata. Zero Trust riduce ASSA del 42.7\% mantenendo latenza <50ms nel 94\% dei casi con architettura edge-based.

\section{\texorpdfstring{Validazione Ipotesi H3 - Compliance Integrata}{C.3 - Validazione Ipotesi H3 - Compliance Integrata}}

\subsection{\texorpdfstring{Analisi Overlap Requisiti}{C.3.1 - Analisi Overlap Requisiti}}

\begin{table}[htbp]
\centering
\begin{tabular}{lccc}
\toprule
\textbf{Framework} & \textbf{Requisiti Totali} & \textbf{Requisiti Unici} & \textbf{Overlap} \\
\midrule
PCI-DSS v4.0 & 387 & 142 (36.7\%) & 63.3\% \\
GDPR & 173 & 67 (38.7\%) & 61.3\% \\
NIS2 & 329 & 103 (31.3\%) & 68.7\% \\
\textbf{Totale Integrato} & \textbf{889} & \textbf{312 (35.1\%)} & \textbf{64.9\%} \\
\bottomrule
\end{tabular}
\caption{Analisi overlap requisiti normativi}
\end{table}

\subsection{\texorpdfstring{Benefici Economici}{C.3.2 - Benefici Economici}}

\begin{table}[htbp]
\centering
\begin{tabular}{lcccc}
\toprule
\textbf{Metrica} & \textbf{Approccio Silos} & \textbf{Integrato} & \textbf{Beneficio} & \textbf{p-value} \\
\midrule
Costo implementazione (k€) & 1080 ± 124 & 673 ± 87 & -37.8\% & <0.001 \\
Effort (person-months) & 142 ± 18 & 84 ± 11 & -41.2\% & <0.001 \\
Tempo implementazione & 18 mesi & 11 mesi & -38.9\% & <0.001 \\
ROI 24 mesi & 145\% & 287\% & +97.9\% & <0.001 \\
\bottomrule
\end{tabular}
\caption{Confronto economico approcci compliance}
\end{table}

\textbf{Conclusione}: H3 validata. L'approccio integrato riduce costi del 37.8\% e effort del 41.2\% con ROI a 24 mesi del 287\%.

\section{\texorpdfstring{Validazione Framework GIST}{C.4 - Validazione Framework GIST}}

\subsection{\texorpdfstring{Distribuzione Score nel Campione}{C.4.1 - Distribuzione Score nel Campione}}

\begin{table}[htbp]
\centering
\begin{tabular}{lccccc}
\toprule
\textbf{Componente} & \textbf{P25} & \textbf{Mediana} & \textbf{P75} & \textbf{Media} & \textbf{Std} \\
\midrule
Physical (P) & 0.42 & 0.58 & 0.71 & 0.57 & 0.18 \\
Architectural (A) & 0.38 & 0.52 & 0.68 & 0.53 & 0.19 \\
Security (S) & 0.45 & 0.59 & 0.72 & 0.59 & 0.17 \\
Compliance (C) & 0.41 & 0.54 & 0.69 & 0.55 & 0.18 \\
\textbf{GIST Totale} & \textbf{41.2} & \textbf{56.8} & \textbf{69.4} & \textbf{55.7} & \textbf{14.3} \\
\bottomrule
\end{tabular}
\caption{Distribuzione score GIST (n=156 organizzazioni)}
\end{table}

\subsection{\texorpdfstring{Effetti Sinergici}{C.4.2 - Effetti Sinergici}}

\begin{table}[htbp]
\centering
\begin{tabular}{lcc}
\toprule
\textbf{Sinergia} & \textbf{Amplificazione} & \textbf{Significatività} \\
\midrule
Physical → Architectural & +27\% & p < 0.001 \\
Architectural → Security & +34\% & p < 0.001 \\
Security → Compliance & +41\% & p < 0.001 \\
\textbf{Sistema Totale} & \textbf{+52\%} & p < 0.001 \\
\bottomrule
\end{tabular}
\caption{Effetti sinergici oltre la somma lineare delle componenti}
\end{table}

\subsection{\texorpdfstring{Correlazione con Outcome Business}{C.4.3 - Correlazione con Outcome Business}}

\begin{table}[htbp]
\centering
\begin{tabular}{lcc}
\toprule
\textbf{Outcome} & \textbf{Correlazione con GIST} & \textbf{p-value} \\
\midrule
Riduzione incidenti sicurezza & -0.72 & <0.001 \\
Miglioramento availability & 0.68 & <0.001 \\
Riduzione TCO & -0.61 & <0.001 \\
Velocità time-to-market & 0.74 & <0.001 \\
Customer satisfaction & 0.53 & <0.01 \\
\bottomrule
\end{tabular}
\caption{Validazione predittiva framework GIST}
\end{table}

% APPENDICE D - GLOSSARIO
\chapter{\texorpdfstring{Glossario e Acronimi}{Appendice D - Glossario e Acronimi}}

\section{\texorpdfstring{Acronimi Principali}{D.1 - Acronimi Principali}}

\begin{tabular}{ll}
\textbf{Acronimo} & \textbf{Significato} \\
\hline
ASSA & Aggregated System Surface Attack \\
CI & Confidence Interval (Intervallo di Confidenza) \\
GIST & Governance-Infrastructure-Security-Technology \\
GDO & Grande Distribuzione Organizzata \\
GDPR & General Data Protection Regulation \\
IC & Intervallo di Confidenza \\
MTBF & Mean Time Between Failures \\
MTTR & Mean Time To Repair \\
NIS2 & Network and Information Security Directive 2 \\
NPV & Net Present Value \\
OPEX & Operational Expenditure \\
PCI-DSS & Payment Card Industry Data Security Standard \\
POS & Point of Sale \\
PUE & Power Usage Effectiveness \\
ROI & Return on Investment \\
SD-WAN & Software-Defined Wide Area Network \\
SIEM & Security Information and Event Management \\
SLA & Service Level Agreement \\
TCO & Total Cost of Ownership \\
ZT & Zero Trust \\
ZTNA & Zero Trust Network Access \\
\end{tabular}

\section{\texorpdfstring{Definizioni Essenziali}{D.2 - Definizioni Essenziali}}

\textbf{Betweenness Centrality}: Misura di centralità in teoria dei grafi che quantifica quanti cammini minimi passano attraverso un nodo.

\textbf{Framework GIST}: Modello proprietario sviluppato in questa ricerca per la valutazione olistica delle infrastrutture IT nella GDO, basato su quattro componenti principali.

\textbf{Monte Carlo}: Metodo computazionale che utilizza campionamento casuale ripetuto per ottenere risultati numerici in presenza di incertezza.

\textbf{Set Covering Problem}: Problema di ottimizzazione combinatoria NP-completo utilizzato per minimizzare i controlli necessari alla compliance multi-framework.

\textbf{Weibull Distribution}: Distribuzione di probabilità utilizzata per modellare i tempi di guasto dei componenti hardware.

\textbf{Zero Trust}: Paradigma di sicurezza che elimina il concetto di trust implicito richiedendo verifica continua di ogni transazione.

% FINE APPENDICI