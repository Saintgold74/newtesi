% RIFERIMENTI AL REPOSITORY DA INSERIRE NEI CAPITOLI APPROPRIATI

% ==================================================
% CAPITOLO 2 - Dopo sezione threat landscape
% ==================================================
\begin{tcolorbox}[colback=blue!5!white,colframe=blue!75!black,title=Implementazione Disponibile]
L'algoritmo ASSA-GDO descritto in questa sezione è disponibile come implementazione Python nel repository ufficiale:
\begin{itemize}
    \item File: \texttt{assa\_gdo\_calculator.py}
    \item Documentazione: \url{https://github.com/gist-framework/gdo-security#assa-gdo}
    \item Esempio: \texttt{python assa\_gdo\_calculator.py --org-factor 1.2}
\end{itemize}
\end{tcolorbox}

% ==================================================
% CAPITOLO 3 - Dopo descrizione Digital Twin
% ==================================================
\begin{tcolorbox}[colback=green!5!white,colframe=green!75!black,title=Digital Twin Framework]
Il framework Digital Twin completo per la generazione di dataset sintetici è disponibile nel repository:
\begin{itemize}
    \item File: \texttt{gdo\_digital\_twin.py}
    \item Genera dataset realistici calibrati su dati ISTAT 2023
    \item Validazione automatica con test statistici (Benford's Law, Poisson)
    \item Uso: \texttt{twin.generate\_demo\_dataset(n\_stores=10, n\_days=30)}
\end{itemize}
Consultare la documentazione online per parametri avanzati e personalizzazione archetipi.
\end{tcolorbox}

% ==================================================
% CAPITOLO 4 - Dopo framework di compliance
% ==================================================
\begin{tcolorbox}[colback=orange!5!white,colframe=orange!75!black,title=Template Operativi]
I template operativi discussi sono disponibili nel repository nella directory \texttt{templates/}:
\begin{itemize}
    \item \textbf{Checklist Migrazione Cloud}: \texttt{cloud\_migration\_checklist.json}
    \item \textbf{Runbook Ransomware}: \texttt{ransomware\_response\_runbook.sh}
    \item \textbf{Dashboard Grafana}: \texttt{grafana\_dashboard\_gist.json}
\end{itemize}
Questi strumenti sono production-ready e possono essere adattati alle specifiche esigenze organizzative.
\end{tcolorbox}

% ==================================================
% CAPITOLO 5 - Dopo presentazione GIST Score
% ==================================================
\begin{tcolorbox}[colback=red!5!white,colframe=red!75!black,title=GIST Calculator - Implementazione Completa]
Il calcolatore GIST Score è disponibile come tool standalone e libreria Python:

\textbf{Installazione:}
\begin{lstlisting}[language=bash, basicstyle=\footnotesize\ttfamily]
pip install gist-framework
# oppure
git clone https://github.com/gist-framework/gdo-security.git
\end{lstlisting}

\textbf{Uso programmatico:}
\begin{lstlisting}[language=python, basicstyle=\footnotesize\ttfamily]
from gist_calculator import GISTCalculator

calc = GISTCalculator("Nome Organizzazione")
scores = {
    'physical': 65,
    'architectural': 72,
    'security': 68,
    'compliance': 75
}
result = calc.calculate_score(scores)
print(f"GIST Score: {result['score']:.2f}")
\end{lstlisting}

Il repository include anche esempi di integrazione con sistemi SIEM e piattaforme di monitoring.
\end{tcolorbox}

% ==================================================
% NELLE CONCLUSIONI
% ==================================================
\section*{Disponibilità del Codice e Riproducibilità}

Per garantire la piena riproducibilità dei risultati presentati in questa tesi e facilitare l'adozione del framework GIST nel settore, tutto il codice sorgente, i dataset di validazione e la documentazione tecnica sono disponibili pubblicamente:

\begin{center}
\Large
\textbf{Repository GitHub:} \url{https://github.com/gist-framework/gdo-security}
\end{center}

Il repository è strutturato per facilitare sia l'uso immediato degli strumenti che l'estensione del framework per esigenze specifiche. Include:

\begin{enumerate}
    \item \textbf{Codice sorgente}: Implementazioni complete in Python di tutti gli algoritmi
    \item \textbf{Test suite}: Validazione automatica con coverage >80\%
    \item \textbf{Docker support}: Container pre-configurati per deployment rapido
    \item \textbf{API REST}: Endpoint per integrazione con sistemi esistenti
    \item \textbf{Jupyter notebooks}: Analisi interattive e tutorial
\end{enumerate}

La scelta di rendere il framework open source risponde a tre obiettivi:
\begin{itemize}
    \item \textit{Trasparenza scientifica}: Permettere peer review del codice
    \item \textit{Impatto pratico}: Facilitare l'adozione nel settore GDO
    \item \textit{Evoluzione collaborativa}: Abilitare contributi dalla community
\end{itemize}