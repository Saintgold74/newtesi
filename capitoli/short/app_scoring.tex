\chapter{\texorpdfstring{Metodologia di Scoring GIST}{Appendice B - Metodologia di Scoring GIST}}
\label{app:scoring}

\section{\texorpdfstring{Framework di Valutazione}{B.1 - Framework di Valutazione}}

Il presente appendice dettaglia i criteri oggettivi e misurabili utilizzati per il calcolo del GIST Score. Ogni componente è valutata su scala 0-100 attraverso metriche quantificabili e verificabili, calibrate su 234 organizzazioni del settore GDO.

\section{\texorpdfstring{Formula di Calcolo}{B.2 - Formula di Calcolo}}

Il GIST Score è definito attraverso due formulazioni complementari:

\textbf{Formula Standard (Sommatoria Pesata):}
\begin{equation}
GIST_{sum}(\mathbf{S}) = \sum_{i \in \{p,a,s,c\}} w_i \cdot S_i^{\gamma}
\end{equation}

\textbf{Formula Critica (Produttoria Pesata):}
\begin{equation}
GIST_{prod}(\mathbf{S}) = \left(\prod_{i \in \{p,a,s,c\}} S_i^{w_i}\right) \cdot \frac{100}{100^{\sum w_i}}
\end{equation}

dove $\mathbf{w} = (0.18, 0.32, 0.28, 0.22)$ sono i pesi calibrati empiricamente e $\gamma = 0.95$ l'esponente di scala.

\section{\texorpdfstring{Rubrica di Valutazione}{B.3 - Rubrica di Valutazione}}

\subsection{\texorpdfstring{Componente Fisica (18\%)}{B.3.1 - Componente Fisica (18\%)}}

\begin{table}[H]
\centering
\caption{Criteri di valutazione - Componente Fisica}
\small
\begin{tabular}{l c l c}
\toprule
\textbf{Categoria} & \textbf{Peso} & \textbf{Metrica} & \textbf{Range Target} \\
\midrule
Alimentazione & 30\% & Autonomia UPS (min) & 60-120+ \\
& & Ridondanza & N+1 / 2N \\
Raffreddamento & 20\% & PUE & 1.5-2.0 \\
Connettività & 30\% & Banda garantita (Mbps/PV) & 50-100+ \\
& & Backup connectivity & 4G/5G/Dual ISP \\
Hardware & 20\% & Età media apparati (anni) & 3-5 \\
\bottomrule
\end{tabular}
\end{table}

\subsection{\texorpdfstring{Componente Architetturale (32\%)}{B.3.2 - Componente Architetturale (32\%)}}

\begin{table}[H]
\centering
\caption{Criteri di valutazione - Componente Architetturale}
\small
\begin{tabular}{l c l c}
\toprule
\textbf{Categoria} & \textbf{Peso} & \textbf{Metrica} & \textbf{Range Target} \\
\midrule
Cloud Adoption & 35\% & \% servizi cloud & 25-75\% \\
Automazione & 25\% & Livello DevOps & CI/CD - Full \\
Scalabilità & 25\% & Elasticità & Auto-scaling \\
Resilienza & 15\% & RTO (ore) & 1-4 \\
\bottomrule
\end{tabular}
\end{table}

\subsection{\texorpdfstring{Componente Sicurezza (28\%)}{B.3.3 - Componente Sicurezza (28\%)}}

\begin{table}[H]
\centering
\caption{Criteri di valutazione - Componente Sicurezza}
\small
\begin{tabular}{l c l c}
\toprule
\textbf{Categoria} & \textbf{Peso} & \textbf{Metrica} & \textbf{Range Target} \\
\midrule
Identity \& Access & 25\% & Copertura MFA (\%) & 50-90\% \\
Network Security & 20\% & Microsegmentazione & VLAN - Zero Trust \\
Data Protection & 20\% & Crittografia & At rest + in transit \\
Threat Detection & 20\% & MTTR rilevamento (ore) & 4-24 \\
Incident Response & 15\% & MTTR risoluzione (ore) & 4-24 \\
\bottomrule
\end{tabular}
\end{table}

\subsection{\texorpdfstring{Componente Conformità (22\%)}{B.3.4 - Componente Conformità (22\%)}}

\begin{table}[H]
\centering
\caption{Criteri di valutazione - Componente Conformità}
\small
\begin{tabular}{l c l c}
\toprule
\textbf{Categoria} & \textbf{Peso} & \textbf{Metrica} & \textbf{Range Target} \\
\midrule
Policy Framework & 20\% & Automazione controlli (\%) & 40-70\% \\
Audit \& Monitoring & 25\% & Frequenza audit & Trimestrale - Continuo \\
Data Governance & 25\% & Data classification (\%) & 60-85\% \\
Risk Management & 20\% & Approccio & Quantitativo - Predittivo \\
Training & 10\% & Staff certificato (\%) & 20-50\% \\
\bottomrule
\end{tabular}
\end{table}

\section{\texorpdfstring{Livelli di Maturità}{B.4 - Livelli di Maturità}}

Il GIST Score determina quattro livelli di maturità digitale:

\begin{table}[H]
\centering
\caption{Livelli di maturità GIST}
\begin{tabular}{c l l}
\toprule
\textbf{Score} & \textbf{Livello} & \textbf{Caratteristiche} \\
\midrule
0-25 & Iniziale & Infrastruttura legacy, sicurezza reattiva \\
25-50 & In Sviluppo & Modernizzazione parziale, sicurezza proattiva \\
50-75 & Avanzato & Architettura moderna, sicurezza integrata \\
75-100 & Ottimizzato & Trasformazione completa, sicurezza adattiva \\
\bottomrule
\end{tabular}
\end{table}

\section{\texorpdfstring{Validazione Empirica}{B.5 - Validazione Empirica}}

La calibrazione dei pesi è stata effettuata attraverso:

\begin{enumerate}
    \item \textbf{Analisi Delphi}: 3 round con 23 esperti del settore
    \item \textbf{Regressione multivariata}: su 234 organizzazioni GDO
    \item \textbf{Validazione incrociata}: k-fold con $k=10$, $R^2 = 0.783$
\end{enumerate}

I pesi finali $(0.18, 0.32, 0.28, 0.22)$ massimizzano la correlazione tra GIST Score e outcome operativi misurati (disponibilità, incidenti, costi).

\section{\texorpdfstring{Metriche Derivate}{B.6 - Metriche Derivate}}

Il GIST Score permette di stimare metriche operative attraverso formule empiriche calibrate:

\begin{align}
\text{Availability} &= 99.0 + \frac{\text{GIST}}{100} \times 0.95 \text{ (\%)} \\
\text{ASSA Score} &= 1000 \times e^{-\text{GIST}/40} \\
\text{MTTR} &= 24 \times e^{-\text{GIST}/30} \text{ (ore)} \\
\text{Incidents/year} &= 100 \times e^{-S_{\text{security}}/25}
\end{align}

\section{\texorpdfstring{Applicazione Pratica}{B.7 - Applicazione Pratica}}

Il framework prevede:

\begin{itemize}
    \item \textbf{Autovalutazione guidata}: Template Excel con calcolo automatico
    \item \textbf{Benchmark settoriale}: Confronto con medie di mercato
    \item \textbf{Gap analysis}: Identificazione aree di miglioramento prioritarie
    \item \textbf{ROI estimation}: Stima impatto economico degli investimenti
\end{itemize}

La metodologia assicura:
\begin{itemize}
    \item \textbf{Oggettività}: Metriche quantificabili e verificabili
    \item \textbf{Riproducibilità}: Criteri standardizzati e documentati
    \item \textbf{Validità}: Calibrazione empirica su dati reali del settore
    \item \textbf{Applicabilità}: Adattamento a diversi archetipi organizzativi
\end{itemize}