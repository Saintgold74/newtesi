\chapter*{Prefazione}
\addcontentsline{toc}{chapter}{Prefazione}

\begin{em} % Testo in corsivo come da regolamento

Il presente lavoro di tesi nasce dall'esigenza di affrontare le sfide moderne nella gestione delle reti di dati, 
con particolare attenzione all'innovazione metodologica e all'ottimizzazione delle architetture distribuite.

Durante il percorso di ricerca, ho avuto l'opportunità di approfondire non solo gli aspetti teorici 
fondamentali, ma anche di sviluppare soluzioni pratiche e innovative che possano rispondere alle 
esigenze concrete del settore.

Desidero ringraziare il Professor Chiar.mo Giovanni Farina per la guida costante e i preziosi consigli 
forniti durante tutto il percorso di ricerca, ed insieme a lui anche a tutti gli altri professori e assistenti che mi hanno accompagnato in questo percorso.
Un ringraziamento particolare va anche ai colleghi ed amici che mi hanno supportato,ed incoraggiato in questa non semplice avventura accademica.

Un pensiero speciale va alla mia compagnia di vita, Laura, per la pazienza e il sostegno incondizionato, dimostrando ancora una volta, se ce ne fosse bisogno, che {\emph{"dietro ogni grande uomo c'è una grande donna"}}.

Questo lavoro rappresenta non solo il culmine del mio percorso universitario, ma anche il punto di partenza per future ricerche nel campo dell' Ingegneria Informatica e della Sicurezza Informatica.

\end{em}

\vspace{2cm}
\begin{flushright}
\textit{Il Candidato}\\
\textit{Marco Santoro}
\end{flushright}