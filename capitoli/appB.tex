\chapter{Dataset e Analisi Statistiche Supplementari}
\label{app:dataset}

\section{B.1 Struttura del Dataset GDO-Bench}

Il dataset GDO-Bench, sviluppato per questa ricerca e reso disponibile alla comunità scientifica, comprende 24 mesi di dati simulati ma realisticamente calibrati per 50 punti vendita virtuali.

\subsection{B.1.1 Schema dei Dati}

\begin{table}[htbp]
\centering
\caption{Schema principale del dataset GDO-Bench}
\begin{tabular}{|l|l|l|p{5cm}|}
\hline
\textbf{Tabella} & \textbf{Record} & \textbf{Dimensione} & \textbf{Descrizione} \\
\hline
transactions & 112M & 45 GB & Transazioni POS con timestamp, importo, metodo pagamento \\
network\_traffic & 2.3B & 180 GB & Flussi di rete aggregati (NetFlow format) \\
security\_events & 14M & 8 GB & Eventi da SIEM, IDS/IPS, firewall \\
performance\_metrics & 420M & 22 GB & Metriche di sistema (CPU, RAM, I/O, latenza) \\
inventory\_movements & 89M & 15 GB & Movimenti di magazzino e giacenze \\
incidents & 3,847 & 120 MB & Incidenti documentati con RCA \\
compliance\_audits & 156 & 45 MB & Report di audit e non conformità \\
\hline
\textbf{Totale} & & \textbf{270.2 GB} & \\
\hline
\end{tabular}
\end{table}

\subsection{B.1.2 Generazione dei Dati Sintetici}

I dati sono stati generati utilizzando modelli statistici calibrati su pattern reali:

\textbf{Generazione delle transazioni:}
\begin{equation}
\lambda(t) = \lambda_0 \cdot \left(1 + A \sin\left(\frac{2\pi t}{T_{day}}\right)\right) \cdot S(w) \cdot H(d)
\end{equation}

dove:
\begin{itemize}
    \item $\lambda_0 = 2.3$ transazioni/minuto (rate base)
    \item $A = 0.7$ (ampiezza variazione intraday)
    \item $T_{day} = 1440$ minuti
    \item $S(w)$ = fattore settimanale (lunedì=0.8, sabato=1.5)
    \item $H(d)$ = fattore festività (normale=1.0, Natale=2.3)
\end{itemize}

\section{B.2 Analisi della Superficie di Attacco}

\subsection{B.2.1 Calcolo Dettagliato ASSA-GDO}

L'analisi della superficie di attacco per le 47 organizzazioni monitorate ha prodotto i seguenti risultati:

\begin{table}[htbp]
\centering
\caption{Statistiche ASSA-GDO per categoria di organizzazione}
\begin{tabular}{|l|c|c|c|c|}
\hline
\textbf{Categoria} & \textbf{N} & \textbf{ASSA Medio} & \textbf{Dev.Std} & \textbf{Range} \\
\hline
Supermercati & 18 & 847.3 & 124.5 & 623-1,142 \\
Discount & 12 & 523.7 & 89.2 & 401-698 \\
Specializzati & 9 & 687.2 & 102.3 & 512-891 \\
Ipermercati & 8 & 1,234.5 & 187.6 & 987-1,567 \\
\hline
\textbf{Totale} & 47 & 798.4 & 234.7 & 401-1,567 \\
\hline
\end{tabular}
\end{table}

\subsection{B.2.2 Analisi delle Componenti Principali}

L'analisi PCA sulla matrice di vulnerabilità ha identificato 4 componenti che spiegano l'82.3% della varianza:

\begin{enumerate}
    \item \textbf{PC1 (34.2\%)}: Complessità infrastrutturale
    \item \textbf{PC2 (23.7\%)}: Esposizione esterna
    \item \textbf{PC3 (15.8\%)}: Maturità dei processi
    \item \textbf{PC4 (8.6\%)}: Fattore umano
\end{enumerate}

\section{B.3 Risultati delle Simulazioni Monte Carlo}

\subsection{B.3.1 Convergenza delle Simulazioni}

La convergenza è stata verificata utilizzando il criterio di Gelman-Rubin:

\begin{equation}
\hat{R} = \sqrt{\frac{\text{Var}(\psi|y)}{W}}
\end{equation}

dove $W$ è la varianza within-chain e $\text{Var}(\psi|y)$ è la stima della varianza marginale posteriore.

\textbf{Risultati di convergenza:}
\begin{itemize}
    \item Disponibilità: $\hat{R} = 1.03$ (convergenza a 3,000 iterazioni)
    \item TCO: $\hat{R} = 1.05$ (convergenza a 4,500 iterazioni)
    \item ASSA: $\hat{R} = 1.02$ (convergenza a 2,800 iterazioni)
    \item Compliance Score: $\hat{R} = 1.04$ (convergenza a 3,200 iterazioni)
\end{itemize}

\subsection{B.3.2 Analisi di Sensitività}

L'analisi di sensitività basata sugli indici di Sobol ha identificato i parametri più influenti:

\begin{table}[htbp]
\centering
\caption{Indici di Sobol per le metriche principali}
\begin{tabular}{|l|c|c|c|}
\hline
\textbf{Parametro} & \textbf{S1 (Main)} & \textbf{ST (Total)} & \textbf{Ranking} \\
\hline
Budget sicurezza & 0.287 & 0.412 & 1 \\
Maturità processi & 0.234 & 0.367 & 2 \\
Architettura (cloud \%) & 0.198 & 0.289 & 3 \\
Turnover personale & 0.156 & 0.234 & 4 \\
Complessità legacy & 0.089 & 0.145 & 5 \\
Altri (12 parametri) & 0.036 & 0.098 & - \\
\hline
\end{tabular}
\end{table}

\section{B.4 Validazione dei Modelli Predittivi}

\subsection{B.4.1 Metriche di Performance}

I modelli predittivi sono stati validati utilizzando cross-validation 10-fold:

\begin{table}[htbp]
\centering
\caption{Performance dei modelli predittivi}
\begin{tabular}{|l|c|c|c|c|}
\hline
\textbf{Modello} & \textbf{R²} & \textbf{RMSE} & \textbf{MAE} & \textbf{MAPE} \\
\hline
Disponibilità & 0.873 & 0.24\% & 0.18\% & 0.19\% \\
TCO & 0.812 & €124k & €89k & 8.7\% \\
Tempo incidente & 0.794 & 3.2 giorni & 2.4 giorni & 14.3\% \\
Compliance score & 0.856 & 4.3 punti & 3.1 punti & 5.2\% \\
\hline
\end{tabular}
\end{table}
