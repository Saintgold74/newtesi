
\chapter{\texorpdfstring{Template e Strumenti Operativi}{Appendice D - Template e Strumenti Operativi}}
\label{app:template}

\section{\texorpdfstring{Template Assessment Infrastrutturale}{D.1 - Template Assessment Infrastrutturale}}

\subsection{\texorpdfstring{Checklist Pre-Migrazione Cloud}{D.1.1 - Checklist Pre-Migrazione Cloud}}

\begin{table}[htbp]
\centering
\caption{Checklist di valutazione readiness per migrazione cloud}
\begin{tabular}{|p{6cm}|c|c|p{4cm}|}
\hline
\textbf{Area di Valutazione} & \textbf{Critico} & \textbf{Status} & \textbf{Note} \\
\hline
\multicolumn{4}{|l|}{\textbf{1. Infrastruttura Fisica}} \\
\hline
Banda disponibile per sede $\geq$ 100 Mbps & Sì & $\square$ & \\
\hline
Connettività ridondante (2+ carrier) & Sì & $\square$ & \\
\hline
Latenza verso cloud provider < 50ms & Sì & $\square$ & \\
\hline
Power backup minimo 4 ore & No & $\square$ & \\
\hline
\multicolumn{4}{|l|}{\textbf{2. Applicazioni}} \\
\hline
Inventory applicazioni completo & Sì & $\square$ & \\
\hline
Dipendenze mappate & Sì & $\square$ & \\
\hline
Licensing cloud-compatible & Sì & $\square$ & \\
\hline
Test di compatibilità eseguiti & No & $\square$ & \\
\hline
\multicolumn{4}{|l|}{\textbf{3. Dati}} \\
\hline
Classificazione dati completata & Sì & $\square$ & \\
\hline
Volume dati da migrare quantificato & Sì & $\square$ & \\
\hline
RPO/RTO definiti per applicazione & Sì & $\square$ & \\
\hline
Strategia di backup cloud-ready & Sì & $\square$ & \\
\hline
\multicolumn{4}{|l|}{\textbf{4. Sicurezza}} \\
\hline
Politiche di accesso cloud definite & Sì & $\square$ & \\
\hline
MFA implementato per admin & Sì & $\square$ & \\
\hline
Crittografia at-rest configurabile & Sì & $\square$ & \\
\hline
Network segmentation plan & No & $\square$ & \\
\hline
\multicolumn{4}{|l|}{\textbf{5. Competenze}} \\
\hline
Team cloud certificato (min 2 persone) & Sì & $\square$ & \\
\hline
Piano di formazione definito & No & $\square$ & \\
\hline
Supporto vendor contrattualizzato & No & $\square$ & \\
\hline
Runbook operativi preparati & Sì & $\square$ & \\
\hline
\end{tabular}
\end{table}

\section{\texorpdfstring{Matrice di Integrazione Normativa}{D.2 - Matrice di Integrazione Normativa}}

\subsection{\texorpdfstring{Template di Controllo Unificato}{D.2.1 - Template di Controllo Unificato}}

\begin{tcolorbox}[
    colback=blue!5!white,
    colframe=blue!75!black,
    title={\textbf{Controllo Unificato CU-001: Gestione Accessi Privilegiati}},
    fonttitle=\bfseries,
    boxrule=1.5pt,
    arc=2mm,
    %breakable
]
\textbf{Requisiti Soddisfatti:}
\begin{itemize}
    \item PCI-DSS 4.0: 7.2, 8.2.3, 8.3.1
    \item GDPR: Art. 32(1)(a), Art. 25
    \item NIS2: Art. 21(2)(d)
\end{itemize}

\textbf{Implementazione Tecnica:}
\begin{enumerate}
    \item Deploy soluzione PAM (CyberArk/HashiCorp Vault)
    \item Configurazione politiche:
    \begin{itemize}
        \item Rotazione password ogni 30 giorni
        \item MFA obbligatorio per accessi admin
        \item Session recording per audit
        \item Approval workflow per accessi critici
    \end{itemize}
    \item Integrazione con:
    \begin{itemize}
        \item Active Directory/LDAP
        \item SIEM per monitoring
        \item Ticketing system per approval
    \end{itemize}
\end{enumerate}

\textbf{Metriche di Conformità:}
\begin{itemize}
    \item \% account privilegiati sotto PAM: Target 100\%
    \item Tempo medio approvazione accessi: < 15 minuti
    \item Password rotation compliance: > 99\%
    \item Failed access attempts: < 1\%
\end{itemize}

\textbf{Evidenze per Audit:}
\begin{itemize}
    \item Report mensile accessi privilegiati
    \item Log di tutte le sessioni privilegiate
    \item Attestazione trimestrale dei privilegi
    \item Recording video sessioni critiche
\end{itemize}

\textbf{Costo Stimato:}
\begin{itemize}
    \item Licenze software: €45k/anno (500 utenti)
    \item Implementazione: €25k (una tantum)
    \item Manutenzione: €8k/anno
    \item Training: €5k (iniziale)
\end{itemize}

\textbf{ROI:}
\begin{itemize}
    \item Riduzione audit effort: -30\% (€15k/anno)
    \item Riduzione incidenti privileged access: -70\% (€50k/anno)
    \item Payback period: 14 mesi
\end{itemize}
\end{tcolorbox}

\section{\texorpdfstring{Runbook Operativi}{D.3 - Runbook Operativi}}

\subsection{\texorpdfstring{Procedura Risposta Incidenti - Ransomware}{D.3.1 - Procedura Risposta Incidenti - Ransomware}}

\begin{lstlisting}[language=bash, caption=Runbook automatizzato per contenimento ransomware]
#!/bin/bash
# Runbook: Contenimento Ransomware GDO
# Versione: 2.0
# Ultimo aggiornamento: 2025-01-15

set -euo pipefail

# Configurazione
INCIDENT_ID=$(date +%Y%m%d%H%M%S)
LOG_DIR="/var/log/incidents/${INCIDENT_ID}"
SIEM_API="https://siem.internal/api/v1"
NETWORK_CONTROLLER="https://sdn.internal/api"

# Funzioni di utilità
log() {
    echo "[$(date +'%Y-%m-%d %H:%M:%S')] $1" | tee -a "${LOG_DIR}/incident.log"
}

alert_team() {
    # Invia alert al team
    curl -X POST https://slack.internal/webhook \
        -d "{\"text\": \"SECURITY ALERT: $1\"}"
}

# STEP 1: Identificazione e Isolamento
isolate_affected_systems() {
    log "STEP 1: Iniziando isolamento sistemi affetti"
    
    # Query SIEM per sistemi con indicatori ransomware
    AFFECTED_SYSTEMS=$(curl -s "${SIEM_API}/query" \
        -d '{"query": "event.type:ransomware_indicator", "last": "1h"}' \
        | jq -r '.results[].host')
    
    for system in ${AFFECTED_SYSTEMS}; do
        log "Isolando sistema: ${system}"
        
        # Isolamento network via SDN
        curl -X POST "${NETWORK_CONTROLLER}/isolate" \
            -d "{\"host\": \"${system}\", \"vlan\": \"quarantine\"}"
        
        # Disable account AD
        ldapmodify -x -D "cn=admin,dc=gdo,dc=local" -w "${LDAP_PASS}" <<EOF
dn: cn=${system},ou=computers,dc=gdo,dc=local
changetype: modify
replace: userAccountControl
userAccountControl: 514
EOF
        
        # Snapshot VM se virtualizzato
        if vmware-cmd -l | grep -q "${system}"; then
            vmware-cmd "${system}" create-snapshot "pre-incident-${INCIDENT_ID}"
        fi
    done
    
    echo "${AFFECTED_SYSTEMS}" > "${LOG_DIR}/affected_systems.txt"
    alert_team "Isolati ${#AFFECTED_SYSTEMS[@]} sistemi"
}

# STEP 2: Contenimento della Propagazione  
contain_lateral_movement() {
    log "STEP 2: Contenimento movimento laterale"
    
    # Blocco SMB su tutti i segmenti non critici
    for vlan in $(seq 100 150); do
        curl -X POST "${NETWORK_CONTROLLER}/acl/add" \
            -d "{\"vlan\": ${vlan}, \"rule\": \"deny tcp any any eq 445\"}"
    done
    
    # Reset password account di servizio
    for account in $(cat /etc/security/service_accounts.txt); do
        NEW_PASS=$(openssl rand -base64 32)
        ldappasswd -x -D "cn=admin,dc=gdo,dc=local" -w "${LDAP_PASS}" \
            -s "${NEW_PASS}" "cn=${account},ou=service,dc=gdo,dc=local"
        
        # Salva in vault
        vault kv put secret/incident/${INCIDENT_ID}/${account} password="${NEW_PASS}"
    done
    
    # Kill processi sospetti
    SUSPICIOUS_PROCS=$(osquery --json \
        "SELECT * FROM processes WHERE 
         (name LIKE '%crypt%' OR name LIKE '%lock%') 
         AND start_time > datetime('now', '-1 hour')")
    
    echo "${SUSPICIOUS_PROCS}" | jq -r '.[]|.pid' | while read pid; do
        kill -9 ${pid} 2>/dev/null || true
    done
}

# STEP 3: Identificazione del Vettore
identify_attack_vector() {
    log "STEP 3: Identificazione vettore di attacco"
    
    # Analisi email phishing ultimi 7 giorni
    PHISHING_CANDIDATES=$(curl -s "${SIEM_API}/email/suspicious" \
        -d '{"days": 7, "min_score": 7}')
    
    echo "${PHISHING_CANDIDATES}" > "${LOG_DIR}/phishing_analysis.json"
    
    # Check vulnerabilità note non patchate
    for system in $(cat "${LOG_DIR}/affected_systems.txt"); do
        nmap -sV --script vulners "${system}" > "${LOG_DIR}/vuln_scan_${system}.txt"
    done
    
    # Analisi log RDP/SSH per accessi anomali
    grep -E "(Failed|Accepted)" /var/log/auth.log | \
        awk '{print $1, $2, $3, $9, $11}' | \
        sort | uniq -c | sort -rn > "${LOG_DIR}/access_analysis.txt"
}

# STEP 4: Preservazione delle Evidenze
preserve_evidence() {
    log "STEP 4: Preservazione evidenze forensi"
    
    for system in $(cat "${LOG_DIR}/affected_systems.txt"); do
        # Dump memoria se accessibile
        if ping -c 1 ${system} &>/dev/null; then
            ssh forensics@${system} "sudo dd if=/dev/mem of=/tmp/mem.dump"
            scp forensics@${system}:/tmp/mem.dump "${LOG_DIR}/${system}_memory.dump"
        fi
        
        # Copia log critici
        rsync -avz forensics@${system}:/var/log/ "${LOG_DIR}/${system}_logs/"
        
        # Hash per chain of custody
        find "${LOG_DIR}/${system}_logs/" -type f -exec sha256sum {} \; \
            > "${LOG_DIR}/${system}_hashes.txt"
    done
}

# STEP 5: Comunicazione e Coordinamento
coordinate_response() {
    log "STEP 5: Coordinamento risposta"
    
    # Genera report preliminare
    cat > "${LOG_DIR}/preliminary_report.md" <<EOF
# Incident Report ${INCIDENT_ID}

## Executive Summary
- Tipo: Ransomware
- Sistemi affetti: $(wc -l < "${LOG_DIR}/affected_systems.txt")
- Impatto stimato: TBD
- Status: CONTENUTO

## Timeline
$(grep "STEP" "${LOG_DIR}/incident.log")

## Sistemi Affetti
$(cat "${LOG_DIR}/affected_systems.txt")

## Prossimi Passi
1. Analisi forense completa
2. Identificazione ransomware variant
3. Valutazione opzioni recovery
4. Comunicazione stakeholder
EOF
    
    # Notifica management
    mail -s "URGENT: Ransomware Incident ${INCIDENT_ID}" \
        ciso@gdo.com security-team@gdo.com < "${LOG_DIR}/preliminary_report.md"
    
    # Apertura ticket
    curl -X POST https://servicenow.internal/api/incident \
        -d "{
            \"priority\": 1,
            \"category\": \"security\",
            \"description\": \"Ransomware containment completed\",
            \"incident_id\": \"${INCIDENT_ID}\"
        }"
}

# Main execution
main() {
    mkdir -p "${LOG_DIR}"
    log "=== Iniziando risposta incidente Ransomware ==="
    
    isolate_affected_systems
    contain_lateral_movement
    identify_attack_vector
    preserve_evidence
    coordinate_response
    
    log "=== Contenimento completato. Procedere con analisi forense ==="
}

# Esecuzione con error handling
trap 'log "ERRORE: Runbook fallito al comando $BASH_COMMAND"' ERR
main "$@"
\end{lstlisting}

\section{\texorpdfstring{Dashboard e KPI Templates}{D.4 - Dashboard e KPI Templates}}

\subsection{\texorpdfstring{GIST Score Dashboard Configuration}{D.4.1 - GIST Score Dashboard Configuration}}
\begin{lstlisting}[language=json, caption=Configurazione Grafana per GIST Score Dashboard]
{
  "dashboard": {
    "title": "GIST Framework - Security Posture Dashboard",
    "panels": [
      {
        "title": "GIST Score Trend",
        "type": "graph",
        "targets": [
          {
            "expr": "gist_total_score",
            "legendFormat": "Total Score"
          },
          {
            "expr": "gist_component_physical",  
            "legendFormat": "Physical"
          },
          {
            "expr": "gist_component_architectural",
            "legendFormat": "Architectural"  
          },
          {
            "expr": "gist_component_security",
            "legendFormat": "Security"
          },
          {
            "expr": "gist_component_compliance",
            "legendFormat": "Compliance"
          }
        ]
      },
      {
        "title": "Attack Surface (ASSA)",
        "type": "gauge",
        "targets": [
          {
            "expr": "assa_score_current",
            "thresholds": {
              "mode": "absolute",
              "steps": [
                {"value": 0, "color": "green"},
                {"value": 500, "color": "yellow"},
                {"value": 800, "color": "orange"},
                {"value": 1000, "color": "red"}
              ]
            }
          }
        ]
      },
      {
        "title": "Compliance Status",
        "type": "stat",
        "targets": [
          {
            "expr": "compliance_score_pcidss",
            "title": "PCI-DSS"
          },
          {
            "expr": "compliance_score_gdpr",
            "title": "GDPR"
          },
          {
            "expr": "compliance_score_nis2",
            "title": "NIS2"
          }
        ]
      },
      {
        "title": "Security Incidents (24h)",
        "type": "table",
        "targets": [
          {
            "expr": "security_incidents_by_severity",
            "format": "table",
            "columns": ["time", "severity", "type", "affected_systems", "status"]
          }
        ]
      },
      {
        "title": "Infrastructure Health",
        "type": "heatmap",
        "targets": [
          {
            "expr": "infrastructure_health_by_location",
            "format": "heatmap"
          }
        ]
      }
    ],
    "refresh": "30s",
    "time": {
      "from": "now-24h",
      "to": "now"
    }
  }
}
\end{lstlisting}