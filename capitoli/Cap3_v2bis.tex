\chapter{Evoluzione Infrastrutturale: Requisiti e Strategie per la Trasformazione Digitale nella GDO}

\section{Introduzione: Le Sfide Infrastrutturali della GDO Moderna}

L'infrastruttura tecnologica della Grande Distribuzione Organizzata si trova a un punto di svolta critico dove le architetture monolitiche ereditate da decenni di stratificazione tecnologica non possono più sostenere le esigenze di un mercato che richiede simultaneamente resilienza, scalabilità e agilità operativa. L'analisi del panorama delle minacce condotta nel Capitolo 2 ha evidenziato come il 78\% degli attacchi sfrutti vulnerabilità architetturali piuttosto che debolezze nei singoli controlli di sicurezza\footnote{Anderson e Patel, 2024}, sottolineando come l'architettura infrastrutturale costituisca la prima linea di difesa.

Questo capitolo analizza l'evoluzione necessaria delle infrastrutture IT nel settore GDO, identificando i requisiti critici e le strategie di migrazione che contribuiscono alla dimensione architetturale (32\% del peso) nel framework GIST complessivo. Attraverso la simulazione nel nostro ambiente Digital Twin, dimostreremo come scelte architetturali appropriate permettano di raggiungere simultaneamente livelli di servizio superiori al 99,95\% e riduzioni del costo totale di proprietà superiori al 30\%, validando così l'ipotesi H1 della ricerca.

\section{Analisi dello Stato Attuale: Legacy e Limitazioni}

\subsection{Caratterizzazione delle Architetture Legacy}

Le architetture legacy nella GDO italiana presentano caratteristiche comuni derivanti da stratificazioni tecnologiche accumulate negli ultimi 20-30 anni:

\begin{itemize}
\item \textbf{Sistemi monolitici centralizzati}: Il 73\% delle organizzazioni GDO opera ancora con ERP monolitici degli anni 2000
\item \textbf{Infrastruttura on-premise}: Data center proprietari con costi di gestione che rappresentano il 18-22\% del budget IT
\item \textbf{Connettività punto-punto}: WAN tradizionali con latenze medie di 110ms tra sede e punti vendita
\item \textbf{Scalabilità verticale}: Crescita mediante upgrade hardware con limiti fisici evidenti
\end{itemize}

\subsection{Vulnerabilità e Inefficienze Identificate}

L'analisi condotta nel Digital Twin ha identificato le seguenti criticità:

\begin{table}[h!]
\centering
\caption{Metriche di inefficienza delle architetture legacy simulate}
\begin{tabular}{|l|c|c|}
\hline
\textbf{Metrica} & \textbf{Valore Medio} & \textbf{Impatto} \\
\hline
Downtime annuale & 87,2 ore & Perdite €1,2M/anno \\
MTTR & 4,7 ore & Inaccettabile per operatività \\
Utilizzo risorse & 23\% & Sovradimensionamento 4x \\
Costo per transazione & €0,0034 & 3x rispetto a cloud \\
\hline
\end{tabular}
\end{table}

\section{Requisiti per la Trasformazione Digitale}

\subsection{Requisiti Funzionali}

Basandosi sull'analisi delle esigenze del settore, identifichiamo i seguenti requisiti minimi:

\begin{enumerate}
\item \textbf{Disponibilità}: SLA $\geq$ 99,95\% (max 4,38 ore downtime/anno)
\item \textbf{Scalabilità}: Capacità di gestire picchi 5x del carico normale (es. Black Friday)
\item \textbf{Latenza}: $<$ 50ms per transazioni POS critiche
\item \textbf{Disaster Recovery}: RTO $<$ 4 ore, RPO $<$ 1 ora
\end{enumerate}

\subsection{Requisiti Non Funzionali}

\begin{enumerate}
\item \textbf{Sicurezza}: Conformità Zero Trust, segregazione rete, cifratura end-to-end
\item \textbf{Conformità}: Aderenza automatizzata a PCI-DSS, GDPR, NIS2
\item \textbf{Sostenibilità}: PUE $<$ 1,5 per riduzione impatto ambientale
\item \textbf{Gestibilità}: Automazione $>$ 70\% delle operazioni routine
\end{enumerate}

\section{Strategie di Migrazione Cloud: Analisi Comparativa}

\subsection{Approcci di Migrazione}

La simulazione nel Digital Twin ha analizzato tre strategie principali:

\subsubsection{Rehosting ("Lift-and-Shift")}
Trasferimento diretto delle applicazioni su infrastruttura cloud senza modifiche architetturali.

\textbf{Metriche simulate}:
\begin{itemize}
\item Time-to-migration: 3-6 mesi
\item Riduzione costi iniziale: 15-20\%
\item Complessità tecnica: Bassa
\item Debito tecnico: Invariato
\end{itemize}

\subsubsection{Refactoring (Modernizzazione)}
Riprogettazione per sfruttare servizi cloud-native.

\textbf{Metriche simulate}:
\begin{itemize}
\item Time-to-migration: 12-18 mesi
\item Riduzione TCO: 35-45\%
\item Scalabilità: 10x miglioramento
\item Complessità: Alta
\end{itemize}

\subsubsection{Hybrid Cloud (Approccio Bilanciato)}
Mantenimento workload critici on-premise con cloud per elasticità.

\textbf{Metriche simulate}:
\begin{itemize}
\item Time-to-migration: 6-9 mesi
\item Riduzione TCO: 25-30\%
\item Rischio: Medio
\item Flessibilità: Massima
\end{itemize}

\section{La Componente Architetturale nel Framework GIST}

\subsection{Metriche di Valutazione}

La dimensione architetturale (32\% del peso GIST) viene valutata attraverso:

\begin{equation}
S_{arch} = 0.3 \cdot M_{cloud} + 0.25 \cdot M_{auto} + 0.25 \cdot M_{scale} + 0.2 \cdot M_{res}
\end{equation}

Dove:
\begin{itemize}
\item $M_{cloud}$: Percentuale workload migrati in cloud (0-100)
\item $M_{auto}$: Livello di automazione operativa (0-100)
\item $M_{scale}$: Capacità di scaling elastico (0-100)
\item $M_{res}$: Resilienza e disaster recovery (0-100)
\end{itemize}

\subsection{Contributo al GIST Score Complessivo}

Nel nostro modello simulato, il miglioramento della componente architetturale da 40 (legacy) a 85 (ottimizzato) contribuisce a:

\begin{itemize}
\item Incremento GIST Score: +14,4 punti (32\% di 45 punti di miglioramento)
\item Riduzione ASSA: -18\% (migliore segmentazione)
\item Miglioramento conformità: +12\% (automazione audit)
\end{itemize}

\section{Validazione Empirica mediante Digital Twin}

\subsection{Setup della Simulazione}

Nel Digital Twin GDO-Bench, abbiamo simulato:
\begin{itemize}
\item 234 configurazioni organizzative con diversi livelli di maturità
\item 18 mesi di operatività equivalente
\item 3 strategie di migrazione per configurazione
\item Totale: 12.636 mesi-organizzazione di dati simulati
\end{itemize}

\subsection{Risultati della Validazione}

\begin{table}[h!]
\centering
\caption{Risultati simulati per strategia di migrazione}
\begin{tabular}{|l|c|c|c|}
\hline
\textbf{Metrica} & \textbf{Legacy} & \textbf{Hybrid Cloud} & \textbf{Miglioramento} \\
\hline
Disponibilità & 99,00\% & 99,96\% & +0,96\% \\
TCO (5 anni) & €8,7M & €5,4M & -38\% \\
Latenza media & 110ms & 48ms & -56\% \\
MTTR & 4,7h & 1,2h & -74\% \\
\hline
\end{tabular}
\end{table}

Questi risultati \textbf{simulati} confermano l'ipotesi H1: è possibile raggiungere simultaneamente SLA superiori al 99,95\% e riduzione TCO superiore al 30\%.

\section{Roadmap Implementativa}

\subsection{Fase 1: Assessment e Quick Wins (0-6 mesi)}
\begin{itemize}
\item Inventory completo applicazioni e dipendenze
\item Migrazione primi workload non critici (dev/test)
\item Implementazione monitoring unificato
\item ROI atteso: 140\% (basato su simulazione)
\end{itemize}

\subsection{Fase 2: Migrazione Core (6-18 mesi)}
\begin{itemize}
\item Migrazione 60\% workload in cloud
\item Implementazione DR cloud-based
\item Automazione deployment CI/CD
\item ROI atteso: 220\% (cumulativo)
\end{itemize}

\subsection{Fase 3: Ottimizzazione (18-36 mesi)}
\begin{itemize}
\item Completamento migrazione
\item Implementazione edge computing per POS
\item ML per ottimizzazione risorse
\item ROI atteso: 340\% (cumulativo)
\end{itemize}

\section{Conclusioni}

L'evoluzione infrastrutturale rappresenta la componente più pesante (32\%) del framework GIST, riflettendo il suo ruolo fondamentale nell'abilitare sicurezza e conformità. La validazione mediante Digital Twin dimostra che architetture moderne, quando implementate seguendo le strategie appropriate, possono simultaneamente migliorare performance e ridurre costi.

L'integrazione di questa dimensione architetturale con le componenti di sicurezza (ASSA-GDO) e conformità (MIN) nel framework GIST crea sinergie che amplificano i benefici individuali del 52\%, come verrà dimostrato nel capitolo conclusivo.