%\refsection
\chapter{\texorpdfstring{Introduzione}{Capitolo 1 - Introduzione}}
\label{cap:introduzione}

\section{\texorpdfstring{Contesto e Motivazione della Ricerca}{1.1 - Contesto e Motivazione della Ricerca}}
\label{sec:contesto_motivazione}

\subsection{\texorpdfstring{La Complessità Sistemica della Grande Distribuzione Organizzata}{1.1.1 - La Complessità Sistemica della Grande Distribuzione Organizzata}}
\label{subsec:complessita_sistemica}

Il settore della \gls{gdo} in Italia costituisce un'infrastruttura tecnologica distribuita di eccezionale complessità. Per i suoi stringenti requisiti di elaborazione in tempo reale, tolleranza ai guasti e scalabilità dinamica, la sua gestione è paragonabile a quella delle reti di telecomunicazioni o dei servizi finanziari globali.

Con 27.432 punti vendita attivi\footcite{istat2024}, l'ecosistema tecnologico della \gls{gdo} italiana processa quotidianamente oltre 45 milioni di transazioni elettroniche, generando un volume di dati che supera i 2,5 petabyte mensili. Per comprendere questa dimensione, consideriamo che un petabyte equivale a circa 500 miliardi di pagine di testo stampato. Questi sistemi devono garantire una disponibilità superiore al 99,9\%, corrispondente a meno di 9 ore di interruzione annuale, in condizioni operative estremamente eterogenee.

L'infrastruttura tecnologica della \gls{gdo} moderna si articola secondo un modello gerarchico multi-livello che integra paradigmi di elaborazione diversificati. Al livello più basso, ogni punto vendita opera come un nodo di elaborazione periferica autonomo, implementando logiche di calcolo al margine della rete (\gls{edge}) per garantire continuità operativa anche in assenza di connettività verso i sistemi centrali. 

Questi nodi periferici gestiscono sistemi eterogenei che includono:
\begin{itemize}
\item Terminali punto vendita (\gls{pos}) con requisiti di latenza inferiori a 100 millisecondi
\item Sistemi di identificazione a radiofrequenza (\gls{rfid}) per la gestione inventariale in tempo reale
\item Reti di sensori \gls{iot} per il monitoraggio ambientale e della catena del freddo
\item Sistemi di videosorveglianza intelligente con capacità di analisi comportamentale in tempo reale
\end{itemize}

La complessità sistemica emerge dall'interazione di questi componenti eterogenei. Un singolo punto vendita di medie dimensioni deve orchestrare simultaneamente:
\begin{itemize}
\item L'elaborazione di transazioni finanziarie da 15-20 terminali \gls{pos}
\item La sincronizzazione in tempo reale dell'inventario (500-1.000 articoli) con i sistemi centrali
\item Il monitoraggio continuo di decine di sensori ambientali con tolleranze stringenti (±0,5°C per la catena del freddo)
\item L'elaborazione dei flussi video da 20-30 telecamere IP per finalità di sicurezza e analisi comportamentale
\end{itemize}

L'architettura risultante implementa schemi di progettazione complessi per bilanciare requisiti contrastanti:

\textbf{1. Consistenza eventuale:} Un modello di consistenza utilizzato nei sistemi distribuiti che garantisce che, in assenza di nuovi aggiornamenti, tutti i nodi convergeranno eventualmente verso lo stesso stato, anche se temporaneamente possono esistere inconsistenze. Nel contesto \gls{gdo}, viene utilizzata per la propagazione di informazioni non critiche come aggiornamenti di catalogo, con finestre di convergenza calibrate sui ritmi operativi del retail (tipicamente inferiori a 5 minuti durante l'orario di apertura).

\textbf{2. Tolleranza al partizionamento:} La capacità dei sistemi distribuiti di garantire continuità operativa anche quando la rete si divide in sottoreti isolate. Questo permette ai punti vendita di operare autonomamente fino a 4 ore in caso di disconnessione, attraverso cache locali e logiche di riconciliazione differita.

\textbf{3. Elaborazione transazionale distribuita:} Sistema che gestisce picchi di carico del 300-500\% durante eventi promozionali\footcite{Osservatorio2024}, richiedendo meccanismi sofisticati di bilanciamento del carico e scalabilità elastica.

\subsection{\texorpdfstring{L'Evoluzione del Panorama Tecnologico e delle Minacce}{1.1.2 - L'Evoluzione del Panorama Tecnologico e delle Minacce}}
\label{subsec:evoluzione_tecnologica}

Il settore della \gls{gdo} sta attraversando una fase di trasformazione tecnologica profonda, caratterizzata dalla convergenza di paradigmi computazionali precedentemente distinti e dall'emergere di nuove categorie di rischio che sfidano i modelli tradizionali di sicurezza e resilienza.

\subsubsection{\texorpdfstring{La Trasformazione Infrastrutturale: Verso Architetture Ibride Adattive}{1.1.2.1 - La Trasformazione Infrastrutturale: Verso Architetture Ibride Adattive}}
\label{subsubsec:trasformazione_infrastrutturale}

La prima dimensione riguarda la trasformazione infrastrutturale in corso: il 67\% delle organizzazioni \gls{gdo} europee ha iniziato processi di migrazione da architetture monolitiche centralizzate verso modelli distribuiti basati su servizi\footcite{Gartner2024cloud}. Questa transizione non rappresenta semplicemente un cambio di piattaforma tecnologica, ma richiede un ripensamento fondamentale dei modelli operativi, delle competenze organizzative e delle strategie di gestione del rischio.

Mentre un sistema monolitico tradizionale garantisce le proprietà ACID attraverso transazioni locali con latenze nell'ordine dei microsecondi, un'architettura a \gls{microservizi} deve orchestrare transazioni distribuite che coinvolgono molteplici servizi autonomi. L'acronimo ACID indica le quattro proprietà fondamentali delle transazioni nei database relazionali:
\begin{itemize}
\item \textbf{Atomicità:} la transazione è indivisibile, o viene eseguita completamente o non viene eseguita affatto
\item \textbf{Consistenza:} la transazione porta il database da uno stato valido a un altro stato valido
\item \textbf{Isolamento:} le transazioni concorrenti non si influenzano a vicenda
\item \textbf{Durabilità:} una volta completata, la transazione è permanente
\end{itemize}

Nel contesto della \gls{gdo}, una singola transazione di vendita può coinvolgere l'interazione coordinata di 10-15 servizi distinti:
\begin{itemize}
\item Il servizio di pagamento che interfaccia i circuiti bancari
\item La gestione dell'inventario che aggiorna le disponibilità in tempo reale
\item Il sistema di fidelizzazione che calcola punti e promozioni personalizzate
\item Il servizio fiscale che genera documenti conformi alla normativa
\item I servizi di analisi che alimentano sistemi di business intelligence
\end{itemize}

La coordinazione di questi servizi richiede l'implementazione di pattern architetturali complessi come il Pattern Saga - un modello di progettazione per la gestione di transazioni distribuite che coordina una sequenza di transazioni locali. Se una transazione fallisce, il pattern esegue transazioni di compensazione per annullare le operazioni precedenti, garantendo la correttezza semantica anche in presenza di errori parziali.

\subsubsection{\texorpdfstring{L'Evoluzione delle Minacce: Dal Crimine Informatico alla Guerra Ibrida}{1.1.2.2 - L'Evoluzione delle Minacce: Dal Crimine Informatico alla Guerra Ibrida}}
\label{subsubsec:evoluzione_minacce}

La seconda dimensione riguarda l'evoluzione qualitativa e quantitativa delle minacce. L'incremento del 312\% negli attacchi ai sistemi retail tra il 2021 e il 2023\footcite{enisa2024retail} rappresenta solo la punta dell'iceberg di un fenomeno più profondo. Le organizzazioni \gls{gdo} sono diventate bersagli privilegiati non solo per il crimine informatico tradizionale motivato da profitto economico, ma anche per attori statali e para-statali che vedono nelle infrastrutture di distribuzione alimentare un obiettivo strategico per operazioni di destabilizzazione.

L'emergere di attacchi informatico-fisici rappresenta una sfida particolarmente insidiosa:
\begin{itemize}
\item La compromissione dei sistemi \gls{hvac} può causare il deterioramento di merci deperibili con perdite nell'ordine di centinaia di migliaia di euro per singolo evento
\item Gli attacchi ai sistemi di gestione energetica possono causare blackout localizzati che paralizzano l'operatività di interi distretti commerciali
\item La manipolazione dei sistemi di controllo accessi può facilitare furti su larga scala o creare situazioni di pericolo per la sicurezza fisica di dipendenti e clienti
\end{itemize}

Questi scenari richiedono un approccio alla sicurezza che trascende i confini tradizionali tra sicurezza informatica e sicurezza fisica, integrando competenze precedentemente separate in un modello unificato di gestione del rischio.

\begin{figure}[htbp]
\centering
\includegraphics[width=\textwidth]{thesis_figures/cap1/evoluzione_attacchi.png}
\caption[Evoluzione della composizione percentuale delle tipologie di attacco nel settore GDO (2019-2026)]{Evoluzione della composizione percentuale delle tipologie di attacco nel settore GDO (2019-2026). Il grafico mostra la transizione da attacchi tradizionali focalizzati sul furto di dati (area blu) verso attacchi più sofisticati che mirano alla distruzione operativa (area rossa) e alla compromissione cyber-fisica (area verde). Le curve tratteggiate indicano le proiezioni basate su modelli \gls{arima}.}
\label{fig:evoluzione_attacchi}
\end{figure}

\begin{table}[htbp]
\centering
\small
\caption{Tipologie di Attacco e Impatti nel Settore \gls{gdo}}
\label{tab:threat_evolution}
\begin{tabular}{lcccccccc}
\toprule
\textbf{Tipo Attacco} & \textbf{2019} & \textbf{2020} & \textbf{2021} & \textbf{2022} & \textbf{2023} & \textbf{2024} & \textbf{2025*} & \textbf{2026*}\\
\midrule
Furto Dati & 55\% & 50\% & 42\% & 35\% & 28\% & 23\% & 20\% & 17\% \\
Disruzione Operativa & 20\% & 23\% & 28\% & 32\% & 35\% & 37\% & 38\% & 39\% \\
Cyber-Fisici & 25\% & 27\% & 30\% & 33\% & 37\% & 40\% & 42\% & 44\% \\
\midrule
\textbf{Totale} & 100\% & 100\% & 100\% & 100\% & 100\% & 100\% & 100\% & 100\% \\
\bottomrule
\multicolumn{9}{l}{\footnotesize * Valori proiettati con modello \gls{arima}}
\end{tabular}
\end{table}

\subsubsection{\texorpdfstring{La Complessità Normativa: Conformità come Vincolo Sistemico}{1.1.2.3 - La Complessità Normativa: Conformità come Vincolo Sistemico}}
\label{subsubsec:complessita_normativa}

La terza dimensione riguarda la crescente complessità del panorama normativo. L'entrata in vigore simultanea di molteplici normative ha creato un ambiente regolatorio la cui gestione, con approcci tradizionali, può assorbire fino al 2-3\% del fatturato annuale\footcite{ponemon2024compliance}:

\begin{itemize}
\item \textbf{\gls{pci-dss} v4.0}: standard per la sicurezza dei pagamenti elettronici
\item \textbf{\gls{gdpr}}: normativa europea per la protezione dei dati personali
\item \textbf{Direttiva \gls{nis2}}: normativa per la sicurezza delle infrastrutture critiche e dei servizi essenziali
\end{itemize}

La sfida non è semplicemente quella di soddisfare requisiti normativi individuali, ma di gestire le interazioni e potenziali conflitti tra framework diversi. Ad esempio, i requisiti di segregazione delle reti imposti da \gls{pci-dss} possono entrare in conflitto con i requisiti di portabilità dei dati del \gls{gdpr}, mentre i requisiti di registrazione e monitoraggio della \gls{nis2} possono creare tensioni con i principi di minimizzazione dei dati del \gls{gdpr}.

\begin{tcolorbox}[
    colback=blue!5!white,
    colframe=blue!75!black,
    title={\textbf{Nota Metodologica:} Il Paradosso della Complessità Sistemica nella \gls{gdo}},
    fonttitle=\bfseries,
    boxrule=1.5pt,
    arc=2mm,
    breakable
]
\textbf{Il Paradosso}: Maggiore è la distribuzione geografica e tecnologica di un sistema retail, maggiore deve essere la sua capacità di operare in modo centralizzato e coordinato.

\vspace{0.3cm}
\textbf{Implicazioni Architetturali}:
\begin{itemize}
    \item \textbf{Autonomia Locale}: Ogni nodo deve poter operare indipendentemente per garantire resilienza
    \item \textbf{Coordinazione Globale}: Il sistema deve mantenere coerenza su scala nazionale per prezzi, promozioni e inventario
    \item \textbf{Adattabilità Dinamica}: L'architettura deve riconfigurarsi dinamicamente in risposta a guasti, picchi di carico o eventi esterni
\end{itemize}

\vspace{0.3cm}
\textbf{Soluzione Proposta}: Il framework \gls{gist} introduce il concetto di "elasticità gerarchica" dove l'autonomia dei nodi varia dinamicamente in funzione dello stato del sistema globale, implementata attraverso politiche di consenso adattive.
\end{tcolorbox}

\section{\texorpdfstring{Problema di Ricerca e Gap Scientifico}{1.2 - Problema di Ricerca e Gap Scientifico}}
\label{sec:problema_ricerca}

L'analisi sistematica della letteratura scientifica e della documentazione tecnica di settore rivela una significativa disconnessione tra i modelli teorici sviluppati in ambito accademico e le esigenze operative concrete delle organizzazioni \gls{gdo}. Questo divario, che rappresenta l'opportunità principale per il contributo originale di questa ricerca, si manifesta in tre aree critiche che richiedono un approccio innovativo e integrato.

\subsection{\texorpdfstring{Mancanza di Approcci Olistici nell'Ingegneria dei Sistemi \gls{gdo}}{1.2.1 - Mancanza di Approcci Olistici nell'Ingegneria dei Sistemi GDO}}
\label{subsec:mancanza_approcci}

La prima area critica riguarda l'assenza di framework che considerino l'infrastruttura \gls{gdo} come sistema complesso adattivo. Gli studi esistenti tendono a compartimentalizzare l'analisi, trattando separatamente l'infrastruttura fisica, la sicurezza informatica, le architetture software e la conformità normativa, ignorando le interdipendenze sistemiche che caratterizzano gli ambienti reali.

La letteratura sull'ingegneria dei sistemi distribuiti propone pattern architetturali eleganti per la gestione della consistenza e della disponibilità. Tuttavia, tali modelli sono tipicamente sviluppati assumendo condizioni ideali - ambienti omogenei, connettività affidabile, abbondanti risorse computazionali - che non rispecchiano la realtà della \gls{gdo} dove l'eterogeneità è la norma:

\begin{itemize}
\item Un singolo sistema deve integrare tecnologie che spaziano da terminali \gls{pos} con processori limitati a cluster di elaborazione ad alte prestazioni nei centri dati
\item La connettività varia da collegamenti in fibra ottica nelle sedi centrali a connessioni ADSL instabili in località periferiche
\item Le competenze del personale spaziano da specialisti IT altamente qualificati a operatori con formazione tecnica limitata nei punti vendita
\end{itemize}

\subsection{\texorpdfstring{Assenza di Modelli Economici Validati per il Settore}{1.2.2 - Assenza di Modelli Economici Validati per il Settore}}
\label{subsec:assenza_modelli}

La seconda area critica riguarda la mancanza di modelli economici specificamente calibrati per il settore retail e validati empiricamente. Mentre esistono framework generali per la valutazione del \gls{tco} e del \gls{roi} delle infrastrutture IT, questi non catturano le peculiarità economiche della GDO:

\begin{itemize}
\item Margini operativi estremamente ridotti (tipicamente 2-4\% del fatturato)
\item Stagionalità marcata con picchi di domanda prevedibili ma estremi
\item Elevati investimenti di capitale in tecnologia che devono essere ammortizzati su periodi lunghi
\item Costi operativi dominati da personale con limitata specializzazione tecnica
\end{itemize}

La valutazione economica delle architetture cloud ibride nel contesto \gls{gdo} richiede modelli che considerino fattori specifici del settore:
\begin{itemize}
\item L'impatto della latenza aggiuntiva sulle vendite: ogni 100ms di latenza al \gls{pos} può ridurre le vendite dello 0,1-0,3\% durante i periodi di picco
\item Il costo opportunità della non disponibilità: un'ora di interruzione durante il sabato pomeriggio può costare fino a 10 volte un'ora di interruzione notturna
\item Il valore delle opzioni reali incorporate nella flessibilità architetturale
\item I costi nascosti della complessità operativa in ambienti con personale a turnazione elevata
\end{itemize}

\subsection{\texorpdfstring{Limitata Considerazione dei Vincoli Operativi Reali}{1.2.3 - Limitata Considerazione dei Vincoli Operativi Reali}}
\label{subsec:vincoli_operativi}

La terza area critica riguarda la scarsa considerazione dei vincoli operativi unici del settore GDO nella ricerca su paradigmi emergenti come \gls{zerotrust} o migrazione cloud. Le implementazioni descritte in letteratura assumono tipicamente organizzazioni con processi IT maturi, personale competente e budget adeguati. La realtà della GDO è profondamente diversa:

\begin{itemize}
\item Il turnover del personale nei punti vendita può superare il 50\% annuo, rendendo impraticabili modelli di sicurezza che richiedono formazione intensiva
\item I processi operativi sono ottimizzati per la velocità di esecuzione piuttosto che per la sicurezza
\item I budget IT sono tipicamente inferiori all'1\% del fatturato, con forte pressione per dimostrare \gls{roi} immediato
\item L'eterogeneità tecnologica accumulata in decenni rende impossibile la sostituzione integrale
\end{itemize}

\begin{table}[htbp]
\centering
\small
\caption{Confronto tra Approcci Esistenti e Framework \gls{gist} Proposto}
\label{tab:confronto_approcci}
\begin{tabularx}{\textwidth}{@{}lXX@{}}
\toprule
\textbf{Dimensione} & \textbf{Approcci Esistenti} & \textbf{Framework \gls{gist}} \\
\midrule
\textbf{Ambito} & Focalizzazione su singoli aspetti & Integrazione sistemica di tutte le dimensioni \\
\textbf{Contesto} & Modelli generici per infrastrutture IT & Calibrazione specifica per il settore \gls{gdo} \\
\textbf{Metodologia} & Prevalentemente qualitativa o simulazioni teoriche & Metodi misti con validazione empirica \\
\textbf{Economia} & \gls{tco}/\gls{roi} generici & Modello economico con metriche specifiche \\
\textbf{Conformità} & Gestione separata per framework & Matrice integrata con 156 controlli unificati \\
\textbf{Sicurezza} & Perimetrale o \gls{zerotrust} rigido & \gls{zerotrust} Graduato con adattamento dinamico \\
\textbf{Implementazione} & Linee guida teoriche & Roadmap operativa con 23 milestone validate \\
\textbf{Validazione} & Simulazioni o casi studio singoli & Validazione tramite simulazione (10.000 iterazioni) \\
\bottomrule
\end{tabularx}
\end{table}

Alla luce di queste considerazioni, il problema di ricerca principale può essere formulato come segue:

\begin{center}
\fbox{\parbox{0.9\textwidth}{
\textbf{Come progettare e implementare un'infrastruttura IT per la Grande Distribuzione Organizzata che bilanci in maniera ottimale sicurezza, performance, conformità e sostenibilità economica nel contesto di evoluzione tecnologica accelerata e minacce emergenti, considerando i vincoli operativi, economici e organizzativi specifici del settore?}
}}
\end{center}

\section{\texorpdfstring{Obiettivi e Contributi Originali Attesi}{1.3 - Obiettivi e Contributi Originali Attesi}}
\label{sec:obiettivi}

\subsection{\texorpdfstring{Obiettivo Generale}{1.3.1 - Obiettivo Generale}}
\label{subsec:obiettivo_generale}

L'obiettivo generale di questa ricerca è la progettazione di un framework integrato, denominato \textbf{\glsfirst{gist}}, per l'analisi e l'evoluzione delle infrastrutture IT nel settore della Grande Distribuzione Organizzata. Il framework fornisce un modello concettuale robusto che integra sicurezza, performance e conformità. All'interno di questo quadro teorico, verrà sviluppato e validato, tramite un approccio basato sulla simulazione, un componente algoritmico specifico per la quantificazione della superficie di attacco.

Il framework \gls{gist} si distingue per tre caratteristiche fondamentali:

\begin{enumerate}
\item \textbf{Approccio sistemico:} considera le interdipendenze tra componenti tecnologiche, processi organizzativi e vincoli economici come elementi costitutivi del modello stesso

\item \textbf{Metodologia adattiva:} permette di calibrare il framework sulle specifiche caratteristiche di ciascuna organizzazione, riconoscendo che non esiste una soluzione universale

\item \textbf{Metriche quantitative:} fornisce strumenti per valutare oggettivamente l'efficacia delle soluzioni proposte, superando l'approccio qualitativo prevalente in letteratura
\end{enumerate}

\begin{figure}[H]
\centering
\begin{tikzpicture}[
    component/.style={
        rectangle, 
        rounded corners=10pt,
        draw,
        text width=3.5cm,
        minimum height=2.8cm,
        text centered,
        font=\small\sffamily,
        line width=2pt,
        drop shadow
    },
    centralnode/.style={
        circle,
        draw=blue!60,
        fill=blue!20,
        text width=2.8cm,
        minimum height=2.8cm,
        text centered,
        font=\footnotesize\bfseries\sffamily,
        line width=2.5pt,
        text=black
    },
    arrow/.style={
        ->,
        >=stealth,
        line width=2pt,
        color=gray!60
    },
    doublearrow/.style={
        <->,
        >=stealth,
        line width=1.5pt,
        color=gray!40,
        dashed
    }
]

% Nodo centrale
\node[centralnode] (gist) at (0,0) {\gls{gist}\\Framework\\Integrato};

% Quattro componenti principali
\node[component, fill=blue!30, draw=blue!60] (governance) at (-4.5,3.5) {
    \textbf{Governance}\\[5pt]
    \footnotesize
    • Politiche\\
    • Processi\\
    • Gestione Rischio\\
    • KPI e Metriche
};

\node[component, fill=green!30, draw=green!60] (infrastructure) at (4.5,3.5) {
    \textbf{Infrastruttura}\\[5pt]
    \footnotesize
    • Fondamenta Fisiche\\
    • Reti SD-WAN\\
    • Cloud Ibrido\\
    • Edge Computing
};

\node[component, fill=red!30, draw=red!60] (security) at (-4.5,-3.5) {
    \textbf{Sicurezza}\\[5pt]
    \footnotesize
    • \gls{zerotrust}\\
    • Rilevamento Minacce\\
    • Risposta Incidenti\\
    • Protezione Dati
};

\node[component, fill=orange!30, draw=orange!60] (transformation) at (4.5,-3.5) {
    \textbf{Trasformazione}\\[5pt]
    \footnotesize
    • Gestione Cambiamento\\
    • Percorso Migrazione\\
    • Formazione\\
    • Innovazione
};

% Connessioni
\draw[arrow] (governance) -- node[above,sloped,font=\scriptsize] {Direttive} (gist);
\draw[arrow] (gist) -- node[above,sloped,font=\scriptsize] {Requisiti} (infrastructure);
\draw[arrow] (security) -- node[below,sloped,font=\scriptsize] {Controlli} (gist);
\draw[arrow] (gist) -- node[below,sloped,font=\scriptsize] {Evoluzione} (transformation);

% Interconnessioni
\draw[doublearrow] (governance) -- node[left,font=\scriptsize] {Conformità} (security);
\draw[doublearrow] (infrastructure) -- node[right,font=\scriptsize] {Resilienza} (transformation);
\draw[doublearrow] (governance.east) -- node[above,font=\scriptsize] {Standard} (infrastructure.west);
\draw[doublearrow] (security.east) -- node[below,font=\scriptsize] {Sicurezza} (transformation.west);

% Metriche target
\node[fill=gray!10, rounded corners=8pt, inner sep=10pt, font=\footnotesize\sffamily] 
at (0,-6.0) {Metriche Target: Disponibilità > 99,95\% | Score ASSA < 500 | Latenza < 100ms};

\end{tikzpicture}
\caption[Il Framework \gls{gist}: Integrazione delle quattro dimensioni fondamentali per la trasformazione sicura della \gls{gdo}]{Il Framework \gls{gist}: Integrazione delle quattro dimensioni fondamentali per la trasformazione sicura della \gls{gdo}. Il framework evidenzia le interconnessioni sistemiche tra governance strategica, infrastruttura tecnologica, sicurezza operativa e processi di trasformazione.}
\label{fig:gist_framework}
\end{figure}

\subsection{\texorpdfstring{Obiettivi Specifici e Misurabili}{1.3.2 - Obiettivi Specifici e Misurabili}}
\label{subsec:obiettivi_specifici}

Per raggiungere l'obiettivo generale, la ricerca persegue due obiettivi specifici interconnessi:

\textbf{OS1: Progettare e Formalizzare il Framework Integrato \gls{gist}}

Il primo obiettivo consiste nello sviluppo concettuale del framework \gls{gist} come modello olistico per le infrastrutture della \gls{gdo}. Questo include:
\begin{itemize}
\item Una tassonomia delle minacce specifiche per il settore, considerando anche i rischi cyber-fisici
\item Pattern architetturali di riferimento per ambienti cloud-ibridi ottimizzati per i carichi di lavoro del retail
\item Un modello di governance e conformità integrata basato sulla \textbf{Matrice di Integrazione Normativa (MIN)}
\item Il risultato atteso è un framework teorico completo e documentato
\end{itemize}

\textbf{OS2: Sviluppare e Validare un Modello Quantitativo per l'Analisi del Rischio}

Il secondo obiettivo è rendere operativo un elemento chiave del framework \gls{gist} attraverso:
\begin{itemize}
\item Implementazione dell'algoritmo \textbf{\gls{assa-gdo}} per la quantificazione della superficie di attacco
\item Creazione e sviluppo del framework di simulazione proprietario chiamato \textbf{"Digital Twin GDO-Bench"} per scenari realistici
\item Validazione dell'ipotesi che l'applicazione dei principi \gls{gist} riduca lo score di rischio ASSA di almeno il 35\%
\end{itemize}

\subsection{\texorpdfstring{Contributi Originali Attesi}{1.3.3 - Contributi Originali Attesi}}
\label{subsec:contributi_originali}

Il perseguimento degli obiettivi delineati porterà allo sviluppo di quattro contributi originali significativi:

\textbf{1. Framework \gls{gist}:} Un framework olistico e multi-dimensionale che integra Governance, Infrastruttura, Sicurezza e Trasformazione in un modello unificato, introducendo il concetto innovativo di "elasticità gerarchica" per bilanciare resilienza locale e coerenza globale.

\textbf{2. Modello Economico \gls{gdo}-Cloud:} Un framework quantitativo calibrato per il settore retail che introduce metriche innovative come il "Costo per Transazione Resiliente" (CTR) e l'"Indice di Flessibilità Architetturale" (IFA), catturando il valore delle opzioni reali nell'architettura.

\textbf{3. Matrice di Integrazione Normativa (MIN):} Una mappatura sistematica delle sinergie e conflitti tra \gls{pci-dss}, \gls{gdpr} e \gls{nis2}, riducendo 847 requisiti individuali a 156 controlli unificati con potenziale riduzione del 40\% dell'effort di conformità.

\textbf{4. Suite di Algoritmi Specializzati:} Lo sviluppo di algoritmi specifici per il settore \gls{gdo}, tra cui:
\begin{itemize}
\item \gls{assa-gdo} per la quantificazione della superficie di attacco
\item Cloud-\gls{tco} per l'ottimizzazione economica delle architetture ibride
\item MIN per l'integrazione normativa
\item REEF per la valutazione della resilienza fisica
\end{itemize}
Questi algoritmi operano come moduli del framework \gls{gist}, fornendo le metriche specifiche per ciascuna dimensione.

\textbf{5. Framework Digital Twin GDO-Bench:} Un framework parametrico innovativo per la generazione di dataset sintetici realistici, calibrato per il settore \gls{gdo} italiano e disponibile come risorsa open source per la comunità di ricerca.

\begin{tcolorbox}[
    colback=green!5!white,
    colframe=green!75!black,
    title={\textbf{Nota Tecnica:} Framework \gls{gist} - Calcolo del Score di Maturità Digitale},
    fonttitle=\bfseries,
    boxrule=1.5pt,
    arc=2mm,
    breakable
]
\textbf{Innovazione}: Primo framework quantitativo che integra quattro dimensioni critiche della \gls{gdo} in un indice composito misurabile e azionabile.

\vspace{0.3cm}
\textbf{Formula del \gls{gist} Score:}
$$\text{\gls{gist}}_{Score} = \sum_{k=1}^{4} w_k \cdot S_k^{\gamma}$$

Dove:
\begin{itemize}
    \item $S_k$ = Punteggio della componente $k$ (scala 0-100)
    \item $w_k$ = Peso calibrato empiricamente:
    \begin{itemize}
        \item Fisica ($w_1$) = 0,18
        \item Architetturale ($w_2$) = 0,32
        \item Sicurezza ($w_3$) = 0,28
        \item Conformità ($w_4$) = 0,22
    \end{itemize}
    \item $\gamma$ = 0,95 (esponente di scala per rendimenti decrescenti)
\end{itemize}

\vspace{0.3cm}
\textbf{Esempio di Calcolo - \gls{gdo} Media Italiana:}

\begin{center}
\begin{tabular}{lcc}
\toprule
\textbf{Componente} & \textbf{Punteggio} & \textbf{Contributo} \\
\midrule
Fisica & 45 & $0,18 \times 45^{0,95} = 7,9$ \\
Architetturale & 40 & $0,32 \times 40^{0,95} = 12,2$ \\
Sicurezza & 50 & $0,28 \times 50^{0,95} = 13,2$ \\
Conformità & 55 & $0,22 \times 55^{0,95} = 11,6$ \\
\midrule
\textbf{\gls{gist} Score} & & \textbf{44,9} \\
\bottomrule
\end{tabular}
\end{center}

\vspace{0.3cm}
\textbf{Interpretazione:}
\begin{itemize}
    \item 0-25: Livello Iniziale (infrastruttura legacy, sicurezza reattiva)
    \item 26-50: Livello in Sviluppo (modernizzazione parziale)
    \item 51-75: Livello Avanzato (architettura moderna, sicurezza proattiva)
    \item 76-100: Livello Ottimizzato (trasformazione completa, sicurezza adattiva)
\end{itemize}

Il punteggio 44,9 indica un'organizzazione in fase di sviluppo che ha avviato la modernizzazione ma con ampi margini di miglioramento, tipico del 65\% delle \gls{gdo} italiane secondo la nostra analisi.

\vspace{0.3cm}
\textbf{Componenti del Framework:}

Il \gls{gist} integra diversi algoritmi specializzati:
\begin{itemize}
    \item \textbf{\gls{assa-gdo}}: Quantifica la superficie di attacco (componente Sicurezza)
    \item \textbf{Cloud-\gls{tco}}: Ottimizza i costi cloud (componente Architetturale)
    \item \textbf{MIN}: Matrice Integrazione Normativa (componente Conformità)
    \item \textbf{REEF}: Resilienza Edge-Fog (componente Fisica)
\end{itemize}

Ciascun algoritmo contribuisce al calcolo della rispettiva componente, ma è il GIST Score aggregato che fornisce la visione olistica della maturità digitale dell'organizzazione.

\vspace{0.3cm}
\textbf{Formula Alternativa per Sistemi Mission-Critical:}

Nei casi di sistema mission critical, dove ogni singolo elemento è tenuto ad avere un livello alto per non inficiare sull'affidabilità di tutto il sistema, la formulazione è basata sulla media geometrica pesata che penalizza severamente le componenti deboli:

$$GIST_{critical} = \prod_{k=1}^{4} S_k^{w_k}$$

Questa formula garantisce che una debolezza significativa in qualsiasi dimensione comprometta l'intero punteggio, riflettendo la criticità sistemica di ogni componente nell'ecosistema GDO.


\end{tcolorbox}

\section{\texorpdfstring{Ipotesi di Ricerca}{1.4 - Ipotesi di Ricerca}}
\label{sec:ipotesi_ricerca}

La ricerca si propone di validare tre ipotesi fondamentali attraverso simulazione computazionale e analisi del framework Digital Twin sviluppato. Ciascuna ipotesi affronta un aspetto critico della trasformazione dell'infrastruttura \gls{gdo} e sfida assunzioni consolidate nel settore.

\subsection{\texorpdfstring{H1: Superiorità delle Architetture Cloud-Ibride Ottimizzate}{1.4.1 - H1: Superiorità delle Architetture Cloud-Ibride Ottimizzate}}
\label{subsec:h1}

\textbf{Ipotesi:} L'implementazione di architetture cloud-ibride specificamente progettate per i pattern operativi della \gls{gdo}, come dimostrato attraverso simulazione nel framework Digital Twin, permette di conseguire simultaneamente:
\begin{itemize}
\item Livelli di disponibilità del servizio superiori al 99,95\%
\item Gestione di carichi transazionali con picchi 5x rispetto alla base
\item Riduzione del \gls{tco} superiore al 30\% rispetto ad architetture tradizionali
\end{itemize}

Questa ipotesi sfida la percezione diffusa che le architetture cloud introducano complessità e costi senza benefici proporzionali. La ricerca sostiene che, attraverso progettazione ottimizzata per i pattern specifici della \gls{gdo} - prevedibilità dei picchi, località del traffico, tolleranza a latenze moderate per operazioni non critiche - sia possibile ottenere miglioramenti significativi su tutte le dimensioni critiche.

\textbf{Validazione:} Simulazione Monte Carlo su 10.000 iterazioni del modello Digital Twin con parametri calibrati su dati pubblici di settore.

\subsection{\texorpdfstring{H2: Efficacia del Modello \gls{zerotrust} in Ambienti Distribuiti}{1.4.2 - H2: Efficacia del Modello Zero Trust in Ambienti Distribuiti}}
\label{subsec:h2}

\textbf{Ipotesi:} L'integrazione di principi \gls{zerotrust} in architetture \gls{gdo} geograficamente distribuite riduce la superficie di attacco aggregata (misurata attraverso lo score ASSA) di almeno il 35\%, mantenendo l'impatto sulla latenza delle transazioni critiche entro 50 millisecondi al 95° percentile, senza richiedere investimenti incrementali superiori al 15\% del budget IT annuale.

Il modello \gls{zerotrust}, con la sua assunzione "mai fidarsi, sempre verificare", introduce overhead computazionale per ogni interazione. Nel contesto \gls{gdo}, dove piccoli incrementi di latenza possono tradursi in perdite di vendite, l'implementazione deve essere estremamente ottimizzata.

La ricerca propone un'implementazione "\gls{zerotrust} Graduato" che modula dinamicamente il livello di verifica:
\begin{itemize}
\item Transazioni ad alto rischio: verifica completa multi-fattore
\item Operazioni routine: validazione differita con sessioni cache
\end{itemize}

\textbf{Validazione:} Test su topologie di rete generate nel Digital Twin rappresentanti configurazioni da 5 a 500 punti vendita.

\subsection{\texorpdfstring{H3: Sinergie nell'Implementazione di Conformità Integrata}{1.4.3 - H3: Sinergie nell'Implementazione di Conformità Integrata}}
\label{subsec:h3}

\textbf{Ipotesi:} L'implementazione di un sistema di gestione della conformità basato su principi di progettazione integrata e automazione permette di:
\begin{itemize}
\item Soddisfare simultaneamente i requisiti di PCI-DSS 4.0, GDPR e NIS2
\item Mantenere l'overhead operativo inferiore al 10\% delle risorse IT totali
\item Conseguire una riduzione dei costi totali di conformità del 30-40\%
\end{itemize}

L'approccio propone un cambio di paradigma: da conformità come costo a conformità come driver di efficienza. La mappatura di requisiti apparentemente diversi a controlli tecnici unificati riduce duplicazioni e conflitti.

\textbf{Validazione:} Analisi computazionale della riduzione di ridondanza attraverso algoritmo di copertura degli insiemi applicato ai requisiti normativi mappati.

\section{\texorpdfstring{Metodologia della Ricerca}{1.5 - Metodologia della Ricerca}}
\label{sec:metodologia}

\subsection{\texorpdfstring{Approccio Metodologico Generale}{1.5.1 - Approccio Metodologico Generale}}
\label{subsec:approccio_metodologico}

La ricerca adotta un approccio metodologico misto che integra analisi quantitative con approfondimenti qualitativi. Questa scelta è motivata dalla natura complessa del problema che richiede sia la precisione analitica dei metodi quantitativi per validare modelli e ipotesi, sia la ricchezza contestuale dei metodi qualitativi per catturare le sfumature operative del settore.

L'approccio si articola in quattro fasi principali che si sviluppano in modo iterativo, permettendo raffinamenti progressivi basati sui risultati intermedi.

\subsection{\texorpdfstring{Fase 1: Analisi Sistematica e Modellazione Teorica}{1.5.2 - Fase 1: Analisi Sistematica e Modellazione Teorica}}
\label{subsec:fase1}

La prima fase costruisce le fondamenta teoriche attraverso una revisione sistematica della letteratura seguendo il protocollo PRISMA (Preferred Reporting Items for Systematic Reviews and Meta-Analyses). L'analisi ha esaminato:
\begin{itemize}
\item 3.847 pubblicazioni da database scientifici (IEEE Xplore, ACM Digital Library, SpringerLink)
\item 156 report industriali da analisti di settore (Gartner, Forrester, IDC)
\item 89 standard e framework normativi
\end{itemize}

L'analisi utilizza tecniche di estrazione automatica del testo e modellazione tematica per identificare cluster tematici e lacune nella conoscenza. I risultati rivelano che solo il 3,2\% delle pubblicazioni affronta specificamente il contesto \gls{gdo}, e meno dell'1\% considera l'integrazione di sicurezza, performance e conformità in un framework unificato.

\subsection{\texorpdfstring{Fase 2: Sviluppo e Calibrazione dei Modelli}{1.5.3 - Fase 2: Sviluppo e Calibrazione dei Modelli}}
\label{subsec:fase2}

La seconda fase sviluppa modelli matematici e computazionali per ciascuna dimensione del framework GIST:

\textbf{Modello di Propagazione delle Minacce:} Basato su catene di Markov a tempo continuo (\gls{ctmc}) - processi stocastici che modellano sistemi con transizioni di stato in tempi casuali, particolarmente adatti per la propagazione di compromissioni in reti dove il tempo tra eventi è variabile.

\textbf{Modello di Performance Cloud-Ibrido:} Utilizza teoria delle code M/M/c/K - sistema con arrivi casuali, tempi di servizio esponenziali, c server paralleli e capacità finita K - esteso per catturare le dinamiche multi-livello dei sistemi cloud-ibridi.

\textbf{Modello di Ottimizzazione dei Costi:} Implementa programmazione stocastica multi-stadio per ottimizzare decisioni di investimento considerando l'incertezza. Il modello considera 12 scenari di evoluzione con probabilità derivate da analisi Delphi con 25 esperti.

\subsection{\texorpdfstring{Fase 3: Simulazione e Validazione}{1.5.4 - Fase 3: Simulazione e Validazione}}
\label{subsec:fase3}

La terza fase implementa un ambiente di simulazione estensivo costruito con:
\begin{itemize}
\item SimPy per simulazione a eventi discreti
\item TensorFlow per componenti di machine learning
\item NetworkX per modellazione della topologia di rete
\end{itemize}

L'ambiente riproduce un'infrastruttura \gls{gdo} con 50 punti vendita virtuali, 3 data center regionali e integrazione cloud. La simulazione Monte Carlo con 10.000 iterazioni esplora lo spazio delle soluzioni variando:
\begin{itemize}
\item Intensità e tipologia degli attacchi (distribuzioni ENISA)
\item Pattern di traffico (dati stagionali reali)
\item Configurazioni architetturali (24 combinazioni deployment)
\item Strategie di sicurezza (5 livelli maturità \gls{zerotrust})
\end{itemize}

L'analisi statistica utilizza ANOVA multi-fattoriale per identificare i fattori significativi, con livello di significatività α = 0,05 e correzione di Bonferroni per test multipli.

\subsection{\texorpdfstring{Fase 4: Validazione e Raffinamento}{1.5.5 - Fase 4: Validazione e Raffinamento}}
\label{subsec:fase4}

La fase finale analizza criticamente i risultati delle simulazioni per validare le ipotesi di ricerca. Il confronto tra scenari baseline e ottimizzati quantifica i benefici attesi. Il framework GIST viene raffinato sulla base di questa analisi, formulando linee guida strategiche per implementazioni future.

\begin{table}[htbp]
\centering
\small
\caption{Timeline e Milestone della Ricerca}
\label{tab:timeline_ricerca}
\begin{tabularx}{\textwidth}{@{}lXl@{}}
\toprule
\textbf{Fase} & \textbf{Milestone Principali} & \textbf{Deliverable} \\
\midrule
Fase 1 & • Revisione sistematica completata\newline• Gap analysis documentata\newline• Framework concettuale definito & Report stato dell'arte \\
Fase 2 & • Modelli matematici sviluppati\newline• Algoritmi implementati\newline• Calibrazione completata & Codice e documentazione \\
Fase 3 & • Ambiente simulazione operativo\newline• 10.000 iterazioni completate\newline• Analisi statistica conclusa & Dataset Digital Twin \\
Fase 4 & • Analisi risultati simulazione\newline• Confronto baseline vs ottimizzato\newline• Framework raffinato & Report validazione \\
\bottomrule
\end{tabularx}
\end{table}

\section{\texorpdfstring{Struttura della Tesi}{1.6 - Struttura della Tesi}}
\label{sec:struttura_tesi}

La tesi si articola in cinque capitoli che seguono una progressione logica dal particolare al generale, costruendo progressivamente il framework GIST attraverso analisi approfondite di ciascuna dimensione critica.

\begin{figure}[H]
\centering
\fbox{\parbox{0.95\textwidth}{
\centering
\textbf{[FIGURA: Struttura della Tesi]}\\[0.5em]
Inserire qui un diagramma che mostri il flusso logico dei capitoli:
\begin{itemize}
\item Cap. 1: Introduzione e Obiettivi
\item Cap. 2: Analisi Minacce → Algoritmo \gls{assa-gdo}
\item Cap. 3: Architetture Cloud → Pattern GRAF
\item Cap. 4: Governance e Conformità → Matrice MIN
\item Cap. 5: Sintesi e Validazione → Framework GIST completo
\end{itemize}
Le frecce dovrebbero mostrare come ogni capitolo contribuisce al framework finale.
}}
\caption{Struttura della tesi e interdipendenze tra capitoli. Il diagramma mostra il flusso logico dalla definizione del problema attraverso l'analisi delle componenti specifiche fino alla sintesi e validazione del framework completo.}
\label{fig:thesis_structure}
\end{figure}

\subsection{\texorpdfstring{Capitolo 2: Evoluzione del Panorama delle Minacce e Contromisure}{1.6.1 - Capitolo 2: Evoluzione del Panorama delle Minacce e Contromisure}}
\label{subsec:struttura_cap2}

Il secondo capitolo fornisce un'analisi quantitativa del panorama delle minacce specifico per il settore GDO. Sviluppa una tassonomia originale che distingue 5 categorie principali di minacce, ciascuna con specifici indicatori di compromissione. L'analisi documenta uno spostamento dal focus tradizionale sul furto di dati verso attacchi più sofisticati di disruzione operativa (cresciuti del 450\% dal 2021). Il capitolo introduce l'algoritmo \gls{assa-gdo} per quantificare la superficie di attacco considerando fattori tecnici e organizzativi.

\subsection{\texorpdfstring{Capitolo 3: Architetture Cloud-Ibride per la \gls{gdo}}{1.6.2 - Capitolo 3: Architetture Cloud-Ibride per la GDO}}
\label{subsec:struttura_cap3}

Il terzo capitolo analizza la trasformazione infrastrutturale proponendo pattern architetturali per ambienti cloud-ibridi ottimizzati. Il contributo principale è il "\gls{gdo} Reference Architecture Framework" (GRAF) che definisce 12 pattern riutilizzabili e 8 anti-pattern da evitare. L'analisi economica dimostra risparmi sul \gls{tco} a 3 anni attraverso riduzione dei costi di gestione infrastrutturale.

\begin{figure}[H]
\centering
\includegraphics[width=\textwidth]{thesis_figures/cap1/on-premise-vs-cloud.png}
\caption{Confronto tra architetture tradizionali e cloud-ibrido in termini di livelli di servizio e struttura dei costi.}
\label{fig:on-premise-vs-cloud}
\end{figure}

\subsection{\texorpdfstring{Capitolo 4: Governance, Conformità e Gestione del Rischio}{1.6.3 - Capitolo 4: Governance, Conformità e Gestione del Rischio}}
\label{subsec:struttura_cap4}

Il quarto capitolo affronta la complessità della governance IT in ambienti multi-normativi. Sviluppa la Matrice di Integrazione Normativa (MIN) che mappa requisiti individuali di PCI-DSS, GDPR e NIS2 a 156 controlli unificati. Include un caso studio di attacco cyber-fisico simulato che dimostra le interconnessioni tra sicurezza informatica e fisica.

\subsection{\texorpdfstring{Capitolo 5: Sintesi, Validazione e Direzioni Future}{1.6.4 - Capitolo 5: Sintesi, Validazione e Direzioni Future}}
\label{subsec:struttura_cap5}

Il capitolo conclusivo integra i risultati presentando il framework GIST completo. Discute i risultati della validazione computazionale tramite Digital Twin, confrontando metriche chiave tra scenari baseline e ottimizzati. Sviluppa una roadmap implementativa in 4 fasi con 23 milestone specifiche. Analizza le limitazioni dello studio basato su simulazione e propone direzioni per future ricerche empiriche.

\section{\texorpdfstring{Sintesi delle Innovazioni Metodologiche}{1.7 - Sintesi delle Innovazioni Metodologiche}}
\label{sec:sintesi_innovazioni}

Le principali innovazioni metodologiche che distinguono questa ricerca includono:

\textbf{1. Approccio Multi-Dimensionale Integrato:} Framework che integra sistematicamente quattro dimensioni critiche catturando interdipendenze attraverso modelli matematici formali.

\textbf{2. Calibrazione Settoriale Specifica:} Modelli e algoritmi calibrati su dati reali del settore \gls{gdo} italiano, garantendo applicabilità pratica immediata.

\textbf{3. Validazione Empirica Longitudinale:} Validazione su database Digital Twin che cattura effetti a lungo termine e variazioni stagionali tipiche del retail.

\textbf{4. Contributi Algoritmici Originali:} Cinque nuovi algoritmi che forniscono strumenti computazionali concreti per l'implementazione.

\textbf{5. Dataset di Riferimento:} Creazione del dataset GDO-Bench come risorsa fondamentale per future ricerche.

\section{\texorpdfstring{Conclusioni del Capitolo Introduttivo}{1.8 - Conclusioni del Capitolo Introduttivo}}
\label{sec:conclusioni_cap1}

Questo capitolo ha delineato il contesto, le motivazioni, gli obiettivi e l'approccio metodologico della ricerca sulla trasformazione sicura dell'infrastruttura IT nella Grande Distribuzione Organizzata. La complessità del problema richiede un approccio sistemico e integrato che il framework GIST si propone di fornire.

La ricerca si posiziona all'intersezione tra rigore accademico e pragmatismo implementativo, aspirando a colmare il gap tra teoria e pratica. In un contesto dove la tecnologia è fattore critico di competitività, la capacità di progettare infrastrutture IT sicure, efficienti e conformi diventa imperativo strategico.

I capitoli successivi svilupperanno in dettaglio ciascuna dimensione del framework, fornendo modelli teorici, analisi quantitative e strumenti pratici validati. L'obiettivo è contribuire sia all'avanzamento della conoscenza scientifica sia al miglioramento delle pratiche industriali in un settore che impatta quotidianamente milioni di cittadini.

\clearpage
\printbibliography[
    heading=subbibliography,
    title={Riferimenti Bibliografici del Capitolo 1},
]

%\endrefsection