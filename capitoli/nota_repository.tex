% Nota sul Repository GitHub - Da inserire dopo abstract o in prefazione

\section*{Materiale Supplementare e Implementazioni}

Il presente lavoro di tesi è corredato da un repository GitHub contenente le implementazioni complete degli algoritmi, strumenti operativi e dataset di validazione descritti nel documento.

\begin{center}
\fcolorbox{blue!50}{blue!10}{
\parbox{0.9\textwidth}{
\centering
\vspace{0.5em}
{\Large \textbf{GIST Framework - Repository Ufficiale}}\\[0.5em]
\url{https://github.com/gist-framework/gdo-security}\\[0.5em]
{\small \texttt{git clone https://github.com/gist-framework/gdo-security.git}}\\[0.5em]
\includegraphics[width=3cm]{figure/qr_code_repository.png}\\[0.3em]
{\footnotesize Scansiona il QR code per accedere al repository}
\vspace{0.5em}
}}
\end{center}

\subsection*{Contenuti del Repository}

Il repository include:

\begin{itemize}
    \item \textbf{Implementazioni Core}:
    \begin{itemize}
        \item GIST Calculator - Sistema completo di calcolo del GIST Score
        \item ASSA-GDO Algorithm - Quantificazione superficie di attacco
        \item Digital Twin Framework - Generazione dataset sintetici
    \end{itemize}

    \item \textbf{Strumenti Operativi}:
    \begin{itemize}
        \item Runbook automatizzati per incident response
        \item Checklist migrazione cloud in formato JSON
        \item Dashboard Grafana per monitoring real-time
    \end{itemize}

    \item \textbf{Dataset e Validazione}:
    \begin{itemize}
        \item Dataset sintetici calibrati su 234 organizzazioni
        \item Script di validazione statistica
        \item Notebook Jupyter con analisi complete
    \end{itemize}

    \item \textbf{Documentazione}:
    \begin{itemize}
        \item API Reference completa
        \item Esempi d'uso e tutorial
        \item Guide deployment per produzione
    \end{itemize}
\end{itemize}

\subsection*{Licenza e Utilizzo}

Il codice è rilasciato sotto licenza MIT, permettendo l'uso commerciale e accademico con attribuzione. Per citare il framework nelle pubblicazioni:

\begin{quote}
\small
\textit{GIST Framework (2025). Framework per la Valutazione della Maturità Digitale nel settore GDO. GitHub repository: \url{https://github.com/gist-framework/gdo-security}}
\end{quote}

\subsection*{Installazione Rapida}

\begin{lstlisting}[language=bash, basicstyle=\small\ttfamily]
# Installazione via pip
pip install gist-framework

# Oppure da sorgente
git clone https://github.com/gist-framework/gdo-security.git
cd gdo-security
pip install -r requirements.txt
python gist_calculator.py --demo
\end{lstlisting}