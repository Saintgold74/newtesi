\chapter{\texorpdfstring{Metodologia di Ricerca}{Appendice A - Metodologia di Ricerca}}
\label{app:metodologia}

\section{\texorpdfstring{Protocollo di Revisione Sistematica}{A.1 - Protocollo di Revisione Sistematica}}

La revisione sistematica della letteratura ha seguito il protocollo PRISMA (Preferred Reporting Items for Systematic Reviews and Meta-Analyses) per garantire rigorosità metodologica e riproducibilità dei risultati.

\subsection{\texorpdfstring{Strategia di Ricerca}{A.1.1 - Strategia di Ricerca}}

La ricerca bibliografica è stata condotta su sei database principali utilizzando la seguente stringa di ricerca:

\begin{verbatim}
("retail" OR "grande distribuzione" OR "GDO" OR "grocery")
AND
("cloud computing" OR "hybrid cloud" OR "infrastructure")
AND
("security" OR "zero trust" OR "compliance")
AND
("PCI-DSS" OR "GDPR" OR "NIS2" OR "framework")
\end{verbatim}

\textbf{Database consultati:}
\begin{itemize}
    \item IEEE Xplore: 1.247 risultati iniziali
    \item ACM Digital Library: 892 risultati
    \item SpringerLink: 734 risultati
    \item ScienceDirect: 567 risultati
    \item Web of Science: 298 risultati
    \item Scopus: 109 risultati
\end{itemize}

\textbf{Totale iniziale}: 3.847 pubblicazioni

\subsection{\texorpdfstring{Criteri di Inclusione ed Esclusione}{A.1.2 - Criteri di Inclusione ed Esclusione}}

\textbf{Criteri di inclusione:}
\begin{enumerate}
    \item Pubblicazioni peer-reviewed dal 2019 al 2025
    \item Studi empirici con dati quantitativi
    \item Focus su infrastrutture distribuite mission-critical
    \item Disponibilità del testo completo
    \item Lingua: inglese o italiano
\end{enumerate}

\textbf{Criteri di esclusione:}
\begin{enumerate}
    \item Abstract, poster o presentazioni senza paper completo
    \item Studi puramente teorici senza validazione
    \item Focus esclusivo su e-commerce B2C
    \item Duplicati o versioni preliminari di studi successivi
\end{enumerate}

\subsection{\texorpdfstring{Processo di Selezione}{A.1.3 - Processo di Selezione}}

Il processo di selezione si è articolato in quattro fasi seguendo il diagramma di flusso PRISMA:

\begin{table}[htbp]
\centering
\caption{Fasi del processo di selezione PRISMA}
\begin{tabular}{|l|c|c|c|}
\hline
\textbf{Fase} & \textbf{Articoli} & \textbf{Esclusi} & \textbf{Rimanenti} \\
\hline
Identificazione & 3.847 & - & 3.847 \\
Rimozione duplicati & 3.847 & 1.023 & 2.824 \\
Screening titolo/abstract & 2.824 & 2.156 & 668 \\
Valutazione testo completo & 668 & 432 & 236 \\
Inclusione finale & 236 & - & 236 \\
\hline
\end{tabular}
\end{table}

\section{\texorpdfstring{Metodologia Digital Twin}{A.2 - Metodologia Digital Twin}}

Per superare le limitazioni di accesso ai dati reali nel settore GDO, è stato sviluppato un framework Digital Twin calibrato su fonti pubbliche verificabili.

\subsection{\texorpdfstring{Archetipi Organizzativi}{A.2.1 - Archetipi Organizzativi}}

Il Digital Twin simula 5 archetipi organizzativi rappresentativi delle 234 configurazioni identificate nella ricerca empirica:

\begin{table}[h]
\centering
\caption{Archetipi organizzativi simulati}
\begin{tabular}{@{}lccc@{}}
\toprule
\textbf{Archetipo} & \textbf{Range PV} & \textbf{Organizzazioni} & \textbf{Trans/giorno} \\
\midrule
Micro & 1-10 & 87 & 450 \\
Piccola & 10-50 & 73 & 1.200 \\
Media & 50-150 & 42 & 2.800 \\
Grande & 150-500 & 25 & 5.500 \\
Enterprise & 500-2000 & 7 & 12.000 \\
\bottomrule
\end{tabular}
\end{table}

\subsection{\texorpdfstring{Parametri di Calibrazione}{A.2.2 - Parametri di Calibrazione}}

I parametri del modello sono calibrati esclusivamente su fonti pubbliche verificabili:

\begin{table}[h]
\centering
\caption{Fonti di calibrazione del Digital Twin}
\begin{tabular}{@{}lll@{}}
\toprule
\textbf{Categoria} & \textbf{Parametri} & \textbf{Fonte} \\
\midrule
Volumi transazionali & 450-12.000 trans/giorno & ISTAT 2023 \\
Valore medio scontrino & €18.50-42.10 & ISTAT 2023 \\
Distribuzione pagamenti & Cash 31\%, Card 59\% & Banca d'Italia 2023 \\
Threat landscape & FP rate 87\% & ENISA 2023 \\
Distribuzione minacce & Malware 28\%, Phishing 22\% & ENISA 2023 \\
\bottomrule
\end{tabular}
\end{table}

\section{\texorpdfstring{Validazione Statistica}{A.3 - Validazione Statistica}}

La validazione del framework comprende test statistici standardizzati per verificare il realismo dei dati generati:

\begin{table}[h]
\centering
\caption{Risultati validazione statistica}
\begin{tabular}{@{}lccc@{}}
\toprule
\textbf{Test Statistico} & \textbf{Statistica} & \textbf{p-value} & \textbf{Risultato} \\
\midrule
Benford's Law (importi) & $\chi^2 = 12.47$ & 0.127 & \cmark PASS \\
Distribuzione Poisson & KS = 0.089 & 0.234 & \cmark PASS \\
Correlazione importo-articoli & r = 0.62 & $<0.001$ & \cmark PASS \\
Test stagionalità & $F = 8.34$ & $<0.001$ & \cmark PASS \\
Completezza dati & missing = 0.0\% & - & \cmark PASS \\
\midrule
\multicolumn{3}{l}{\textbf{Test superati: 16/18}} & \textbf{88.9\%} \\
\bottomrule
\end{tabular}
\end{table}

\section{\texorpdfstring{Protocollo Etico}{A.4 - Protocollo Etico}}

La ricerca ha ricevuto approvazione del Comitato Etico Universitario (Protocollo n. 2023/147) con garanzie di:

\begin{enumerate}
    \item Anonimizzazione completa dei dati aziendali
    \item Aggregazione minima di 5 organizzazioni per statistiche pubblicate
    \item Non divulgazione di vulnerabilità specifiche non remediate
    \item K-anonimity garantita con $k \geq 5$ per tutti i dataset
\end{enumerate}

\section{\texorpdfstring{Limitazioni Metodologiche}{A.5 - Limitazioni Metodologiche}}

Le principali limitazioni identificate includono:

\begin{itemize}
    \item \textbf{Bias di selezione}: Focus su organizzazioni con maturità IT sufficiente per partecipare alla ricerca
    \item \textbf{Validità temporale}: Dati calibrati su periodo 2019-2025, necessario aggiornamento periodico
    \item \textbf{Generalizzabilità}: Risultati specifici per il contesto italiano della GDO
    \item \textbf{Completezza simulazione}: Digital Twin non replica tutte le complessità operative reali
\end{itemize}
