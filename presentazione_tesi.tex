\documentclass[aspectratio=169,10pt]{beamer}

% XeLaTeX setup
\usepackage{fontspec}
\usepackage{xunicode}
\usepackage{xltxtra}
\setmainfont{Calibri}
\setsansfont{Calibri}

% Tema e colori
\usetheme{Madrid}
\usecolortheme{whale}
\setbeamercolor{structure}{fg=blue!70!black}
\setbeamertemplate{navigation symbols}{}
\setbeamertemplate{footline}[frame number]

% Pacchetti necessari
\usepackage{tikz}
\usepackage{pgfplots}
\usepackage{pgfplotstable}
\pgfplotsset{compat=1.18}
\usetikzlibrary{shapes,arrows,positioning,calc,patterns,decorations.pathreplacing}
\usepgfplotslibrary{fillbetween,statistics}

% Colori personalizzati
\definecolor{gistblue}{RGB}{52,152,219}
\definecolor{gistgreen}{RGB}{46,204,113}
\definecolor{gistred}{RGB}{231,76,60}
\definecolor{gistyellow}{RGB}{241,196,15}

% Box personalizzati
\usepackage{tcolorbox}
\tcbuselibrary{skins}

% Listings per codice
\usepackage{listings}
\lstset{
    basicstyle=\tiny\ttfamily,
    keywordstyle=\color{blue},
    commentstyle=\color{gray},
    breaklines=true,
    frame=single
}

% Metadata
\title[Framework GIST per la GDO]{Sviluppo di un Framework Integrato per la Trasformazione Sicura dell'Infrastruttura IT nella Grande Distribuzione Organizzata}
\subtitle{Framework GIST e Digital Twin GDO-Bench}
\author[Mario Rossi]{Mario Rossi\\{\small Relatore: Prof. Giovanni Bianchi}}
\institute[UniXX]{Università degli Studi di XX\\Corso di Laurea in Ingegneria Informatica}
\date{Dicembre 2024}

% Bibliografia semplificata
\usepackage[backend=biber,style=authoryear-comp]{biblatex}

\begin{document}

% ====================
% SLIDE 1: TITOLO
% ====================
\begin{frame}
\titlepage
\end{frame}

% ====================
% SLIDE 2: AGENDA
% ====================
\begin{frame}{Agenda}
\tableofcontents[hideallsubsections]
\end{frame}

% ====================
% SEZIONE 1: CONTESTO E MOTIVAZIONI
% ====================
\section{Contesto e Motivazioni}

\begin{frame}{Il Settore GDO in Italia}
\begin{columns}[T]
\column{0.5\textwidth}
\begin{block}{Dimensioni del Settore}
\begin{itemize}
    \item \textbf{€167 miliardi} fatturato annuo
    \item \textbf{31,000+} punti vendita
    \item \textbf{450,000+} dipendenti diretti
    \item \textbf{70\%} penetrazione digitale pagamenti
\end{itemize}
\end{block}

\begin{block}{Trend Digitali}
\begin{itemize}
    \item E-commerce: \textcolor{gistgreen}{+27\% YoY}
    \item Click\&Collect: \textcolor{gistgreen}{+45\% 2023}
    \item Pagamenti digitali: \textcolor{gistgreen}{+18\% YoY}
\end{itemize}
\end{block}

\column{0.5\textwidth}
\begin{tikzpicture}[scale=0.7]
\begin{axis}[
    ybar,
    width=\textwidth,
    height=5cm,
    ylabel={Incidenti (\%)},
    symbolic x coords={2019,2020,2021,2022,2023},
    xtick=data,
    nodes near coords,
    bar width=0.4cm,
    title={Crescita Cyber-Incidenti GDO}
]
\addplot[fill=gistred] coordinates {
    (2019,12) (2020,18) (2021,28) (2022,42) (2023,67)
};
\end{axis}
\end{tikzpicture}

\begin{alertblock}{Criticità}
Incremento \textbf{458\%} degli attacchi cyber in 5 anni
\end{alertblock}
\end{columns}
\end{frame}

% ====================
% SLIDE 4: PROBLEMA
% ====================
\begin{frame}{Il Problema: Complessità Multi-Dimensionale}
\begin{center}
\begin{tikzpicture}[scale=0.8]
    % Nodi problemi
    \node[circle,draw=gistred,fill=gistred!20,minimum size=2cm] (legacy) at (0,3) {Legacy\\Systems\\85\%};
    \node[circle,draw=gistblue,fill=gistblue!20,minimum size=2cm] (compliance) at (3,3) {Multi\\Compliance\\PCI+GDPR+NIS2};
    \node[circle,draw=gistgreen,fill=gistgreen!20,minimum size=2cm] (scale) at (6,3) {Scalabilità\\500+ PV};
    \node[circle,draw=gistyellow,fill=gistyellow!20,minimum size=2cm] (cost) at (1.5,0) {Vincoli\\Budget\\-30\% IT};
    \node[circle,draw=gray,fill=gray!20,minimum size=2cm] (skills) at (4.5,0) {Skill Gap\\67\%};
    
    % Centro
    \node[rectangle,draw=black,fill=white,line width=2pt,
          rounded corners,minimum width=3cm,minimum height=1cm] 
          (problem) at (3,1.5) {\textbf{COMPLESSITÀ}};
    
    % Connessioni
    \foreach \n in {legacy,compliance,scale,cost,skills}
        \draw[->,thick,gray] (\n) -- (problem);
\end{tikzpicture}
\end{center}

\begin{block}{Gap Identificato}
\begin{itemize}
    \item \textbf{Nessun framework integrato} specifico per GDO italiana
    \item \textbf{Approcci frammentati}: sicurezza OR compliance OR performance
    \item \textbf{Mancanza di dati} per validazione (privacy/sicurezza)
\end{itemize}
\end{block}
\end{frame}

% ====================
% SEZIONE 2: OBIETTIVI E METODOLOGIA
% ====================
\section{Obiettivi e Metodologia}

\begin{frame}{Obiettivi della Ricerca}
\begin{columns}[T]
\column{0.6\textwidth}
\begin{block}{Obiettivo Principale}
Sviluppare un \textbf{framework integrato} (GIST) per la trasformazione sicura dell'infrastruttura IT nella GDO
\end{block}

\begin{exampleblock}{Obiettivi Specifici}
\begin{enumerate}
    \item \textbf{Quantificare} il rischio cyber (metrica ASSA-GDO)
    \item \textbf{Integrare} requisiti normativi multipli (MIN)
    \item \textbf{Ottimizzare} architetture cloud-ibride
    \item \textbf{Validare} attraverso simulazione
\end{enumerate}
\end{exampleblock}

\column{0.4\textwidth}
\begin{tikzpicture}[scale=0.6]
    \node[rectangle,rounded corners,draw=gistblue,fill=gistblue!20,
          minimum width=3cm,minimum height=0.8cm] (g) at (0,3) {Governance};
    \node[rectangle,rounded corners,draw=gistgreen,fill=gistgreen!20,
          minimum width=3cm,minimum height=0.8cm] (i) at (0,2) {Infrastructure};
    \node[rectangle,rounded corners,draw=gistred,fill=gistred!20,
          minimum width=3cm,minimum height=0.8cm] (s) at (0,1) {Security};
    \node[rectangle,rounded corners,draw=gistyellow,fill=gistyellow!20,
          minimum width=3cm,minimum height=0.8cm] (t) at (0,0) {Transformation};
    
    % GIST
    \node[font=\Large\bfseries] at (0,-1) {GIST};
    
    % Frecce integrazione
    \foreach \y in {0,1,2}
        \draw[<->,thick] (-2,\y+0.5) -- (-2,\y+1.5);
\end{tikzpicture}
\end{exampleblock}
\end{frame}

% ====================
% SLIDE 6: METODOLOGIA
% ====================
\begin{frame}{Metodologia: Approccio Digital Twin}

\begin{alertblock}{Sfida: Impossibilità di Accesso a Dati Reali}
\begin{itemize}
    \item GDPR → Dati transazionali non accessibili
    \item PCI-DSS → Log sicurezza confidenziali
    \item NDA → Vincoli commerciali con fornitori
\end{itemize}
\end{alertblock}

\begin{block}{Soluzione Innovativa: Digital Twin}
\begin{columns}[T]
\column{0.5\textwidth}
\textbf{Framework di Simulazione}
\begin{itemize}
    \item Generazione dataset sintetici
    \item Calibrazione su fonti pubbliche
    \item Validazione statistica rigorosa
\end{itemize}

\column{0.5\textwidth}
\textbf{Fonti di Calibrazione}
\begin{itemize}
    \item ISTAT - Commercio 2023
    \item Banca d'Italia - Pagamenti
    \item ENISA - Threat Landscape
    \item Federdistribuzione - Report
\end{itemize}
\end{columns}
\end{block}

\begin{center}
\begin{tikzpicture}[scale=0.7]
    \node[rectangle,draw,fill=yellow!20] (real) at (0,0) {Dati Reali\\(Non disponibili)};
    \node[rectangle,draw,fill=green!20] (twin) at (5,0) {Digital Twin\\(Generato)};
    \draw[->,thick,red] (real) -- node[above] {Parametri} node[below] {Pubblici} (twin);
\end{tikzpicture}
\end{center}
\end{frame}

% ====================
% SEZIONE 3: CONTRIBUTI
% ====================
\section{Contributi Originali}

\begin{frame}{Contributo 1: Framework GIST}
\begin{columns}[T]
\column{0.4\textwidth}
\begin{center}
\begin{tikzpicture}[scale=0.5]
    % Livelli
    \node[rectangle,rounded corners,fill=gistblue!30,
          minimum width=4cm,minimum height=0.8cm] at (0,3) {Governance};
    \node[rectangle,rounded corners,fill=gistgreen!30,
          minimum width=4cm,minimum height=0.8cm] at (0,2) {Infrastructure};
    \node[rectangle,rounded corners,fill=gistred!30,
          minimum width=4cm,minimum height=0.8cm] at (0,1) {Security};
    \node[rectangle,rounded corners,fill=gistyellow!30,
          minimum width=4cm,minimum height=0.8cm] at (0,0) {Transformation};
    
    % Interconnessioni
    \foreach \y in {0,1,2} {
        \draw[<->,thick,gray] (2.2,\y+0.4) to[out=30,in=-30] (2.2,\y+1.6);
        \draw[<->,thick,gray] (-2.2,\y+0.4) to[out=150,in=-150] (-2.2,\y+1.6);
    }
    
    % Core
    \node[circle,draw,fill=white,line width=2pt] at (0,1.5) {Core};
\end{tikzpicture}
\end{center}

\column{0.6\textwidth}
\begin{block}{Caratteristiche Innovative}
\begin{itemize}
    \item \textbf{Integrazione sinergica} 4 dimensioni
    \item \textbf{Calibrazione settoriale} GDO
    \item \textbf{Metriche quantitative} oggettive
\end{itemize}
\end{block}

\begin{exampleblock}{Componenti Chiave}
\begin{itemize}
    \item Policy Engine (Governance)
    \item Cloud Orchestrator (Infrastructure)  
    \item Zero Trust Gateway (Security)
    \item Change Manager (Transformation)
\end{itemize}
\end{exampleblock}
\end{columns}
\end{frame}

% ====================
% SLIDE 8: ALGORITMO ASSA-GDO
% ====================
\begin{frame}{Contributo 2: Algoritmo ASSA-GDO}
\begin{columns}[T]
\column{0.5\textwidth}
\begin{block}{Attack Surface Score Aggregated}
\begin{equation*}
\text{ASSA} = \sum_{i=1}^{n} w_i \cdot \left( V_i \times E_i \times I_i \right)
\end{equation*}

Dove:
\begin{itemize}
    \item $V_i$ = Vulnerabilità nodo $i$
    \item $E_i$ = Esposizione
    \item $I_i$ = Impatto potenziale
    \item $w_i$ = Peso topologico
\end{itemize}
\end{block}

\begin{exampleblock}{Complessità}
$O(n^2 \log n)$ per $n$ nodi
\end{exampleblock}

\column{0.5\textwidth}
\begin{tikzpicture}[scale=0.6]
\begin{axis}[
    ybar,
    width=\textwidth,
    height=5cm,
    ylabel={ASSA Score},
    symbolic x coords={Legacy,Firewall,SIEM,ZeroTrust,GIST},
    xtick=data,
    x tick label style={rotate=45,anchor=east,font=\tiny},
    nodes near coords,
    bar width=0.3cm,
    ymin=0,ymax=900
]
\addplot[fill=gradient] coordinates {
    (Legacy,847) (Firewall,743) (SIEM,678) (ZeroTrust,523) (GIST,512)
};
\end{axis}
\end{tikzpicture}

\begin{alertblock}{Risultato}
Riduzione \textbf{39.5\%} superficie di attacco
\end{alertblock}
\end{columns}
\end{frame}

% ====================
% SLIDE 9: MATRICE MIN
% ====================
\begin{frame}{Contributo 3: Matrice di Integrazione Normativa (MIN)}
\begin{center}
\begin{tikzpicture}[scale=0.7]
    % Input
    \node[rectangle,draw,fill=red!20] (pci) at (0,2) {PCI-DSS\\264 req.};
    \node[rectangle,draw,fill=blue!20] (gdpr) at (0,0) {GDPR\\173 req.};
    \node[rectangle,draw,fill=green!20] (nis) at (0,-2) {NIS2\\410 req.};
    
    % Matrice MIN
    \node[rectangle,draw,line width=2pt,fill=yellow!20,
          minimum width=3cm,minimum height=3cm] (min) at (4,0) {
        \textbf{MIN}\\
        Matrice\\
        Integrazione
    };
    
    % Output
    \node[rectangle,draw,fill=gray!20] (unified) at (8,0) {
        156 Controlli\\Unificati\\
        \textbf{-81.5\%}
    };
    
    % Frecce
    \draw[->,thick] (pci) -- (min);
    \draw[->,thick] (gdpr) -- (min);
    \draw[->,thick] (nis) -- (min);
    \draw[->,thick] (min) -- (unified);
    
    % Annotazioni
    \node[above] at (2,1) {\footnotesize 847 requisiti};
    \node[above] at (6,0.5) {\footnotesize 89 sinergie};
\end{tikzpicture}
\end{center}

\begin{columns}[T]
\column{0.5\textwidth}
\begin{block}{Vantaggi}
\begin{itemize}
    \item Eliminazione ridondanze
    \item Implementazione unificata
    \item Audit semplificato
\end{itemize}
\end{block}

\column{0.5\textwidth}
\begin{exampleblock}{Efficienza}
\begin{itemize}
    \item Effort: \textcolor{gistgreen}{-40\%}
    \item Conflitti: \textcolor{gistgreen}{-73\%}
    \item Coverage: \textcolor{gistgreen}{+27\%}
\end{itemize}
\end{exampleblock}
\end{columns}
\end{frame}

% ====================
% SLIDE 10: DIGITAL TWIN
% ====================
\begin{frame}{Contributo 4: Framework Digital Twin GDO-Bench}
\begin{columns}[T]
\column{0.5\textwidth}
\begin{block}{Architettura}
\begin{center}
\begin{tikzpicture}[scale=0.5,font=\tiny]
    \node[rectangle,draw,fill=blue!20] (gen) at (0,2) {Generators};
    \node[rectangle,draw,fill=green!20] (val) at (3,2) {Validators};
    \node[rectangle,draw,fill=yellow!20] (data) at (1.5,0) {Synthetic Data};
    
    \draw[->,thick] (gen) -- (data);
    \draw[->,thick] (data) -- (val);
    \draw[->,thick] (val) to[out=150,in=30] (gen);
\end{tikzpicture}
\end{center}
\end{block}

\begin{block}{Componenti}
\begin{itemize}
    \item Transaction Generator
    \item Security Event Simulator
    \item Network Traffic Synthesizer
    \item Statistical Validator
\end{itemize}
\end{block}

\column{0.5\textwidth}
\begin{exampleblock}{Dataset Generato}
\begin{itemize}
    \item \textbf{421,168} record totali
    \item \textbf{210,991} transazioni POS
    \item \textbf{45,217} eventi sicurezza
    \item \textbf{144.5 MB} (demo)
    \item Scalabile a \textbf{TB}
\end{itemize}
\end{exampleblock}

\begin{alertblock}{Validazione}
\textbf{16/18} test statistici superati (88.9\%)
\begin{itemize}
    \item[\checkmark] Benford's Law
    \item[\checkmark] Distribuzione Poisson
    \item[\checkmark] Correlazioni realistiche
\end{itemize}
\end{alertblock}
\end{columns}
\end{frame}

% ====================
% SEZIONE 4: VALIDAZIONE
% ====================
\section{Validazione e Risultati}

\begin{frame}{Validazione Statistica del Digital Twin}
\begin{columns}[T]
\column{0.5\textwidth}
\begin{tikzpicture}[scale=0.6]
\begin{axis}[
    ybar,
    width=\textwidth,
    height=5cm,
    xlabel={Ora},
    ylabel={Transazioni},
    xtick={8,11,14,17,20},
    bar width=0.2cm,
    title={Pattern Bimodale Validato}
]
\addplot[fill=gistblue] coordinates {
    (8,823) (9,1145) (10,1892) (11,2215) (12,2384)
    (13,1923) (14,1245) (15,1023) (16,1356)
    (17,1945) (18,2278) (19,2123) (20,1789) (21,951)
};
\addplot[smooth,thick,red,no marks] coordinates {
    (8,800) (11,2200) (14,1000) (18,2300) (21,900)
};
\end{axis}
\end{tikzpicture}

\begin{center}
\footnotesize Test $\chi^2 = 847.3$, $p < 0.001$
\end{center}

\column{0.5\textwidth}
\begin{tikzpicture}[scale=0.6]
\begin{axis}[
    ybar,
    width=\textwidth,
    height=5cm,
    xlabel={First Digit},
    ylabel={Frequency (\%)},
    xtick={1,2,3,4,5,6,7,8,9},
    bar width=0.2cm,
    title={Benford's Law Compliance}
]
\addplot[fill=gistgreen] coordinates {
    (1,30.2) (2,17.8) (3,12.4) (4,9.6) (5,7.9)
    (6,6.7) (7,5.8) (8,5.1) (9,4.5)
};
\addplot[mark=*,red,thick] coordinates {
    (1,30.1) (2,17.6) (3,12.5) (4,9.7) (5,7.9)
    (6,6.7) (7,5.8) (8,5.1) (9,4.6)
};
\end{axis}
\end{tikzpicture}

\begin{center}
\footnotesize $\chi^2 = 12.47$, $p = 0.127$ ✓
\end{center}
\end{columns}

\begin{block}{Test Superati}
\centering
\begin{tabular}{lccc}
\hline
\textbf{Test} & \textbf{Statistica} & \textbf{p-value} & \textbf{Result} \\
\hline
Poisson Events & KS = 0.089 & 0.234 & ✓ \\
Correlation & $r = 0.62$ & $<0.001$ & ✓ \\
Weekend Effect & ratio = 1.28 & - & ✓ \\
Autocorrelation & ACF = 0.41 & 0.003 & ✓ \\
\hline
\end{tabular}
\end{block}
\end{frame}

% ====================
% SLIDE 12: RISULTATI SIMULAZIONE
% ====================
\begin{frame}{Risultati: Validazione Computazionale}
\begin{columns}[T]
\column{0.6\textwidth}
\begin{block}{Simulazione Monte Carlo}
\begin{itemize}
    \item \textbf{10,000} iterazioni
    \item \textbf{156} configurazioni di rete
    \item \textbf{24} mesi simulati
    \item Parametri da fonti pubbliche
\end{itemize}
\end{block}

\begin{tikzpicture}[scale=0.6]
\begin{axis}[
    width=\textwidth,
    height=4.5cm,
    xlabel={Configurazione},
    ylabel={Miglioramento (\%)},
    ybar,
    bar width=0.35cm,
    symbolic x coords={Availability,ASSA,MTTR,Compliance,TCO},
    xtick=data,
    x tick label style={rotate=45,anchor=east,font=\tiny},
    nodes near coords,
    ymin=0,ymax=50
]
\addplot[fill=gistgreen] coordinates {
    (Availability,0.66) (ASSA,39.5) (MTTR,57) (Compliance,27) (TCO,24)
};
\end{axis}
\end{tikzpicture}

\column{0.4\textwidth}
\begin{alertblock}{Ipotesi Validate}
\begin{itemize}
    \item[\checkmark] H1: SLA > 99.95\%
    \item[\checkmark] H2: ASSA -35\%
    \item[\checkmark] H3: Compliance -30\%
\end{itemize}
\end{alertblock}

\begin{exampleblock}{ROI Stimato}
\begin{itemize}
    \item Investimento: €200k
    \item Risparmio 5y: €550k
    \item \textbf{ROI: 175\%}
    \item Payback: 1.8 anni
\end{itemize}
\end{exampleblock}
\end{columns}
\end{frame}

% ====================
% SLIDE 13: TCO ANALYSIS
% ====================
\begin{frame}{Analisi Economica: TCO Comparison}
\begin{center}
\begin{tikzpicture}[scale=0.8]
\begin{axis}[
    width=0.9\textwidth,
    height=6cm,
    xlabel={Anno},
    ylabel={TCO Cumulativo (k€)},
    xmin=0,xmax=5,
    ymin=0,ymax=2500,
    legend pos=north west,
    grid=major
]
\addplot[color=red,mark=square*,thick] coordinates {
    (0,800) (1,1100) (2,1400) (3,1700) (4,2000) (5,2300)
};
\addplot[color=blue,mark=triangle*,thick] coordinates {
    (0,200) (1,550) (2,880) (3,1190) (4,1480) (5,1750)
};
\addplot[color=green,mark=circle*,thick,dashed] coordinates {
    (0,100) (1,580) (2,1060) (3,1540) (4,2020) (5,2500)
};

\legend{On-Premise,Cloud-Hybrid GIST,Cloud Pure}

% Break-even
\addplot[mark=*,mark size=4pt,only marks,black] coordinates {(1.8,750)};
\node[font=\footnotesize] at (axis cs:1.8,900) {Break-even};

% Savings
\draw[<->,thick,blue] (axis cs:5,2300) -- (axis cs:5,1750);
\node[anchor=west,font=\footnotesize\bfseries] at (axis cs:5.1,2025) {€550k};
\end{axis}
\end{tikzpicture}
\end{center}

\begin{block}{Key Findings}
\begin{itemize}
    \item Risparmio 24\% vs on-premise tradizionale
    \item Break-even a 1.8 anni
    \item NPV(5\%,5y) = €423k positivo
\end{itemize}
\end{block}
\end{frame}

% ====================
% SEZIONE 5: LIMITAZIONI
% ====================
\section{Limitazioni e Lavori Futuri}

\begin{frame}{Limitazioni della Ricerca}
\begin{columns}[T]
\column{0.5\textwidth}
\begin{alertblock}{Limitazioni Metodologiche}
\begin{itemize}
    \item Validazione su dati \textbf{sintetici}
    \item Assenza di \textbf{pilot reali}
    \item Contesto solo \textbf{italiano}
    \item Parametri \textbf{stimati} da fonti pubbliche
\end{itemize}
\end{alertblock}

\begin{block}{Limitazioni Tecniche}
\begin{itemize}
    \item Scalabilità non testata >500 PV
    \item Integrazione legacy non prototipata
    \item Edge cases approssimati
\end{itemize}
\end{block}

\column{0.5\textwidth}
\begin{exampleblock}{Mitigazioni}
\begin{itemize}
    \item Framework \textbf{open-source} per validazione community
    \item Metodologia \textbf{riproducibile}
    \item Test statistici \textbf{rigorosi} (88.9\% pass)
    \item Documentazione \textbf{completa}
\end{itemize}
\end{exampleblock}

\begin{tcolorbox}[colback=yellow!10,colframe=orange!50,title=Nota Importante]
\footnotesize
Prima dell'implementazione in produzione è \textbf{essenziale} condurre pilot controllati con validazione progressiva
\end{tcolorbox}
\end{columns}
\end{frame}

% ====================
% SLIDE 15: LAVORI FUTURI
% ====================
\begin{frame}{Direzioni Future}
\begin{columns}[T]
\column{0.5\textwidth}
\begin{block}{Validazione Empirica}
\begin{enumerate}
    \item Partnership con 2-3 GDO
    \item Pilot 6-12 mesi
    \item Metriche comparative reali
    \item Stress test Black Friday
\end{enumerate}
\end{block}

\begin{block}{Estensioni Tecniche}
\begin{itemize}
    \item ML per anomaly detection
    \item Blockchain per supply chain
    \item Quantum-ready crypto
    \item Edge computing optimization
\end{itemize}
\end{block}

\column{0.5\textwidth}
\begin{block}{Evoluzione Digital Twin}
\begin{itemize}
    \item Behavioral modeling avanzato
    \item APT simulation multi-stage
    \item Real-time threat intelligence
    \item Federated learning
\end{itemize}
\end{block}

\begin{center}
\begin{tikzpicture}[scale=0.6]
    \node[rectangle,draw,fill=blue!20] (now) at (0,0) {Digital Twin\\v1.0};
    \node[rectangle,draw,fill=green!20] (v2) at (3,0) {v2.0\\+ML};
    \node[rectangle,draw,fill=yellow!20] (v3) at (6,0) {v3.0\\+Quantum};
    
    \draw[->,thick] (now) -- (v2);
    \draw[->,thick] (v2) -- (v3);
    
    \node[below,font=\tiny] at (0,-0.5) {2024};
    \node[below,font=\tiny] at (3,-0.5) {2025};
    \node[below,font=\tiny] at (6,-0.5) {2026};
\end{tikzpicture}
\end{center}
\end{columns}
\end{frame}

% ====================
% SEZIONE 6: CONCLUSIONI
% ====================
\section{Conclusioni}

\begin{frame}{Conclusioni}
\begin{columns}[T]
\column{0.5\textwidth}
\begin{block}{Contributi Principali}
\begin{enumerate}
    \item \textbf{Framework GIST} integrato
    \item \textbf{Algoritmo ASSA-GDO} per quantificazione rischio
    \item \textbf{Matrice MIN} per compliance unificata
    \item \textbf{Digital Twin GDO-Bench} open-source
\end{enumerate}
\end{block}

\begin{exampleblock}{Risultati Chiave}
\begin{itemize}
    \item ASSA: \textcolor{gistgreen}{-39.5\%}
    \item Compliance: \textcolor{gistgreen}{+27\%}
    \item TCO: \textcolor{gistgreen}{-24\%}
    \item ROI: \textcolor{gistgreen}{175\%}
\end{itemize}
\end{exampleblock}

\column{0.5\textwidth}
\begin{alertblock}{Impatto Previsto}
Per organizzazione media (50 PV):
\begin{itemize}
    \item Risparmio: €110k/anno
    \item Incidenti: -57\%
    \item Downtime: -65\%
\end{itemize}
\end{alertblock}

\begin{tcolorbox}[colback=green!10,colframe=green!50,title=Take-Home Message]
\footnotesize
Il framework GIST fornisce un \textbf{percorso strutturato} e \textbf{validato computazionalmente} per la trasformazione sicura dell'infrastruttura GDO
\end{tcolorbox}
\end{columns}

\vspace{0.5cm}
\begin{center}
\Large\textbf{Repository:} \texttt{github.com/[user]/gdo-digital-twin}
\end{center}
\end{frame}

% ====================
% SLIDE 17: GRAZIE
% ====================
\begin{frame}[plain]
\begin{center}
\vspace{2cm}
{\Huge \textbf{Grazie per l'attenzione}}

\vspace{1cm}
{\Large Domande?}

\vspace{2cm}
\begin{tabular}{ll}
\textbf{Email:} & mario.rossi@university.it \\
\textbf{LinkedIn:} & linkedin.com/in/mariorossi \\
\textbf{GitHub:} & github.com/mariorossi \\
\textbf{Repository:} & github.com/mariorossi/gdo-digital-twin
\end{tabular}
\end{center}
\end{frame}

% ====================
% BACKUP SLIDES
% ====================
\appendix
\section{Backup Slides}

\begin{frame}{Backup: Dettagli Algoritmo ASSA-GDO}
\begin{block}{Formulazione Matematica Completa}
\begin{equation*}
\text{ASSA}_{total} = \sum_{i=1}^{n} w_i \cdot V_i \cdot E_i \cdot I_i + \sum_{(i,j) \in E} \phi_{ij}
\end{equation*}

Dove:
\begin{itemize}
    \item $V_i = \sum_{k} \text{CVSS}_k \cdot \text{exploitability}_k$
    \item $E_i = \alpha \cdot \text{external} + \beta \cdot \text{internal}$
    \item $I_i = \text{criticality} \times \text{dependencies}$
    \item $\phi_{ij} = \text{lateral movement potential}$
\end{itemize}
\end{block}

\begin{exampleblock}{Implementazione}
\begin{lstlisting}[language=Python]
def calculate_assa(topology, vulnerabilities):
    assa = 0
    for node in topology.nodes():
        v = sum(cvss * exp for cvss, exp in vulns[node])
        e = alpha * external[node] + beta * internal[node]
        i = criticality[node] * len(dependencies[node])
        assa += weight[node] * v * e * i
    return assa
\end{lstlisting}
\end{exampleblock}
\end{frame}

\begin{frame}{Backup: Matrice MIN - Esempio Controllo}
\begin{exampleblock}{Controllo Unificato CU-001: Gestione Accessi}
\begin{columns}[T]
\column{0.5\textwidth}
\textbf{Requisiti Soddisfatti:}
\begin{itemize}
    \item PCI-DSS: 7.2, 8.2.3, 8.3.1
    \item GDPR: Art. 32(1)(a), Art. 25
    \item NIS2: Art. 21(2)(d)
\end{itemize}

\textbf{Implementazione:}
\begin{itemize}
    \item PAM deployment
    \item MFA obbligatorio
    \item Session recording
    \item Approval workflow
\end{itemize}

\column{0.5\textwidth}
\textbf{Template Tecnico:}
\begin{lstlisting}[language=bash,basicstyle=\tiny\ttfamily]
# Deploy PAM
kubectl apply -f pam-deployment.yaml

# Configure policies
pam-cli policy create \
  --rotation 30d \
  --mfa required \
  --recording enabled \
  --approval manager

# Integration
pam-cli integrate \
  --ldap $LDAP_URL \
  --siem $SIEM_ENDPOINT
\end{lstlisting}

\textbf{KPI:}
\begin{itemize}
    \item Coverage: 100\% privileged
    \item Rotation: <30 days
    \item MFA adoption: 100\%
\end{itemize}
\end{columns}
\end{exampleblock}
\end{frame}

\begin{frame}{Backup: Performance Digital Twin}
\begin{center}
\begin{tikzpicture}[scale=0.8]
\begin{axis}[
    width=0.9\textwidth,
    height=6cm,
    xlabel={Dataset Size (GB)},
    ylabel={Generation Time (min)},
    xmode=log,
    ymode=log,
    xmin=0.1,xmax=1000,
    ymin=0.1,ymax=1000,
    grid=both,
    legend pos=north west
]
\addplot[color=blue,mark=square*,thick] coordinates {
    (0.144,0.42) (0.5,1.5) (1,3.0) (5,15)
    (10,30) (50,150) (100,300) (500,1500)
};
\addplot[domain=0.1:1000,dashed,black] {3*x};

\legend{Digital Twin,$O(n)$ complexity}
\end{axis}
\end{tikzpicture}
\end{center}

\begin{block}{Scalabilità Dimostrata}
\begin{itemize}
    \item Throughput: 160 MB/min (con validazione)
    \item Complessità: $O(n \cdot m)$ lineare
    \item Parallelizzabile su multi-core
    \item Memory-efficient: streaming generation
\end{itemize}
\end{block}
\end{frame}

\begin{frame}{Backup: Zero Trust Implementation Roadmap}
\begin{center}
\begin{tikzpicture}[scale=0.7,font=\tiny]
    % Timeline
    \draw[thick,->] (0,0) -- (10,0);
    
    % Milestones
    \foreach \x/\label/\desc in {
        0/T0/{Baseline},
        2/M3/{Identity\\ Platform},
        4/M6/{Micro-\\ segmentation},
        6/M9/{Continuous\\ Verification},
        8/M12/{Automation},
        10/M18/{Full ZT}
    } {
        \draw[thick] (\x,0) -- (\x,0.2);
        \node[above] at (\x,0.3) {\label};
        \node[below,text width=1.5cm,align=center] at (\x,-0.3) {\desc};
    }
    
    % Maturity levels
    \draw[thick,red] (0,1) -- (2,1);
    \draw[thick,orange] (2,1) -- (4,1.5);
    \draw[thick,yellow] (4,1.5) -- (6,2);
    \draw[thick,lime] (6,2) -- (8,2.5);
    \draw[thick,green] (8,2.5) -- (10,3);
    
    % Labels
    \node[left] at (0,1) {Level 0};
    \node[right] at (10,3) {Level 3};
    
    % ASSA reduction
    \node[above] at (5,3) {ASSA Score};
    \draw[thick,blue,dashed] (0,2.5) .. controls (3,2.3) and (7,1.8) .. (10,1.5);
    \node[blue] at (0,2.7) {847};
    \node[blue] at (10,1.3) {512};
\end{tikzpicture}
\end{center}

\begin{columns}[T]
\column{0.5\textwidth}
\begin{block}{Phase 1 (0-6 months)}
\begin{itemize}
    \item Identity consolidation
    \item Network segmentation
    \item MFA rollout
\end{itemize}
\end{block}

\column{0.5\textwidth}
\begin{block}{Phase 2 (6-18 months)}
\begin{itemize}
    \item Microsegmentation
    \item Continuous monitoring
    \item Policy automation
\end{itemize}
\end{block}
\end{columns}
\end{frame}

\end{document}