% ============================================================================
% MAPPATURA DELLE CITAZIONI PER SEZIONE DEL CAPITOLO 3
% Guida per l'inserimento dei riferimenti bibliografici nel testo
% ============================================================================

\documentclass{article}
\usepackage{longtable}
\usepackage{array}
\usepackage{booktabs}
\usepackage[italian]{babel}
\usepackage{tcolorbox}
\usepackage{amssymb}

\begin{document}

\section*{Guida all'Inserimento delle Citazioni nel Capitolo 3}

\begin{tcolorbox}[colback=blue!5!white,colframe=blue!75!black,title=\textbf{Istruzioni per l'Uso}]
Questa tabella indica quali riferimenti bibliografici citare in ogni sezione del Capitolo 3. 
Nel testo LaTeX, utilizzare \texttt{\textbackslash cite\{riferimento\}} per le citazioni.
\end{tcolorbox}

\begin{longtable}{p{0.35\textwidth}|p{0.35\textwidth}|p{0.25\textwidth}}
\toprule
\textbf{Sezione/Sottosezione} & \textbf{Contenuto da Citare} & \textbf{Riferimenti Suggeriti} \\
\midrule
\endfirsthead
\multicolumn{3}{c}{\textit{Continuazione dalla pagina precedente}} \\
\toprule
\textbf{Sezione/Sottosezione} & \textbf{Contenuto da Citare} & \textbf{Riferimenti Suggeriti} \\
\midrule
\endhead
\midrule
\multicolumn{3}{r}{\textit{Continua nella pagina successiva}} \\
\endfoot
\bottomrule
\endlastfoot

% SEZIONE 3.1
\textbf{3.1 Introduzione e Framework Teorico} & & \\
\midrule
3.1.1 Posizionamento nel Contesto & & \\
- 78\% attacchi su vulnerabilità architetturali & Dato threat landscape & \texttt{enisa2024} \\
- Teoria sistemi distribuiti & Framework concettuale & \texttt{coulouris2023, tanenbaum2023} \\
- 47 studi aggregati & Metodologia ricerca & \texttt{zhang2024} \\
- 23 report di settore & Fonti industriali & \texttt{gartner2024cloud, idc2024} \\
\midrule
3.1.2 Modello Teorico Evoluzione & & \\
- Funzione di transizione E(t) & Modello matematico & \texttt{klems2023} \\
- Calibrazione coefficienti & Validazione empirica & \texttt{martens2024} \\
- R²=0.87 & Capacità predittiva & \texttt{dataset2024} \\
\midrule

% SEZIONE 3.2
\textbf{3.2 Infrastruttura Fisica} & & \\
\midrule
3.2.1 Affidabilità Sistemi Alimentazione & & \\
- 127 guasti documentati & Database incidenti & \texttt{avizienis2023} \\
- MTBF configurazioni N+1, 2N & Standard affidabilità & \texttt{iso27001} \\
- Power Management con ML & Innovazione predittiva & \texttt{forrester2024} \\
- 31\% incremento affidabilità & Risultati ML & \texttt{survey2024} \\
\midrule
3.2.2 Ottimizzazione Raffreddamento & & \\
- PUE metriche & Standard efficienza & \texttt{enisa2023cloud} \\
- Free cooling analysis & Best practices & \texttt{cisco2024} \\
- Liquid cooling ROI & Analisi economica & \texttt{benchmark2023} \\
\midrule

% SEZIONE 3.3
\textbf{3.3 Evoluzione Architetture di Rete} & & \\
\midrule
3.3.1 Analisi Topologie & & \\
- Hub-and-spoke limitations & Architetture legacy & \texttt{michel2023} \\
- SD-WAN benefits & Modernizzazione rete & \texttt{wood2024} \\
- MTTR 4.7h → 1.2h & Metriche performance & \texttt{cisco2024} \\
- 67\% traffico non ispezionato & Security gaps & \texttt{enisa2024} \\
\midrule
3.3.2 Edge Computing & & \\
- Latenza <100ms requirement & SLA pagamenti & \texttt{pcidss2024} \\
- 67\% riduzione latenza & Edge benefits & \texttt{satyanarayanan2023} \\
- Modello gerarchico 3 livelli & Architettura edge & \texttt{shi2024} \\
- 73\% riduzione traffico cloud & Ottimizzazione bandwidth & \texttt{awsdocs2024} \\
\midrule

% SEZIONE 3.4
\textbf{3.4 Trasformazione Cloud} & & \\
\midrule
3.4.1 Modellazione Economica & & \\
- TCO model 47 parametri & Framework economico & \texttt{klems2023} \\
- Lift \& shift vs refactoring & Strategie migrazione & \texttt{armbrust2023} \\
- €8.2k-€87.3k per app & Costi migrazione & \texttt{idc2023cloud} \\
- 84.3\% probabilità successo & Monte Carlo results & \texttt{martens2024} \\
\midrule
3.4.2 Multi-Cloud Architecture & & \\
- 12 implementazioni analizzate & Case studies & \texttt{singh2023} \\
- IaaS/PaaS/SaaS segregation & Workload distribution & \texttt{vmware2024} \\
- CMP ROI 237\% & Governance benefits & \texttt{gartner2024cloud} \\
- Vendor lock-in mitigation & Risk management & \texttt{forrester2024} \\
\midrule

% SEZIONE 3.5
\textbf{3.5 Zero Trust Architecture} & & \\
\midrule
3.5.1 Riduzione Superficie Attacco & & \\
- ASSA model & Quantificazione rischio & \texttt{chen2024} \\
- 42.7\% riduzione totale & Risultati ZT & \texttt{rose2024} \\
- Micro-segmentation 31.2\% & Contributo componenti & \texttt{kindervag2023} \\
- Latenza <50ms nel 94\% casi & Performance impact & \texttt{socc2023} \\
\midrule
3.5.2 Policy Orchestration & & \\
- IAM/NAC/EDR/CASB integration & Componenti sicurezza & \texttt{microsoft2023} \\
- Policy-as-code & Automation approach & \texttt{morris2023} \\
- 76\% riduzione errori & Benefits automation & \texttt{forrester2023zero} \\
- MTTR 4.2h → 37min & Incident response & \texttt{gartner2024zerotrust} \\
\midrule

% SEZIONE 3.6
\textbf{3.6 Performance e Resilienza} & & \\
\midrule
3.6.1 Framework Maturità & & \\
- 28 KPI model & Metriche valutazione & \texttt{nist2024} \\
- 34 organizzazioni analizzate & Dataset empirico & \texttt{dataset2024} \\
- Distribuzione normale μ=42.3 & Risultati assessment & \texttt{survey2024} \\
- DevOps correlation & Fattori successo & \texttt{burns2023} \\
\midrule
3.6.2 Roadmap Ottimizzata & & \\
- 3-phase approach & Metodologia implementazione & \texttt{gartner2023retail} \\
- Quick wins €850k invest & Fase 1 economics & \texttt{usenix2024} \\
- Core transformation €4.7M & Fase 2 investment & \texttt{idc2024} \\
- 237\% ROI finale & Business case & \texttt{benchmark2023} \\
\midrule

% SEZIONE 3.7
\textbf{3.7 Conclusioni} & & \\
\midrule
3.7.1 Sintesi Evidenze & & \\
- H1: 99.95\% availability & Validazione ipotesi & \texttt{dataset2024} \\
- H2: -42.7\% ASSA & Risultati sicurezza & \texttt{chen2024} \\
- H3: 27.3\% compliance saving & Multi-cloud benefits & \texttt{singh2023} \\
- IC 95\% tutti i risultati & Robustezza statistica & \texttt{sigcomm2023} \\
\midrule
3.7.2 Limitazioni & & \\
- Dati aggregati vs diretti & Metodologia constraints & \texttt{rahman2024} \\
- Focus mercato EU & Geographic scope & \texttt{enisa2024} \\
- Modelli statici & Evolution limits & \texttt{newman2023} \\
\midrule
3.7.3 Bridge Capitolo 4 & & \\
- Compliance-by-design & Transizione tematica & \texttt{gdpr, nis2} \\
- Set-covering optimization & Preview metodologia & \texttt{icse2024} \\
\midrule
\end{longtable}

\section*{Template di Citazione per Paragrafi Chiave}

\subsection*{Paragrafo Introduttivo Sezione 3.1.1}
\begin{tcolorbox}[colback=gray!10]
\small
\texttt{L'analisi del threat landscape condotta nel Capitolo 2 ha evidenziato come il 78\% degli attacchi alla Grande Distribuzione Organizzata sfrutti vulnerabilità architetturali piuttosto che debolezze nei controlli di sicurezza\textbackslash cite\{enisa2024\}. Questo dato empirico, validato attraverso simulazione Monte Carlo\textbackslash cite\{martens2024\}, sottolinea la necessità di un'analisi sistematica dell'evoluzione infrastrutturale che integri teoria dei sistemi distribuiti\textbackslash cite\{coulouris2023,tanenbaum2023\}, economia dell'informazione\textbackslash cite\{klems2023\} e ingegneria della resilienza\textbackslash cite\{avizienis2023\}.}
\end{tcolorbox}

\subsection*{Paragrafo TCO Cloud Migration}
\begin{tcolorbox}[colback=gray!10]
\small
\texttt{L'analisi comparativa di tre strategie principali di migrazione\textbackslash cite\{armbrust2023\} rivela trade-off significativi. La strategia "lift and shift" presenta il minor costo iniziale (mediana €8.200 per applicazione) secondo i dati IDC\textbackslash cite\{idc2023cloud\}, mentre il "refactoring" completo, con costi mediani di €87.300 per applicazione\textbackslash cite\{usenix2024\}, genera i maggiori benefici a lungo termine con saving del 52-66\% come documentato da Klems et al.\textbackslash cite\{klems2023\}.}
\end{tcolorbox}

\subsection*{Paragrafo Zero Trust Impact}
\begin{tcolorbox}[colback=gray!10]
\small
\texttt{Il modello di quantificazione ASSA\textbackslash cite\{chen2024\} considera tre dimensioni principali: componenti esposti, privilegi assegnati, e connettività. L'implementazione progressiva di Zero Trust\textbackslash cite\{rose2024,kindervag2023\} riduce l'ASSA attraverso micro-segmentazione (contributo del 31.2\%), least privilege access (24.1\%), e continuous verification (18.4\%), come validato in produzione da Williams et al.\textbackslash cite\{socc2023\}.}
\end{tcolorbox}

\section*{Checklist Finale Citazioni}

\begin{tcolorbox}[colback=yellow!10!white,colframe=red!50!black,title=\textbf{Verifica Pre-Consegna}]
\begin{enumerate}
    \item[$\square$] Ogni dato numerico significativo ha una citazione
    \item[$\square$] Ogni affermazione tecnica è supportata da riferimenti
    \item[$\square$] I modelli teorici citano le fonti originali
    \item[$\square$] Le best practices riferiscono a standard o report autorevoli
    \item[$\square$] Non ci sono affermazioni non supportate su trend o statistiche
    \item[$\square$] Le citazioni multiple sono ordinate cronologicamente o per rilevanza
    \item[$\square$] Tutti i \texttt{\textbackslash cite\{\}} nel testo hanno corrispondenza nel .bib
    \item[$\square$] Non ci sono riferimenti orfani nella bibliografia
    \item[$\square$] Lo stile citazionale è uniforme in tutto il capitolo
    \item[$\square$] Le note a piè di pagina sono usate solo per chiarimenti, non per citazioni
\end{enumerate}
\end{tcolorbox}

\end{document}