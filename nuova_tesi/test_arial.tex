\documentclass{article}
\usepackage{fontspec}
\usepackage{lipsum}

% Test Arial su Windows 11
\setmainfont{Arial}[
    BoldFont = {Arial Bold},
    ItalicFont = {Arial Italic},
    BoldItalicFont = {Arial Bold Italic}
]

\begin{document}

\title{Test Font Arial - Windows 11}
\author{Test Compilazione XeLaTeX}
\date{\today}
\maketitle

\section{Test Stili Font}

\subsection{Testo Normale}
Questo è testo normale in Arial 12pt. Verifica che il font sia corretto.
àèéìòù ÀÈÉÌÒÙ - Caratteri accentati italiani.

\subsection{Grassetto}
\textbf{Questo è testo in grassetto (Arial Bold).}

\subsection{Corsivo}
\textit{Questo è testo in corsivo (Arial Italic).}

\subsection{Grassetto Corsivo}
\textbf{\textit{Questo è testo in grassetto corsivo (Arial Bold Italic).}}

\subsection{Maiuscoletto}
\textsc{Questo è Testo in Maiuscoletto per Autori}

\section{Test Dimensioni}

{\fontsize{10}{12}\selectfont Testo 10pt per note a piè di pagina.}

{\fontsize{11}{13}\selectfont\bfseries Testo 11pt grassetto per titoli sezioni.}

{\fontsize{12}{18}\selectfont Testo 12pt con interlinea 1.5 per corpo principale.}

\section{Test Caratteri Speciali}

Matematica: $\alpha + \beta = \gamma$

Simboli: © ® ™ € £ ¥ § ¶ † ‡

Virgolette: "inglesi" «francesi» „tedesche"

\end{document}