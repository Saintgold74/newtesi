% Figure per il Capitolo 1 - Codice XeLaTeX
% Da inserire nel documento principale o in file separati

% Preambolo necessario
% \usepackage{tikz}
% \usepackage{pgfplots}
% \pgfplotsset{compat=1.18}
% \usepackage{pgfplotstable}
% \usetikzlibrary{shapes,arrows,positioning,patterns,shadows,backgrounds}

% % Definizione colori tema
% \definecolor{primary}{HTML}{2E86AB}
% \definecolor{secondary}{HTML}{A23B72}
% \definecolor{success}{HTML}{73AB84}
% \definecolor{warning}{HTML}{F18F01}
% \definecolor{danger}{HTML}{C73E1D}
% \definecolor{neutral}{HTML}{7A7A7A}

% ===== FIGURA 1.1: FRAMEWORK GIST =====
\begin{figure}[htbp]
\centering
\begin{tikzpicture}[
    component/.style={
        rectangle, 
        rounded corners=10pt,
        draw,
        text width=3.5cm,
        minimum height=2.8cm,
        text centered,
        font=\small\sffamily,
        line width=2pt,
        drop shadow
    },
    centralnode/.style={
        circle,
        draw=secondary,
        fill=secondary!90,
        text width=2.8cm,
        minimum height=2.8cm,
        text centered,
        font=\footnotesize\bfseries\sffamily,
        line width=2.5pt,
        text=white,
        drop shadow
    },
    arrow/.style={
        ->,
        >=stealth,
        line width=2pt,
        color=gray!60
    },
    doublearrow/.style={
        <->,
        >=stealth,
        line width=1.5pt,
        color=gray!40,
        dashed
    },
    label/.style={
        font=\scriptsize\sffamily,
        fill=white,
        inner sep=2pt,
        rounded corners=3pt
    }
]

% Nodo centrale
\node[centralnode] (gist) at (0,0) {GIST\\Framework\\Integrato};

% Quattro componenti principali
\node[component, fill=primary!90, text=white, draw=primary] (governance) at (-4.5,3.5) {
    \textbf{Governance}\\[5pt]
    \footnotesize
    • Politiche\\
    • Processi\\
    • Risk Management\\
    • KPI e Metriche
};

\node[component, fill=success!90, text=white, draw=success] (infrastructure) at (4.5,3.5) {
    \textbf{Infrastructure}\\[5pt]
    \footnotesize
    • Fondamenta Fisiche\\
    • Reti SD-WAN\\
    • Cloud Ibrido\\
    • Edge Computing
};

\node[component, fill=danger!90, text=white, draw=danger] (security) at (-4.5,-3.5) {
    \textbf{Security}\\[5pt]
    \footnotesize
    • Zero Trust\\
    • Threat Detection\\
    • Incident Response\\
    • Data Protection
};

\node[component, fill=warning!90, text=white, draw=warning] (transformation) at (4.5,-3.5) {
    \textbf{Transformation}\\[5pt]
    \footnotesize
    • Change Management\\
    • Migration Path\\
    • Training\\
    • Innovation
};

% Connessioni con il centro
\draw[arrow] (governance) -- node[label,above,sloped] {Direttive} (gist);
\draw[arrow] (gist) -- node[label,above,sloped] {Requisiti} (infrastructure);
\draw[arrow] (security) -- node[label,below,sloped] {Controlli} (gist);
\draw[arrow] (gist) -- node[label,below,sloped] {Evoluzione} (transformation);

% Interconnessioni tra componenti
\draw[doublearrow] (governance) -- node[label,left] {Compliance} (security);
\draw[doublearrow] (infrastructure) -- node[label,right] {Resilienza} (transformation);
\draw[doublearrow] (governance.east) -- node[label,above] {Standards} (infrastructure.west);
\draw[doublearrow] (security.east) -- node[label,below] {Sicurezza} (transformation.west);

% Metriche esterne con sfondo
\begin{scope}[on background layer]
    \node[fill=gray!10, rounded corners=8pt, inner sep=10pt] at (0,-5.5) {
        \phantom{\textbf{Metriche Chiave:} Availability $\geq$99.95\% | TCO -38\% | ASSA -42\% | ROI 287\%}
    };
\end{scope}
\node[font=\footnotesize\sffamily\bfseries] at (0,-5.5) {
    Metriche Chiave: Availability $\geq$99.95\% | TCO -38\% | ASSA -42\% | ROI 287\%
};

\end{tikzpicture}
\caption{Il Framework GIST: Integrazione delle quattro dimensioni fondamentali per la trasformazione sicura della GDO. Il framework evidenzia le interconnessioni sistemiche tra governance strategica, infrastruttura tecnologica, sicurezza operativa e processi di trasformazione.}
\label{fig:gist_framework}
\end{figure}

% ===== FIGURA 1.2: EVOLUZIONE ATTACCHI CYBER =====
\begin{figure}[htbp]
\centering
\begin{tikzpicture}
\begin{axis}[
    width=0.9\textwidth,
    height=7cm,
    xlabel={Anno},
    ylabel={Numero di Incidenti},
    y2 label={Impatto Economico (M€)},
    xmin=2019.5, xmax=2025.5,
    ymin=0, ymax=900,
    y2min=0, y2max=250,
    xtick={2020,2021,2022,2023,2024,2025},
    legend pos=north west,
    ymajorgrids=true,
    grid style=dashed,
    grid alpha=0.3,
    axis y line*=left,
    axis x line*=bottom,
    y2 axis,
    y2 axis line style={blue!70!black},
    y2 tick label style={blue!70!black},
    y2 label style={blue!70!black},
    ylabel style={danger},
    y tick label style={danger},
]

% Area sotto la curva incidenti
\addplot[
    color=danger,
    fill=danger,
    fill opacity=0.2,
    draw=none,
    area legend,
    ]
    coordinates {
    (2020,0) (2020,142)
    (2021,187)
    (2022,312)
    (2023,584)
    (2024,721)
    (2025,847)
    (2025,0)
    } \closedcycle;

% Prima serie: Numero di incidenti
\addplot[
    color=danger,
    mark=*,
    line width=2pt,
    mark size=3pt,
    ]
    coordinates {
    (2020,142)
    (2021,187)
    (2022,312)
    (2023,584)
    (2024,721)
    (2025,847)
    };
    \addlegendentry{Numero Incidenti}

% Proiezione 2025 (tratteggiata)
\addplot[
    color=danger,
    dashed,
    line width=2pt,
    mark=none,
    forget plot,
    ]
    coordinates {
    (2024,721)
    (2025,847)
    };

% Seconda serie: Impatto economico (asse y2)
\addplot[
    color=primary,
    mark=square*,
    line width=2pt,
    mark size=3pt,
    y axis=right,
    ]
    coordinates {
    (2020,23)
    (2021,34)
    (2022,67)
    (2023,124)
    (2024,189)
    (2025,234)
    };
    \addlegendentry{Impatto (M€)}

% Proiezione 2025 impatto (tratteggiata)
\addplot[
    color=primary,
    dashed,
    line width=2pt,
    mark=none,
    y axis=right,
    forget plot,
    ]
    coordinates {
    (2024,189)
    (2025,234)
    };

% Annotazione per evidenziare il picco
\node[
    anchor=south west, 
    draw=danger, 
    fill=white, 
    rounded corners,
    line width=1.5pt,
    font=\footnotesize\bfseries
] at (axis cs:2022.3,400) {+312\%\\(2021-2023)};
\draw[->,danger,line width=2pt] (axis cs:2023,430) -- (axis cs:2023,560);

% Annotazione proiezione
\node[
    font=\scriptsize,
    color=danger
] at (axis cs:2025,870) {(Proiez.)};

\end{axis}
\end{tikzpicture}
\caption{Evoluzione degli attacchi cyber al settore retail italiano (2020-2025). L'incremento esponenziale del 312\% nel periodo 2021-2023 evidenzia l'urgenza di strategie di sicurezza avanzate. Dati 2025 proiettati. Fonte: aggregazione da CERT nazionali ed ENISA (2024).}
\label{fig:cyber_evolution}
\end{figure}

% ===== FIGURA 1.5: CONFRONTO TCO =====
\begin{figure}[htbp]
\centering
\begin{tikzpicture}
\begin{axis}[
    width=0.9\textwidth,
    height=7cm,
    xlabel={Anni dall'Implementazione},
    ylabel={TCO Cumulativo (Indice base=100)},
    xmin=-0.5, xmax=5.5,
    ymin=0, ymax=600,
    xtick={0,1,2,3,4,5},
    legend pos=north west,
    ymajorgrids=true,
    xmajorgrids=true,
    grid style={dashed, gray!30},
]

% Area di risparmio (dal punto 2 in poi)
\addplot[
    color=success,
    fill=success!30,
    draw=none,
    area legend,
    forget plot,
] coordinates {
    (2,265) (3,355) (4,450) (5,550)
    (5,390) (4,330) (3,275) (2,220)
} \closedcycle;

% TCO Tradizionale
\addplot[
    color=danger,
    mark=triangle*,
    line width=2pt,
    mark size=3pt,
    dashed,
    ]
    coordinates {
    (0,100) (1,180) (2,265) (3,355) (4,450) (5,550)
    };
    \addlegendentry{TCO On-Premise}

% TCO Cloud-Ibrido
\addplot[
    color=primary,
    mark=*,
    line width=2pt,
    mark size=3pt,
    ]
    coordinates {
    (0,120) (1,170) (2,220) (3,275) (4,330) (5,390)
    };
    \addlegendentry{TCO Cloud-Ibrido}

% Break-even point
\addplot[
    color=warning,
    mark=star,
    mark size=6pt,
    only marks,
    line width=2pt,
    ]
    coordinates {(1.31,198)};
    \addlegendentry{Break-even Point}
    
\node[
    anchor=south, 
    draw=warning, 
    fill=white, 
    rounded corners,
    line width=1.5pt,
    font=\footnotesize\bfseries
] at (axis cs:1.31,240) {Break-even\\15.7 mesi};
\draw[->,warning,line width=2pt] (axis cs:1.31,225) -- (axis cs:1.31,208);

% Annotazione risparmio
\node[
    anchor=west, 
    draw=success!80!black, 
    fill=white, 
    rounded corners,
    line width=1.5pt,
    font=\footnotesize\bfseries
] at (axis cs:3.5,400) {Risparmio\\38.2\%};

% Valori di risparmio annuali
\foreach \i in {2,3,4,5} {
    \pgfmathsetmacro{\tradval}{100 + 80*\i + 5*\i*\i}
    \pgfmathsetmacro{\cloudval}{120 + 50*\i}
    \pgfmathsetmacro{\saving}{\tradval - \cloudval}
    \node[
        font=\scriptsize\bfseries,
        color=success!80!black
    ] at (axis cs:\i,\cloudval-20) {+\pgfmathprintnumber{\saving}};
}

\end{axis}
\end{tikzpicture}
\caption{Analisi comparativa del Total Cost of Ownership (TCO) tra architetture tradizionali on-premise e soluzioni cloud-ibride su orizzonte quinquennale. Il break-even si raggiunge a 15.7 mesi con un risparmio cumulativo del 38.2\% al quinto anno.}
\label{fig:tco_comparison}
\end{figure}

% ===== FIGURA 1.7: PROGRESSIONE ROI =====
\begin{figure}[htbp]
\centering
\begin{tikzpicture}
\begin{axis}[
    ybar,
    width=0.9\textwidth,
    height=7cm,
    bar width=0.7cm,
    xlabel={Trimestri},
    ylabel={ROI Cumulativo (\%)},
    ymin=-60, ymax=320,
    xtick=data,
    symbolic x coords={Q1,Q2,Q3,Q4,Q5,Q6,Q7,Q8},
    ymajorgrids=true,
    grid style={dashed, gray!30},
    nodes near coords,
    nodes near coords style={font=\scriptsize\bfseries},
    every node near coord/.append style={
        /pgf/number format/fixed,
        /pgf/number format/precision=0
    },
]

% Barre ROI
\addplot[
    fill=danger!80,
    draw=black,
    line width=1pt,
] coordinates {
    (Q1,-15)
    (Q2,8)
};

\addplot[
    fill=success!80,
    draw=black,
    line width=1pt,
] coordinates {
    (Q3,34)
    (Q4,67)
    (Q5,112)
    (Q6,178)
    (Q7,234)
    (Q8,287)
};

% Linea zero
\draw[black, line width=1pt] (axis cs:Q1,0) -- (axis cs:Q8,0);

% Linea target
\draw[warning, dashed, line width=2pt] (axis cs:Q1,287) -- (axis cs:Q8,287);
\node[
    anchor=west,
    font=\footnotesize\bfseries,
    color=warning
] at (axis cs:Q7,287) {Target: 287\%};

% Break-even point
\node[
    circle,
    fill=danger,
    inner sep=3pt,
    label={[font=\scriptsize\bfseries]above:Break-even}
] at (axis cs:Q2,0) {};

\end{axis}
\end{tikzpicture}
\caption{Progressione del Return on Investment (ROI) nell'implementazione del framework GIST su 24 mesi. Il break-even viene raggiunto nel secondo trimestre con accelerazione progressiva fino al target del 287\%.}
\label{fig:roi_progression}
\end{figure}