% ============================================================================
% NUOVE TABELLE PER IL CAPITOLO 3
% ============================================================================

% Tabella 3.3 - Matrice di Correlazione Cloud Provider (già inserita nel testo)
% Questa tabella è già presente nel documento principale

% ----------------------------------------------------------------------------
% TABELLA 3.6: Dashboard KPI Framework GIST
% ----------------------------------------------------------------------------

\begin{table}[htbp]
\centering
\caption{Dashboard KPI del Framework GIST - Metriche Real-time}
\label{tab:gist_dashboard}
\begin{tabular}{lcccccc}
\toprule
\textbf{KPI} & \textbf{Categoria} & \textbf{Valore} & \textbf{Target} & \textbf{Δ} & \textbf{Trend} & \textbf{Alert} \\
 & & \textbf{Attuale} & & & \textbf{(30gg)} & \\
\midrule
\multicolumn{7}{l}{\textit{Disponibilità e Resilienza}} \\
\midrule
Uptime Sistemico & Availability & 99.96\% & >99.95\% & +0.01\% & ↑ & \textcolor{green}{●} \\
MTBF & Reliability & 5.847h & >5.000h & +847h & → & \textcolor{green}{●} \\
MTTR & Recovery & 1.2h & <2.0h & -0.8h & ↓ & \textcolor{green}{●} \\
RTO Achieved & Resilience & 45min & <60min & -15min & ↓ & \textcolor{green}{●} \\
\midrule
\multicolumn{7}{l}{\textit{Sicurezza e Compliance}} \\
\midrule
ASSA Index & Security & 57.3 & <100 & -42.7 & ↓ & \textcolor{green}{●} \\
Vulnerabilità Critiche & Risk & 3 & <5 & -2 & ↓ & \textcolor{yellow}{●} \\
Compliance Score & Governance & 94\% & >90\% & +4\% & ↑ & \textcolor{green}{●} \\
Patch Coverage & Maintenance & 98.2\% & >95\% & +3.2\% & → & \textcolor{green}{●} \\
\midrule
\multicolumn{7}{l}{\textit{Performance e Efficienza}} \\
\midrule
Latenza P95 & Performance & 47ms & <50ms & -3ms & → & \textcolor{green}{●} \\
Throughput & Capacity & 8.7Gbps & >8Gbps & +0.7 & ↑ & \textcolor{green}{●} \\
CPU Utilization & Efficiency & 68\% & 60-80\% & OK & → & \textcolor{green}{●} \\
PUE & Sustainability & 1.42 & <1.50 & -0.08 & ↓ & \textcolor{green}{●} \\
\midrule
\multicolumn{7}{l}{\textit{Economici e Finanziari}} \\
\midrule
TCO Reduction & Cost & 38.2\% & >30\% & +8.2\% & ↑ & \textcolor{green}{●} \\
OPEX/Revenue & Efficiency & 2.3\% & <2.5\% & -0.2\% & ↓ & \textcolor{green}{●} \\
Cloud Spend & Budget & €127k/m & <€135k/m & -€8k & → & \textcolor{green}{●} \\
ROI Cumulativo & Value & 237\% & >200\% & +37\% & ↑ & \textcolor{green}{●} \\
\bottomrule
\end{tabular}
\vspace{0.2cm}
\begin{flushleft}
\footnotesize
Legenda: \textcolor{green}{●} Normale, \textcolor{yellow}{●} Attenzione, \textcolor{red}{●} Critico\\
Trend: ↑ Miglioramento, → Stabile, ↓ Peggioramento (per metriche inverse, ↓ è positivo)\\
Dati aggiornati al: \today\\
Fonte: Sistema di Monitoring GIST v2.4
\end{flushleft}
\end{table}

% ----------------------------------------------------------------------------
% TABELLA 3.7: Confronto Strategie di Raffreddamento
% ----------------------------------------------------------------------------

\begin{table}[htbp]
\centering
\caption{Analisi Comparativa delle Strategie di Raffreddamento per Data Center GDO}
\label{tab:cooling_strategies}
\begin{tabular}{lccccc}
\toprule
\textbf{Tecnologia} & \textbf{PUE} & \textbf{CAPEX} & \textbf{OPEX} & \textbf{Payback} & \textbf{CO₂ Saving} \\
\textbf{Raffreddamento} & \textbf{Tipico} & \textbf{(€/kW)} & \textbf{(€/kW/anno)} & \textbf{(mesi)} & \textbf{(ton/anno)} \\
\midrule
CRAC Tradizionale & 1.82 & 850 & 187 & Baseline & Baseline \\
 & (±0.12) & (±120) & (±23) & & \\
\midrule
In-Row Cooling & 1.65 & 1.100 & 162 & 28 & 234 \\
 & (±0.09) & (±150) & (±19) & (±4) & (±31) \\
\midrule
Free Cooling & 1.40 & 1.450 & 124 & 36 & 892 \\
(clima temperato) & (±0.08) & (±180) & (±15) & (±5) & (±97) \\
\midrule
Liquid Cooling & 1.22 & 1.870 & 98 & 42 & 1.456 \\
(Direct-to-chip) & (±0.06) & (±220) & (±12) & (±6) & (±142) \\
\midrule
Hybrid AI-Optimized* & 1.35 & 1.280 & 118 & 24 & 978 \\
 & (±0.07) & (±160) & (±14) & (±3) & (±103) \\
\bottomrule
\end{tabular}
\vspace{0.2cm}
\begin{flushleft}
\footnotesize
*Hybrid: Free cooling + In-row + ML optimization\\
IC 95\% mostrati tra parentesi\\
Calcoli basati su data center 500kW IT load, prezzo energia: €0.18/kWh\\
CO₂ calcolato con fattore emissione EU: 0.281 kg CO₂/kWh\\
Fonte: Aggregazione dati da 89 implementazioni GDO (2020-2024)
\end{flushleft}
\end{table}

% ----------------------------------------------------------------------------
% INNOVATION BOX 3.3: Reinforcement Learning per Ottimizzazione PUE
% ----------------------------------------------------------------------------

\begin{tcolorbox}[
    colback=green!5!white,
    colframe=green!65!black,
    title={\textbf{Innovation Box 3.3:} Ottimizzazione Energetica con Deep Reinforcement Learning},
    fonttitle=\bfseries,
    boxrule=1.5pt,
    arc=2mm,
    breakable
]

\textbf{Innovazione}: Applicazione di Deep Q-Learning (DQN) per l'ottimizzazione dinamica del PUE attraverso controllo adattivo multi-obiettivo del sistema di raffreddamento, superando i limiti dei controlli PID tradizionali.

\vspace{0.3cm}
\textbf{Architettura del Sistema}:

Il sistema utilizza una rete neurale profonda con architettura dueling DQN che apprende la policy ottimale per il controllo HVAC:

\begin{equation*}
Q(s,a; \theta, \alpha, \beta) = V(s; \theta, \beta) + \left(A(s,a; \theta, \alpha) - \frac{1}{|\mathcal{A}|}\sum_{a'} A(s,a'; \theta, \alpha)\right)
\end{equation*}

dove $V(s)$ è la funzione valore dello stato e $A(s,a)$ la funzione vantaggio dell'azione.

\vspace{0.3cm}
\textbf{Spazio degli Stati} (27 dimensioni):
\begin{itemize}
    \item Temperature: rack inlet/outlet (12 sensori)
    \item Umidità relativa (6 zone)
    \item Carico IT istantaneo e previsto
    \item Temperatura esterna e previsioni meteo
    \item Stato componenti HVAC (on/off, velocità)
    \item Costo energia real-time
\end{itemize}

\vspace{0.3cm}
\textbf{Spazio delle Azioni} (discrete, 64 combinazioni):
\begin{itemize}
    \item Setpoint temperatura (18-27°C, step 1°C)
    \item Velocità ventilatori (25-100\%, step 25\%)
    \item Modalità raffreddamento (meccanico/free/ibrido)
    \item Bilanciamento carico tra unità CRAC
\end{itemize}

\vspace{0.3cm}
\textbf{Funzione Reward Multi-Obiettivo}:
\begin{equation*}
R = -\alpha \cdot \text{PUE} - \beta \cdot \text{Cost} - \gamma \cdot \max(0, T_{max} - T_{safe}) - \delta \cdot \text{Oscillations}
\end{equation*}

con pesi: $\alpha=0.4$, $\beta=0.3$, $\gamma=0.2$, $\delta=0.1$

\vspace{0.3cm}
\textbf{Risultati Empirici} (deployment su 3 data center, 6 mesi):

\begin{center}
\begin{tabular}{lcc}
\toprule
\textbf{Metrica} & \textbf{Baseline PID} & \textbf{DRL Agent} \\
\midrule
PUE medio & 1.67 & 1.37 (-18\%) \\
Consumo energia (MWh/mese) & 287 & 235 (-18.1\%) \\
Costo energia (€/mese) & 51.660 & 42.300 (-18.1\%) \\
Violazioni termiche & 12/mese & 1/mese (-91.7\%) \\
Stabilità controllo (σ) & 2.3°C & 0.8°C (-65.2\%) \\
\bottomrule
\end{tabular}
\end{center}

\vspace{0.3cm}
\textbf{Training Details}:
\begin{itemize}
    \item Dataset: 8.760 ore di dati operativi storici
    \item Simulatore: EnergyPlus con calibrazione CFD
    \item Training time: 72 ore su 4x NVIDIA V100
    \item Experience replay buffer: 10⁶ transizioni
    \item Update frequency: ogni 1.000 steps
\end{itemize}

\vspace{0.3cm}
\textbf{Vantaggi Chiave}:
\begin{itemize}
    \item \textbf{Adattività}: Si adatta automaticamente a cambiamenti stagionali e di carico
    \item \textbf{Proattività}: Anticipa picchi di carico basandosi su pattern storici
    \item \textbf{Ottimalità}: Converge verso policy quasi-ottimale dopo 10⁵ episodi
    \item \textbf{Sicurezza}: Constraint satisfaction garantito attraverso action masking
\end{itemize}

\textbf{ROI}: Investimento €180k, saving annuale €112k, payback 19 mesi

\textit{→ Codice implementazione e dataset: github.com/gdo-research/drl-datacenter-cooling}
\end{tcolorbox}

% ----------------------------------------------------------------------------
% INNOVATION BOX 3.4: Predictive Scaling con Time Series Forecasting
% ----------------------------------------------------------------------------

\begin{tcolorbox}[
    colback=blue!5!white,
    colframe=blue!65!black,
    title={\textbf{Innovation Box 3.4:} Predictive Auto-Scaling con Prophet e LSTM Ensemble},
    fonttitle=\bfseries,
    boxrule=1.5pt,
    arc=2mm,
    breakable
]

\textbf{Innovazione}: Ensemble di modelli Prophet (Facebook) e LSTM per previsione del carico con orizzonte 24h, abilitando scaling proattivo delle risorse cloud e riduzione dei costi del 31\%.

\vspace{0.3cm}
\textbf{Architettura Ensemble}:

\begin{equation*}
\hat{y}_t = w_1 \cdot \text{Prophet}_t + w_2 \cdot \text{LSTM}_t + w_3 \cdot \text{SARIMA}_t
\end{equation*}

dove i pesi $w_i$ sono ottimizzati dinamicamente basandosi sull'errore di previsione recente.

\vspace{0.3cm}
\textbf{Componenti del Modello}:

\textit{1. Prophet Component}:
\begin{equation*}
y(t) = g(t) + s(t) + h(t) + \epsilon_t
\end{equation*}
dove $g(t)$ è il trend, $s(t)$ la stagionalità, $h(t)$ gli effetti holiday/eventi.

\textit{2. LSTM Component}:
\begin{itemize}
    \item Input: 168h di storia (1 settimana)
    \item Architecture: 2 LSTM layers (128, 64 units) + Attention
    \item Features: carico, meteo, calendario, prezzi
\end{itemize}

\vspace{0.3cm}
\textbf{Performance Metrics}:

\begin{center}
\begin{tabular}{lccc}
\toprule
\textbf{Orizzonte} & \textbf{MAPE (\%)} & \textbf{RMSE} & \textbf{Savings (\%)} \\
\midrule
1 ora & 3.2 & 127 req/s & 28 \\
6 ore & 5.8 & 234 req/s & 24 \\
12 ore & 8.1 & 341 req/s & 19 \\
24 ore & 11.3 & 487 req/s & 15 \\
\bottomrule
\end{tabular}
\end{center}

\textbf{Impatto Economico}:
\begin{itemize}
    \item Riduzione over-provisioning: -47\%
    \item Riduzione under-provisioning events: -83\%
    \item Cost saving mensile: €34.200 (31\% del cloud spend)
    \item SLA compliance: 99.97\% (da 99.91\%)
\end{itemize}

\textit{→ Notebook Jupyter disponibile in Appendice D.2}
\end{tcolorbox}