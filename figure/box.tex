\documentclass{article}

% --- PACCHETTI NECESSARI ---
\usepackage{fontspec} % Indispensabile per XeLaTeX
\usepackage{amsmath}  % Per il comando \text in modalità matematica
\usepackage{xcolor}   % Per i colori e \colorbox
\usepackage{caption}  % Per le didascalie delle figure
\usepackage{tikz}
\usetikzlibrary{matrix, positioning} % Librerie TikZ per matrici e posizionamento

\begin{document}

\begin{figure}[h!] % h! incoraggia LaTeX a posizionare la figura qui
\centering
\begin{tikzpicture}[
    % --- DEFINIZIONE DEGLI STILI ---
    test/.style={
        rectangle, 
        draw=black, 
        rounded corners, 
        minimum width=3.8cm, % Leggermente aumentato per una migliore leggibilità
        minimum height=1.0cm,
        font=\footnotesize,
        align=center % Fondamentale per centrare il testo su più righe
    },
    pass/.style={test, fill=green!20, draw=green!60!black, thick},
    fail/.style={test, fill=red!20, draw=red!60!black, thick},
    partial/.style={test, fill=yellow!20, draw=orange!60!black, thick},
]

% --- GRIGLIA DEI TEST ---
% La griglia è stata creata con 'matrix of nodes' per una sintassi più pulita e robusta.
% È stata riorganizzata su 5 righe per avere un numero di colonne consistente.
\matrix (grid) [
    matrix of nodes, 
    row sep=0.5cm, 
    column sep=0.5cm,
    nodes={test} % Applica lo stile 'test' di base a tutti i nodi
] 
{
 % Riga 1
    |[pass]|    {$\text{Benford's Law}$\\$p = 0.127$} &
    |[pass]|    {$\text{Poisson Events}$\\$p = 0.234$} &
    |[pass]|    {$\text{Correlazione}$\\$r = 0.62$} \\
    % Riga 2
    |[pass]|    {$\text{Weekend Effect}$\\$ratio = 1.28$} &
    |[pass]|    {$\text{Stagionalità}$\\$F = 8.34$} &
    |[pass]|    {$\text{Autocorrelazione}$\\$ACF = 0.41$} \\
    % Riga 3
    |[pass]|    {$\text{Non-uniformità}$\\$\chi^2 = 847.3$} &
    |[pass]|    {$\text{Completezza}$\\$missing = 0\%$} & % CORRETTO: Rimossa parentesi graffa in più
    |[partial]| {$\text{CVSS Distribution}$\\$\Delta = 0.08$} \\
    % Riga 4
    |[pass]|    {$\text{ID Unicità}$\\$dupl. = 0$} &
    |[partial]| {$\text{Latenza Rete}$\\$95\% < 50ms$} &
    |[pass]|    {$\text{Payment Mix}$\\$\chi^2 = 3.21$} \\
    % Riga 5 (con celle vuote per mantenere l'allineamento)
    |[pass]|    {$\text{Geolocalizzazione}$\\$dist. = 0.08$} &
    {} & % Cella vuota
    {} % Cella vuota
    \\
};

% --- TITOLO ---
% Posizionato in modo relativo sopra la griglia per un layout responsive
\node[font=\large\bfseries, above=0.7cm of grid.north] 
    {Dashboard Validazione Statistica Digital Twin};

% --- BOX DI RIEPILOGO ---
% Posizionato in modo relativo sotto la griglia per centrarlo automaticamente
\node[
    draw=black, thick, fill=blue!10, rounded corners,
    minimum width=9cm, minimum height=1.5cm,
    below=1cm of grid.south, font=\footnotesize
] (summary) {
    \begin{tabular}{lc|lc}
        \textbf{Test Superati:} & 16/18 & \textbf{Tasso Successo:} & 88.9\% \\
        \textbf{Conformità Statistica:} & Alta & \textbf{Validità:} & Confermata
    \end{tabular}
};

% --- LEGENDA ---
% Posizionata in modo relativo sotto il riepilogo
\node[below=0.5cm of summary, font=\scriptsize] {
    \colorbox{green!20}{PASS} = Test superato \quad
    \colorbox{yellow!20}{PARTIAL} = Parzialmente superato \quad
    \colorbox{red!20}{FAIL} = Test fallito
};

\end{tikzpicture}
\caption{Dashboard riassuntivo validazione: 88.9\% dei test statistici superati 
conferma la validità del framework Digital Twin per la generazione di dati 
sintetici realistici.}
\label{fig:validation-dashboard}
\end{figure}

\end{document}