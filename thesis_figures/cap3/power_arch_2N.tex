% File: figures/power_configurations_simple.tex
% Confronto semplificato tra configurazioni N+1 e 2N
% Per uso con \input{} nel documento principale

\begin{tikzpicture}[
    % Stili base
    ups/.style={rectangle, draw=blue!70, fill=blue!15, very thick, minimum width=1.8cm, minimum height=2.5cm, rounded corners=5pt},
    load/.style={rectangle, draw=green!70, fill=green!15, thick, minimum width=3cm, minimum height=1.5cm, rounded corners=5pt},
    fail/.style={rectangle, draw=red!70, fill=red!15, opacity=0.5, very thick, minimum width=1.8cm, minimum height=2.5cm, rounded corners=5pt},
    grid/.style={rectangle, draw=black, thick, minimum width=1.5cm, minimum height=1cm},
    battery/.style={rectangle, draw=orange!70, fill=orange!15, thick, minimum width=1.2cm, minimum height=0.6cm},
    % Frecce
    powerok/.style={->, very thick, draw=green!60},
    powerfail/.style={->, very thick, draw=red!60, dashed},
    % Testo
    title/.style={font=\large\bfseries},
    config/.style={font=\normalsize\bfseries, fill=white, text=black, rounded corners=3pt},
    label/.style={font=\small},
    value/.style={font=\footnotesize\ttfamily}
]

% === TITOLO PRINCIPALE ===
\node[title] at (0,7) {Confronto Configurazioni di Ridondanza Sistemi UPS};

% === CONFIGURAZIONE N+1 (SINISTRA) ===
\begin{scope}[shift={(-5,0)}]
    % Titolo configurazione
    \node[config, fill=blue!30] at (0,5) {Configurazione N+1};
    
    % Rete elettrica
    \node[grid] (grid_n1) at (0,3) {RETE};
    
    % UPS
    \node[ups] (ups1) at (-2,0) {UPS 1 \newline 100kW};
    \node[ups] (ups2) at (0,0) {UPS 2 \newline 100kW};
    \node[ups] (ups3) at (2,0) {UPS 3 \newline 100kW};
    \node[label, text=blue!70] at (0,-1.8) {3 attivi per 200kW carico};
    
    % Batterie (semplificate)
    \node[battery] at (-2,-1) {Batt};
    \node[battery] at (0,-1) {Batt};
    \node[battery] at (2,-1) {Batt};
    
    % Carico
    \node[load] (load_n1) at (0,-3.5) {CARICO IT \newline 200kW};
    
    % Connessioni
    \draw[powerok] (grid_n1) -- (ups1);
    \draw[powerok] (grid_n1) -- (ups2);
    \draw[powerok] (grid_n1) -- (ups3);
    \draw[powerok] (ups1) -- (load_n1);
    \draw[powerok] (ups2) -- (load_n1);
    \draw[powerok] (ups3) -- (load_n1);
    
    % Box informativo
    \node[draw=gray!50, thick, rounded corners, text width=4cm] at (0,-5.5) {
        \centering\footnotesize
        \textbf{Disponibilità: 99.82\%}\\[2pt]
        MTBF: 52.560 ore\\
        Costo: 100 (base)\\
        \textcolor{green!60}{✓ Economico}\\
        \textcolor{red!60}{✗ Manutenzione difficile}
    };
\end{scope}

% === SEPARATORE ===
\draw[gray!40, very thick, dashed] (0,5.5) -- (0,-6.5);

% === CONFIGURAZIONE 2N (DESTRA) ===
\begin{scope}[shift={(5,0)}]
    % Titolo configurazione
    \node[config, fill=green!30] at (0,5) {Configurazione 2N};
    
    % Due reti separate
    \node[grid] (grid_a) at (-1.5,3) {RETE A};
    \node[grid] (grid_b) at (1.5,3) {RETE B};
    
    % Sistema A
    \node[ups] (ups_a) at (-1.5,0) {UPS A \newline 200kW};
    \node[battery] at (-1.5,-1) {Batt A};
    
    % Sistema B
    \node[ups] (ups_b) at (1.5,0) {UPS B \newline 200kW};
    \node[battery] at (1.5,-1) {Batt B};
    
    \node[label, text=green!70] at (0,-1.8) {Ogni sistema gestisce 100\% carico};
    
    % Carico con doppia alimentazione
    \node[load] (load_2n) at (0,-3.5) {CARICO IT \newline 200kW \newline (2x PSU)};
    
    % Connessioni
    \draw[powerok] (grid_a) -- (ups_a);
    \draw[powerok] (grid_b) -- (ups_b);
    \draw[powerok] (ups_a) -- (load_2n.north west);
    \draw[powerok] (ups_b) -- (load_2n.north east);
    
    % Box informativo
    \node[draw=gray!50, thick, rounded corners, text width=4cm] at (0,-5.5) {
        \centering\footnotesize
        \textbf{Disponibilità: 99.94\%}\\[2pt]
        MTBF: 175.200 ore\\
        Costo: 143 (+43\%)\\
        \textcolor{green!60}{✓ Manutenzione online}\\
        \textcolor{green!60}{✓ Zero downtime}
    };
\end{scope}

% === SCENARIO DI GUASTO (PARTE INFERIORE) ===
\node[title, font=\normalsize\bfseries] at (0,-7.5) {Comportamento in Caso di Guasto};

% N+1 con guasto
\begin{scope}[shift={(-5,-10)}]
    \node[label, text=red!70] at (0,1) {N+1: Guasto UPS};
    
    % UPS con uno guasto
    \node[fail] (ups1f) at (-2,0) {UPS 1\\GUASTO};
    \node[ups] (ups2f) at (0,0) {UPS 2\\100kW};
    \node[ups] (ups3f) at (2,0) {UPS 3\\100kW};
    
    % Carico
    \node[load] (load_n1f) at (0,-2) {CARICO\\200kW};
    
    % Connessioni
    \draw[powerfail] (ups1f) -- (load_n1f);
    \draw[powerok] (ups2f) -- (load_n1f);
    \draw[powerok] (ups3f) -- (load_n1f);
    
    \node[label, text=orange!70] at (0,-3.2) {⚠ Nessuna ridondanza residua};
\end{scope}

% 2N con guasto
\begin{scope}[shift={(5,-10)}]
    \node[label, text=red!70] at (0,1) {2N: Guasto Sistema A};
    
    % Sistema A guasto
    \node[fail] (ups_af) at (-1.5,0) {UPS A\\GUASTO};
    
    % Sistema B operativo
    \node[ups] (ups_bf) at (1.5,0) {UPS B\\200kW};
    
    % Carico
    \node[load] (load_2nf) at (0,-2) {CARICO\\200kW};
    
    % Connessioni
    \draw[powerfail] (ups_af) -- (load_2nf);
    \draw[powerok] (ups_bf) -- (load_2nf);
    
    \node[label, text=green!70] at (0,-3.2) {✓ Sistema B gestisce 100\% carico};
\end{scope}

% === LEGENDA CENTRALE ===
\node[draw=gray!40, thick, rounded corners] at (0,-13.5) {
    \footnotesize
    \textbf{Vantaggi chiave 2N:} \quad
    +0.12\% disponibilità \quad
    3.3× MTBF \quad
    Manutenzione senza downtime \quad
    ROI 28 mesi
};

\end{tikzpicture}