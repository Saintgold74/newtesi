% Proposta di figura aggiuntiva per il Capitolo 1
% Da inserire dopo la sezione 1.2.2 (L'emergere di nuove minacce)

\begin{figure}[htbp]
\centering
\begin{tikzpicture}[
    node distance=2cm,
    box/.style={rectangle, rounded corners, draw=black!70, fill=blue!10, 
                minimum width=3cm, minimum height=1cm, text centered},
    threat/.style={rectangle, rounded corners, draw=red!70, fill=red!10,
                   minimum width=2.5cm, minimum height=0.8cm, text centered},
    arrow/.style={->, >=stealth, thick},
    doublearrow/.style={<->, >=stealth, thick}
]

% Nodo centrale
\node[box, fill=green!20, minimum width=4cm, minimum height=1.5cm] (gdo) {\textbf{Sistema GDO}\\Punto Vendita};

% Livelli di minacce (cerchi concentrici)
\node[threat, above=2.5cm of gdo] (cyber) {Minacce\\Informatiche};
\node[threat, below=2.5cm of gdo] (fisiche) {Minacce\\Fisiche};
\node[threat, left=3cm of gdo] (supply) {Minacce\\Supply Chain};
\node[threat, right=3cm of gdo] (insider) {Minacce\\Interne};

% Collegamenti
\draw[arrow, red!60] (cyber) -- node[right, font=\small] {Ransomware\\DDoS} (gdo);
\draw[arrow, red!60] (fisiche) -- node[right, font=\small] {Sabotaggi\\Furti} (gdo);
\draw[arrow, red!60] (supply) -- node[above, font=\small] {Fornitori\\compromessi} (gdo);
\draw[arrow, red!60] (insider) -- node[above, font=\small] {Errori\\Frodi} (gdo);

% Impatti
\node[box, fill=orange!20, below right=1.5cm and 2cm of gdo, minimum width=3.5cm] (impatti) {
    \textbf{Impatti}\\
    \small
    • Perdite economiche\\
    • Danni reputazionali\\
    • Interruzione servizio\\
    • Sanzioni normative
};

\draw[arrow, orange!80, dashed] (gdo) -- (impatti);

% Legenda temporale
\node[below=4cm of gdo, text width=8cm, align=center] {
    \textbf{Evoluzione temporale:}\\
    \small 2019-2021: Focus su minacce cyber tradizionali\\
    2022-2023: Aumento attacchi fisico-digitali integrati\\
    2024-2025: Minacce sistemiche alla supply chain
};

\end{tikzpicture}
\caption{Panorama integrato delle minacce alla GDO: convergenza tra domini fisici e digitali}
\label{fig:panorama_minacce_integrato}
\end{figure}

% ============================================
% Proposta di tabella aggiuntiva per la sezione 1.4
% Confronto tra approcci tradizionali e framework GIST
% ============================================

\begin{table}[htbp]
\centering
\caption{Confronto tra approccio tradizionale e framework GIST}
\label{tab:confronto_approcci}
\begin{tabular}{|p{3.5cm}|p{5cm}|p{5cm}|}
\hline
\textbf{Dimensione} & \textbf{Approccio Tradizionale} & \textbf{Framework GIST} \\
\hline
\hline
\textbf{Architettura} & 
Sistemi monolitici centralizzati, singolo punto di controllo & 
Architettura distribuita modulare, resilienza intrinseca \\
\hline
\textbf{Sicurezza} & 
Perimetrale, focus su firewall e antivirus & 
Zero-trust, sicurezza multilivello integrata \\
\hline
\textbf{Conformità} & 
Gestione separata per normativa, ridondanza controlli & 
Matrice unificata, 156 controlli integrati \\
\hline
\textbf{Gestione rischi} & 
Valutazioni periodiche statiche (annuali/semestrali) & 
Monitoraggio continuo con ML, previsione proattiva \\
\hline
\textbf{Costi IT} & 
CAPEX elevato, ROI a 5-7 anni & 
Modello OPEX, ROI in 18-24 mesi \\
\hline
\textbf{Agilità} & 
Modifiche richiedono 6-12 mesi & 
Deployment continuo, modifiche in giorni \\
\hline
\textbf{Competenze} & 
Team IT generalista interno & 
Modello ibrido con competenze specializzate \\
\hline
\textbf{Scalabilità} & 
Verticale, costosa e complessa & 
Orizzontale elastica, pay-per-use \\
\hline
\end{tabular}
\end{table}

% ============================================
% Schema riassuntivo del Framework GIST
% Da inserire nella sezione 1.3.1
% ============================================

\begin{figure}[htbp]
\centering
\begin{tikzpicture}[
    component/.style={rectangle, rounded corners=10pt, draw=blue!60, 
                     fill=blue!20, minimum width=3.5cm, minimum height=2cm,
                     text centered, font=\footnotesize},
    core/.style={circle, draw=green!60, fill=green!20, minimum size=2.5cm,
                 text centered, font=\small\bfseries},
    tool/.style={rectangle, rounded corners=5pt, draw=gray!60, 
                fill=gray!10, minimum width=2.5cm, minimum height=0.8cm,
                text centered, font=\scriptsize},
    flow/.style={->, >=stealth, thick, blue!60},
    biflow/.style={<->, >=stealth, thick, green!60}
]

% Core centrale
\node[core] (gist) at (0,0) {Framework\\GIST};

% Quattro componenti principali
\node[component] (security) at (-4,3) {\textbf{Sicurezza}\\Algoritmo ASSA\\Scoring rischio\\ML predittivo};
\node[component] (arch) at (4,3) {\textbf{Architettura}\\Edge-Cloud\\Multi-Cloud\\Compliance-native};
\node[component] (gov) at (-4,-3) {\textbf{Governance}\\Matrice MIN\\156 controlli\\Audit continuo};
\node[component] (ops) at (4,-3) {\textbf{Operazioni}\\Digital Twin\\Simulazione\\KPI real-time};

% Collegamenti
\draw[biflow] (gist) -- (security);
\draw[biflow] (gist) -- (arch);
\draw[biflow] (gist) -- (gov);
\draw[biflow] (gist) -- (ops);

% Interconnessioni tra componenti
\draw[flow, dashed, orange!60] (security) -- node[above, font=\scriptsize] {Requisiti} (arch);
\draw[flow, dashed, orange!60] (arch) -- node[right, font=\scriptsize] {Metriche} (ops);
\draw[flow, dashed, orange!60] (ops) -- node[below, font=\scriptsize] {Conformità} (gov);
\draw[flow, dashed, orange!60] (gov) -- node[left, font=\scriptsize] {Policy} (security);

% Strumenti di supporto
\node[tool] at (-6,1) {Calcolatore GIST};
\node[tool] at (6,1) {GDO-Bench Dataset};
\node[tool] at (-6,-1) {Dashboard Monitor};
\node[tool] at (6,-1) {Risk Scorer XGB};

% Output
\node[below=4.5cm of gist, text width=10cm, align=center] {
    \colorbox{yellow!20}{
    \parbox{9.5cm}{
        \centering
        \textbf{Output del Framework:}\\
        \small
        • Riduzione superficie attacco: -45\%\\
        • Disponibilità sistema: 99,96\%\\
        • Tempo conformità: -60\%\\
        • ROI: 18-24 mesi
    }
    }
};

\end{tikzpicture}
\caption{Architettura concettuale del Framework GIST con componenti principali e strumenti di supporto}
\label{fig:gist_framework_overview}
\end{figure}
