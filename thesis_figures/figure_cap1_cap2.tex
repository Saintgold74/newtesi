% ===================================================================
% FIGURE PER CAPITOLO 1 E CAPITOLO 2 - TESI FRAMEWORK GIST
% ===================================================================
% Autore: Framework per Tesi di Laurea in Ingegneria Informatica
% Università: Cusano
% Data: 2025
% ===================================================================

\documentclass[11pt,a4paper]{article}
\usepackage[utf8]{inputenc}
\usepackage[italian]{babel}
\usepackage{tikz}
\usepackage{pgfplots}
\usepackage{pgfplotstable}
\usepackage{graphicx}
\usepackage{amsmath}
\usepackage{float}

% Configurazione pgfplots
\pgfplotsset{compat=1.18}

% Definizione colori personalizzati per il tema della tesi
\definecolor{gdoblue}{RGB}{0, 102, 204}
\definecolor{gdogreen}{RGB}{76, 175, 80}
\definecolor{gdored}{RGB}{244, 67, 54}
\definecolor{gdoorange}{RGB}{255, 152, 0}
\definecolor{gdopurple}{RGB}{156, 39, 176}
\definecolor{gdogray}{RGB}{158, 158, 158}
\definecolor{gdolightblue}{RGB}{179, 229, 252}
\definecolor{gdodarkblue}{RGB}{25, 32, 102}

% Librerie TikZ necessarie
\usetikzlibrary{arrows.meta,positioning,calc,patterns,decorations.pathreplacing,shadows,shapes.geometric,shapes.symbols,backgrounds,fit}

\begin{document}

% ===================================================================
% CAPITOLO 1 - FIGURA 1: EVOLUZIONE DEGLI ATTACCHI
% ===================================================================
\section*{Figura 1.1: Evoluzione della composizione percentuale delle tipologie di attacco nel settore GDO}

\begin{figure}[H]
\centering
\begin{tikzpicture}
\begin{axis}[
    width=14cm,
    height=8cm,
    area style,
    xlabel={Anno},
    ylabel={Percentuale (\%)},
    xmin=2019, xmax=2026,
    ymin=0, ymax=100,
    xtick={2019,2020,2021,2022,2023,2024,2025,2026},
    ytick={0,20,40,60,80,100},
    legend style={
        at={(1.02,0.5)},
        anchor=west,
        font=\small,
        draw=none,
        fill=white,
        fill opacity=0.8
    },
    grid=major,
    grid style={dashed,gray!30},
    axis lines=left,
    axis line style={->,thick},
    tick label style={font=\small},
    label style={font=\normalsize}
]

% Dati per il grafico ad area impilata
\addplot+[
    fill=gdoblue!70,
    draw=gdoblue,
    thick,
    area legend,
    forget plot
] coordinates {
    (2019,65) (2020,58) (2021,52) (2022,45) (2023,38) (2024,32) (2025,28) (2026,25)
} \closedcycle;

\addplot+[
    fill=gdored!70,
    draw=gdored,
    thick,
    area legend,
    forget plot
] coordinates {
    (2019,80) (2020,75) (2021,72) (2022,70) (2023,68) (2024,67) (2025,68) (2026,70)
} \closedcycle;

\addplot+[
    fill=gdogreen!70,
    draw=gdogreen,
    thick,
    area legend,
    forget plot
] coordinates {
    (2019,100) (2020,100) (2021,100) (2022,100) (2023,100) (2024,100) (2025,100) (2026,100)
} \closedcycle;

% Linee di tendenza tratteggiate per le proiezioni
\draw[dashed,thick,gdodarkblue] (axis cs:2024,0) -- (axis cs:2024,100);
\node[anchor=north,font=\footnotesize] at (axis cs:2024,0) {Dati storici};
\node[anchor=north,font=\footnotesize] at (axis cs:2025.5,0) {Proiezioni};

% Aggiungiamo le etichette delle aree
\node[font=\small,white] at (axis cs:2021,26) {Furto Dati};
\node[font=\small,white] at (axis cs:2021,62) {Disruzione};
\node[font=\small,white] at (axis cs:2021,86) {Cyber-Fisici};

\legend{Attacchi tradizionali (Furto dati),Disruzione operativa,Compromissione informatico-fisica};

\end{axis}
\end{tikzpicture}
\caption{Evoluzione della composizione percentuale delle tipologie di attacco nel settore GDO (2019-2026)}
\end{figure}

% ===================================================================
% CAPITOLO 1 - FIGURA 2: STRUTTURA DELLA TESI
% ===================================================================
\section*{Figura 1.2: Struttura della tesi e flusso logico dell'argomentazione}

\begin{figure}[H]
\centering
\begin{tikzpicture}[
    node distance=2cm,
    chapter/.style={
        rectangle,
        draw=gdoblue,
        fill=gdolightblue!30,
        text width=3.5cm,
        minimum height=2cm,
        align=center,
        font=\small\bfseries,
        drop shadow
    },
    framework/.style={
        ellipse,
        draw=gdogreen,
        fill=gdogreen!20,
        text width=3cm,
        minimum height=1.8cm,
        align=center,
        font=\small,
        drop shadow
    },
    arrow/.style={
        ->,
        thick,
        gdoblue,
        >=Stealth
    },
    feedback/.style={
        <->,
        dashed,
        gdogray,
        >=Stealth
    }
]

% Nodi dei capitoli
\node[chapter] (cap1) {Capitolo 1\\Introduzione\\e Contesto};
\node[chapter, right=of cap1] (cap2) {Capitolo 2\\Minacce e\\ASSA-GDO};
\node[chapter, right=of cap2] (cap3) {Capitolo 3\\Architetture\\GRAF};
\node[chapter, below=of cap2] (cap4) {Capitolo 4\\Conformità\\MIN};
\node[chapter, right=of cap4] (cap5) {Capitolo 5\\Framework\\GIST};

% Nodi dei framework
\node[framework, below=1cm of cap2] (assa) {ASSA-GDO\\28\%};
\node[framework, below=1cm of cap3] (graf) {GRAF\\35\%};
\node[framework, below=1cm of cap4] (min) {MIN\\37\%};

% Framework centrale GIST
\node[
    rectangle,
    draw=gdored,
    fill=gdored!10,
    text width=4cm,
    minimum height=2.5cm,
    align=center,
    font=\large\bfseries,
    drop shadow,
    below=3.5cm of cap3
] (gist) {Framework\\GIST\\Score: 0-100};

% Frecce principali
\draw[arrow] (cap1) -- (cap2);
\draw[arrow] (cap2) -- (cap3);
\draw[arrow] (cap3) -- (cap5);
\draw[arrow] (cap2) -- (cap4);
\draw[arrow] (cap4) -- (cap5);

% Frecce verso i framework componenti
\draw[arrow,gdogreen] (cap2) -- (assa);
\draw[arrow,gdogreen] (cap3) -- (graf);
\draw[arrow,gdogreen] (cap4) -- (min);

% Frecce dai framework a GIST
\draw[arrow,gdored] (assa) -- (gist);
\draw[arrow,gdored] (graf) -- (gist);
\draw[arrow,gdored] (min) -- (gist);

% Frecce di feedback
\draw[feedback] (cap5) to[bend right=30] (cap2);
\draw[feedback] (cap5) to[bend left=30] (cap3);

% Annotazioni
\node[
    text width=3cm,
    align=center,
    font=\footnotesize\itshape,
    above=0.5cm of cap1
] {Identificazione\\del problema};

\node[
    text width=3cm,
    align=center,
    font=\footnotesize\itshape,
    right=0.5cm of cap5
] {Validazione\\empirica};

% Legenda dei pesi
\node[
    draw=gdogray,
    dashed,
    text width=5cm,
    align=left,
    font=\footnotesize,
    below right=0.5cm and 1cm of gist
] {
    Pesi nel GIST Score:\\
    • ASSA-GDO: 28\%\\
    • GRAF: 35\%\\
    • MIN: 37\%
};

\end{tikzpicture}
\caption{Struttura della tesi e interdipendenze tra capitoli}
\end{figure}

% ===================================================================
% CAPITOLO 2 - FIGURA 1: TOPOLOGIA DELLA RETE GDO
% ===================================================================
\section*{Figura 2.1: Topologia della rete distribuita nel settore GDO}

\begin{figure}[H]
\centering
\begin{tikzpicture}[
    scale=0.9,
    transform shape,
    store/.style={
        circle,
        draw=gdoblue,
        fill=gdolightblue!50,
        minimum size=0.8cm,
        font=\tiny
    },
    dc/.style={
        rectangle,
        draw=gdored,
        fill=gdored!20,
        minimum width=2cm,
        minimum height=1.5cm,
        align=center,
        font=\small\bfseries
    },
    supplier/.style={
        diamond,
        draw=gdogreen,
        fill=gdogreen!20,
        minimum size=1.2cm,
        align=center,
        font=\tiny
    },
    cloudstyle/.style={
        cloud,
        draw=gdopurple,
        fill=gdopurple!10,
        minimum width=2.5cm,
        minimum height=1.8cm,
        align=center,
        font=\small
    },
    connection/.style={
        draw=gdogray,
        thick
    },
    attack/.style={
        draw=gdored,
        ultra thick,
        dashed,
        ->,
        >=Stealth
    }
]

% Data Center centrale
\node[dc] (datacenter) at (0,0) {Data Center\\Centrale};

% Cloud services
\node[cloudstyle, above=2cm of datacenter] (cloud) {Cloud\\Services};

% Punti vendita - Regione Nord
\foreach \i in {1,2,3} {
    \node[store] (storeN\i) at (-4+\i*1.2, 3) {PV};
}

% Punti vendita - Regione Sud
\foreach \i in {1,2,3} {
    \node[store] (storeS\i) at (-4+\i*1.2, -3) {PV};
}

% Punti vendita - Regione Est
\foreach \i in {1,2} {
    \node[store] (storeE\i) at (4, 1-\i*1.5) {PV};
}

% Punti vendita - Regione Ovest
\foreach \i in {1,2} {
    \node[store] (storeW\i) at (-4, 1-\i*1.5) {PV};
}

% Fornitori
\node[supplier] (supplier1) at (-3, 0) {Fornitore\\1};
\node[supplier] (supplier2) at (3, 0) {Fornitore\\2};

% Hub regionali
\node[
    rectangle,
    draw=gdoorange,
    fill=gdoorange!20,
    minimum width=1.5cm,
    minimum height=1cm
] (hubN) at (0, 3) {Hub Nord};

\node[
    rectangle,
    draw=gdoorange,
    fill=gdoorange!20,
    minimum width=1.5cm,
    minimum height=1cm
] (hubS) at (0, -3) {Hub Sud};

% Connessioni principali
\draw[connection] (datacenter) -- (cloud);
\draw[connection] (datacenter) -- (hubN);
\draw[connection] (datacenter) -- (hubS);
\draw[connection] (datacenter) -- (supplier1);
\draw[connection] (datacenter) -- (supplier2);

% Connessioni ai punti vendita
\foreach \i in {1,2,3} {
    \draw[connection] (hubN) -- (storeN\i);
    \draw[connection] (hubS) -- (storeS\i);
}

\foreach \i in {1,2} {
    \draw[connection] (datacenter) -- (storeE\i);
    \draw[connection] (datacenter) -- (storeW\i);
}

% Indicatori di attacco
\node[font=\small\bfseries,gdored] at (5, 3.5) {Vettori di Attacco};
\draw[attack] (5, 3) -- (cloud);
\draw[attack] (5, 2.5) -- (supplier2);
\draw[attack] (5, 2) -- (storeE1);

% Calcolo della superficie di attacco
\node[
    draw=gdogray,
    dashed,
    text width=4.5cm,
    align=left,
    font=\footnotesize,
    below=4cm of datacenter
] {
    \textbf{Metriche di Rete:}\\
    • Nodi totali (N): 15\\
    • Connettività (C): 0.47\\
    • Accessibilità (A): 0.23\\
    • SAD = 147
};

\end{tikzpicture}
\caption{Topologia tipica della rete distribuita GDO con indicazione dei principali vettori di attacco}
\end{figure}

% ===================================================================
% CAPITOLO 2 - FIGURA 2: EVOLUZIONE DELLE TATTICHE RANSOMWARE
% ===================================================================
\section*{Figura 2.2: Evoluzione temporale delle tattiche ransomware nel settore GDO}

\begin{figure}[H]
\centering
\begin{tikzpicture}
\begin{axis}[
    width=13cm,
    height=7cm,
    xlabel={Anno},
    ylabel={Tempo medio (ore)},
    xmin=2020.5, xmax=2024.5,
    ymin=0, ymax=180,
    xtick={2021,2022,2023,2024},
    ytick={0,20,40,60,80,100,120,140,160,180},
    legend style={
        at={(0.02,0.98)},
        anchor=north west,
        font=\small,
        draw=none,
        fill=white,
        fill opacity=0.9
    },
    grid=major,
    grid style={dashed,gray!30},
    axis lines=left,
    axis line style={->,thick}
]

% Tempo dalla compromissione al cifraggio
\addplot[
    color=gdored,
    mark=square*,
    thick,
    mark size=3pt
] coordinates {
    (2021,72) (2022,48) (2023,24) (2024,11)
};
\addlegendentry{Tempo al cifraggio};

% Tempo di rilevamento
\addplot[
    color=gdoblue,
    mark=triangle*,
    thick,
    mark size=3pt
] coordinates {
    (2021,127) (2022,96) (2023,54) (2024,24)
};
\addlegendentry{Tempo di rilevamento};

% Tempo di recupero
\addplot[
    color=gdogreen,
    mark=o,
    thick,
    mark size=3pt
] coordinates {
    (2021,168) (2022,144) (2023,96) (2024,72)
};
\addlegendentry{Tempo di recupero};

% Annotazioni per trend significativi
\node[
    draw=gdored,
    fill=white,
    align=center,
    font=\footnotesize
] at (axis cs:2023.5,15) {-85\%\\in 3 anni};

\draw[->,thick,gdored] (axis cs:2023.3,15) -- (axis cs:2024,11);

% Area di criticità
\fill[gdored!10] (axis cs:2024,0) rectangle (axis cs:2024.5,30);
\node[font=\footnotesize,rotate=90] at (axis cs:2024.35,15) {Zona critica};

\end{axis}
\end{tikzpicture}
\caption{Evoluzione delle metriche temporali negli attacchi ransomware alla GDO}
\end{figure}

% ===================================================================
% CAPITOLO 2 - FIGURA 3: ARCHITETTURA ZERO TRUST
% ===================================================================
\section*{Figura 2.3: Architettura Zero Trust per la GDO}

\begin{figure}[H]
\centering
\begin{tikzpicture}[
    scale=0.85,
    transform shape,
    policy/.style={
        rectangle,
        draw=gdoblue,
        fill=gdoblue!20,
        minimum width=2.5cm,
        minimum height=1cm,
        align=center,
        font=\small
    },
    component/.style={
        rectangle,
        rounded corners,
        draw=gdogreen,
        fill=gdogreen!15,
        minimum width=2cm,
        minimum height=0.8cm,
        align=center,
        font=\footnotesize
    },
    zone/.style={
        rectangle,
        draw=gdored,
        dashed,
        thick,
        fill=gdored!5,
        minimum width=3cm,
        minimum height=2.5cm
    },
    decision/.style={
        diamond,
        draw=gdopurple,
        fill=gdopurple!20,
        minimum width=1.8cm,
        minimum height=1.8cm,
        align=center,
        font=\footnotesize
    },
    flow/.style={
        ->,
        thick,
        >=Stealth
    }
]

% Policy Engine al centro
\node[policy, minimum width=3cm, minimum height=1.5cm] (engine) at (0,0) {\textbf{Policy Engine}\\Zero Trust};

% Componenti di verifica
\node[component] (identity) at (-4, 2) {Verifica\\Identità};
\node[component] (device) at (-4, 0) {Verifica\\Dispositivo};
\node[component] (context) at (-4, -2) {Analisi\\Contesto};

% Componenti di enforcement
\node[component] (micro) at (4, 2) {Micro-\\segmentazione};
\node[component] (privilege) at (4, 0) {Privilegio\\Minimo};
\node[component] (inspection) at (4, -2) {Ispezione\\Pervasiva};

% Zone di sicurezza
\node[zone, label={[font=\footnotesize]above:Zona Non Fidata}] (untrusted) at (-6, 0) {};
\node[zone, label={[font=\footnotesize]above:Zona Controllata}] (controlled) at (6, 0) {};

% Decision Points
\node[decision] (decision1) at (0, 2.5) {Autorizza?};
\node[decision] (decision2) at (0, -2.5) {Monitora};

% Flussi di verifica
\draw[flow, gdoblue] (identity) -- (engine);
\draw[flow, gdoblue] (device) -- (engine);
\draw[flow, gdoblue] (context) -- (engine);

% Flussi di enforcement
\draw[flow, gdogreen] (engine) -- (micro);
\draw[flow, gdogreen] (engine) -- (privilege);
\draw[flow, gdogreen] (engine) -- (inspection);

% Flussi decisionali
\draw[flow, gdopurple] (engine) -- (decision1);
\draw[flow, gdopurple] (engine) -- (decision2);

% Feedback loops
\draw[flow, dashed, gdogray] (decision2) to[bend right=45] (context);
\draw[flow, dashed, gdogray] (inspection) to[bend right=30] (decision2);

% Metriche
\node[
    draw=black,
    fill=yellow!10,
    text width=4cm,
    align=center,
    font=\footnotesize,
    below=3.5cm of engine
] {
    \textbf{Risultati Zero Trust:}\\
    • SAD: -42.7\%\\
    • MTTD: 24h (-81\%)\\
    • Contenimento: 77\%
};

% Etichette delle zone
\node[font=\footnotesize\itshape] at (-6, -3) {Richieste};
\node[font=\footnotesize\itshape] at (6, -3) {Risorse};

\end{tikzpicture}
\caption{Architettura Zero Trust implementata con componenti ASSA-GDO}
\end{figure}

% ===================================================================
% CAPITOLO 2 - FIGURA 4: CONFRONTO METRICHE PRIMA/DOPO
% ===================================================================
\section*{Figura 2.4: Confronto delle metriche di sicurezza prima e dopo l'implementazione Zero Trust}

\begin{figure}[H]
\centering
\begin{tikzpicture}
\begin{axis}[
    ybar,
    bar width=0.35cm,
    width=13cm,
    height=7cm,
    xlabel={Metrica di Sicurezza},
    ylabel={Valore},
    symbolic x coords={SAD,MTTD,MTTR,Propagazione,Contenimento},
    xtick=data,
    x tick label style={rotate=45, anchor=east, font=\small},
    ymin=0,
    ymax=260,
    legend style={
        at={(0.98,0.98)},
        anchor=north east,
        font=\small
    },
    grid=major,
    grid style={dashed,gray!30},
    axis lines=left,
    nodes near coords,
    nodes near coords style={font=\tiny},
    every node near coord/.append style={rotate=90, anchor=west}
]

% Prima dell'implementazione
\addplot[
    fill=gdored!70,
    draw=gdored
] coordinates {
    (SAD,254) (MTTD,127) (MTTR,168) (Propagazione,87) (Contenimento,23)
};

% Dopo l'implementazione
\addplot[
    fill=gdogreen!70,
    draw=gdogreen
] coordinates {
    (SAD,147) (MTTD,24) (MTTR,72) (Propagazione,12) (Contenimento,77)
};

\legend{Pre Zero Trust, Post Zero Trust};

% Linee di miglioramento percentuale
\node[font=\footnotesize,gdoblue] at (axis cs:SAD,200) {-42\%};
\node[font=\footnotesize,gdoblue] at (axis cs:MTTD,75) {-81\%};
\node[font=\footnotesize,gdoblue] at (axis cs:MTTR,120) {-57\%};
\node[font=\footnotesize,gdoblue] at (axis cs:Propagazione,50) {-86\%};
\node[font=\footnotesize,gdoblue] at (axis cs:Contenimento,50) {+235\%};

\end{axis}
\end{tikzpicture}
\caption{Impatto quantitativo dell'implementazione Zero Trust sulle metriche chiave}
\end{figure}

% ===================================================================
% CAPITOLO 2 - FIGURA 5: ANALISI ROI MONTE CARLO
% ===================================================================
\section*{Figura 2.5: Analisi Monte Carlo del ritorno sull'investimento per Zero Trust}

\begin{figure}[H]
\centering
\begin{tikzpicture}
\begin{axis}[
    width=13cm,
    height=7cm,
    xlabel={ROI (\%)},
    ylabel={Densità di Probabilità},
    xmin=-50, xmax=400,
    ymin=0, ymax=0.015,
    xtick={-50,0,50,100,150,200,250,300,350,400},
    ytick={0,0.003,0.006,0.009,0.012,0.015},
    legend style={
        at={(0.98,0.98)},
        anchor=north east,
        font=\small
    },
    grid=major,
    grid style={dashed,gray!30},
    axis lines=left,
    axis line style={->,thick}
]

% Distribuzione scenario pessimistico (efficienza 0.4)
\addplot[
    smooth,
    thick,
    color=gdored,
    fill=gdored!20,
    fill opacity=0.3,
    domain=-50:250,
    samples=100
] {0.004*exp(-((x-80)^2)/(2*40^2))};
\addlegendentry{Pessimistico (eff. 0.4)};

% Distribuzione scenario realistico (efficienza 0.6)
\addplot[
    smooth,
    thick,
    color=gdoblue,
    fill=gdoblue!20,
    fill opacity=0.3,
    domain=50:350,
    samples=100
] {0.007*exp(-((x-187)^2)/(2*35^2))};
\addlegendentry{Realistico (eff. 0.6)};

% Distribuzione scenario ottimale (efficienza 0.8)
\addplot[
    smooth,
    thick,
    color=gdogreen,
    fill=gdogreen!20,
    fill opacity=0.3,
    domain=150:400,
    samples=100
] {0.010*exp(-((x-287)^2)/(2*30^2))};
\addlegendentry{Ottimale (eff. 0.8)};

% Linee verticali per i valori mediani
\draw[dashed,thick,gdored] (axis cs:80,0) -- (axis cs:80,0.015);
\draw[dashed,thick,gdoblue] (axis cs:187,0) -- (axis cs:187,0.015);
\draw[dashed,thick,gdogreen] (axis cs:287,0) -- (axis cs:287,0.015);

% Annotazioni
\node[anchor=north,font=\footnotesize] at (axis cs:80,-0.001) {80\%};
\node[anchor=north,font=\footnotesize] at (axis cs:187,-0.001) {187\%};
\node[anchor=north,font=\footnotesize] at (axis cs:287,-0.001) {287\%};

% Area di ROI positivo
\fill[black!10] (axis cs:0,0) rectangle (axis cs:400,0.015);
\draw[thick] (axis cs:0,0) -- (axis cs:0,0.015);
\node[font=\footnotesize,rotate=90] at (axis cs:-25,0.007) {ROI Positivo};

% Box con statistiche
\node[
    draw=gdogray,
    fill=white,
    text width=3.5cm,
    align=left,
    font=\footnotesize,
    at={(axis cs:320,0.012)}
] {
    \textbf{Statistiche:}\\
    • P(ROI>0): 95\%\\
    • Payback: 18 mesi\\
    • σ: ±30\%
};

\end{axis}
\end{tikzpicture}
\caption{Distribuzione probabilistica del ROI basata su 10.000 simulazioni Monte Carlo}
\end{figure}

% ===================================================================
% NOTE PER L'IMPLEMENTAZIONE
% ===================================================================
\clearpage
\section*{Note per l'Implementazione delle Figure}

\subsection*{Requisiti LaTeX}
Per compilare correttamente queste figure, assicurarsi di avere i seguenti pacchetti nel preambolo del documento principale:

\begin{verbatim}
\usepackage{tikz}
\usepackage{pgfplots}
\pgfplotsset{compat=1.18}
\usetikzlibrary{arrows.meta,positioning,calc,patterns,
                decorations.pathreplacing,shadows,
                shapes.geometric,backgrounds,fit}
\end{verbatim}

\subsection*{Generazione delle Figure}
Ogni figura può essere:
\begin{enumerate}
    \item \textbf{Integrata direttamente} nel documento LaTeX principale
    \item \textbf{Compilata separatamente} e inclusa come PDF usando:
    \begin{verbatim}
    \includegraphics[width=\textwidth]{nome_figura.pdf}
    \end{verbatim}
    \item \textbf{Esternalizzata} usando il pacchetto \texttt{tikzexternalize} per velocizzare la compilazione
\end{enumerate}

\subsection*{Personalizzazione}
I colori sono definiti all'inizio del documento e possono essere modificati per adattarsi al tema della presentazione o alle linee guida dell'università:

\begin{verbatim}
\definecolor{gdoblue}{RGB}{0, 102, 204}
\definecolor{gdogreen}{RGB}{76, 175, 80}
\definecolor{gdored}{RGB}{244, 67, 54}
% ... altri colori ...
\end{verbatim}

\subsection*{Ottimizzazione per la Stampa}
Per la versione stampata della tesi, considerare:
\begin{itemize}
    \item Aumentare lo spessore delle linee (\texttt{thick} → \texttt{ultra thick})
    \item Verificare la leggibilità in bianco e nero
    \item Aggiungere pattern diversi per distinguere le aree nei grafici
\end{itemize}

\subsection*{Directory Structure Consigliata}
\begin{verbatim}
thesis/
├── chapters/
│   ├── cap1.tex
│   └── cap2.tex
├── figures/
│   ├── cap1/
│   │   ├── evoluzione_attacchi.tex
│   │   └── thesis_structure.tex
│   └── cap2/
│       ├── topologia_rete.tex
│       ├── evoluzione_ransomware.tex
│       ├── zero_trust_architecture.tex
│       ├── confronto_metriche.tex
│       └── roi_analysis.tex
└── main.tex
\end{verbatim}

\end{document}