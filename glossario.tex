% ===================================================================
% GLOSSARIO TERMINI - TESI GDO
% File: glossario.tex (file sorgente da modificare)
% Utilizzare con: \loadglsentries{glossario.tex} nel preambolo
% ===================================================================

% ===== ACRONIMI PRINCIPALI =====

\newacronym[
    description={Settore del commercio al dettaglio caratterizzato da catene di punti vendita con gestione centralizzata e volumi significativi.}]
    {gdo}{GDO}{Grande Distribuzione Organizzata}
\newacronym[
    description={Framework integrato per la misurazione del grado di integrazione }]
    {gist}{GIST}{GDO Integrated Security Transformation}
\newacronym[
    description={Tecnologia di identificazione a radiofrequenza.}]
    {rfid}{RFId}{Radio Frequency Identification}

\newacronym[
    description={Codice univoco utilizzato per la gestione delle scorte.}]
    {sku}{SKU}{Stock Keeping Unit}
\newacronym[
    description={ E' un insieme di tecnologie e sistemi integrati progettati per controllare e ottimizzare la qualità dell'aria, la temperatura e l'umidità negli ambienti interni di edifici residenziali, commerciali e industriali.}]
    {hvac}{HVAC}{Heating, Ventilation, and Air Conditioning}
\newacronym[
    description={Sistema di alimentazione ininterrotta che fornisce energia temporanea ai dispositivi collegati in caso di interruzione della corrente elettrica.}]
    {ups}{UPS}{Uninterruptible Power Supply}
\newacronym[
    description={Dispositivo di protezione elettrica che interrompe il flusso di corrente in caso di sovraccarico o cortocircuito.}]
    {circuit-breaker}{Circuit Breaker}{Circuit Breaker}
\newacronym[
    description={Algoritmo che quantifica la superficie di attacco considerando non solo vulnerabilità tecniche ma anche fattori organizzativi e processuali}]
    {assa-gdo}{ASSA-GDO}{Attack Surface Score Aggregated for GDO}


% ===== STANDARD E NORMATIVE =====

\newacronym[
    description={Standard di sicurezza internazionale per la protezione dei dati delle carte di pagamento, richiesto per tutti gli esercenti che processano transazioni con carte di credito.}]
    {pci-dss}{PCI-DSS}{Payment Card Industry Data Security Standard}

\newacronym[
    description={Regolamento (UE) 2016/679 sulla protezione dei dati personali e sulla libera circolazione di tali dati nell'Unione Europea.}]
    {gdpr}{GDPR}{General Data Protection Regulation}

\newacronym[
    description={Direttiva (UE) 2022/2555 relativa a misure per un livello comune elevato di cibersicurezza nell'Unione.}]
    {nis2}{NIS2}{Network and Information Security Directive 2}

% ===== METRICHE E KPI =====

\newacronym[
    description={Metodologia di valutazione che considera tutti i costi diretti e indiretti sostenuti durante l'intero ciclo di vita di un sistema informatico.}]
    {tco}{TCO}{Total Cost of Ownership}

\newacronym[
    description={Tempo medio intercorrente tra guasti consecutivi di un sistema, utilizzato come indicatore di affidabilità.}]
    {mtbf}{MTBF}{Mean Time Between Failures}

\newacronym[
    description={Tempo medio necessario per ripristinare la piena operatività di un sistema dopo un guasto o un incidente.}]
    {mttr}{MTTR}{Mean Time To Recovery}

\newacronym[
    description={Scheda plug-in o di dispositivo esterno che salvaguarda e gestisce i segreti (soprattutto le chiavi digitali), esegue funzioni di crittografia per le firme digitali e altre funzioni crittografiche.}]
    {hsm}{HSM}{Hardware Security Module}

\newacronym[
    description={Protocollo ISO standard di messaggistica leggero di tipo publish-subscribe posizionato in cima a TCP/IP, progettato per le situazioni in cui è richiesto un basso impatto energetico e dove la banda è limitata.}]
    {mqtt}{MQTT}{Message Queuing Telemetry Transport}

\newacronym[
    description={Tecnologia di crittografia che utilizza una coppia di chiavi, una pubblica e una privata, per garantire la riservatezza e l'integrità delle comunicazioni.}]
    {pkc}{PKC}{Public Key Cryptography}

\newacronym[
    description={Tecnica di crittografia che utilizza la stessa chiave segreta per cifrare e decifrare i dati.}]
    {skc}{SKC}{Symmetric Key Cryptography}

\newacronym[
    description={Costo complessivo sostenuto da un'organizzazione per rispettare normative, regolamenti e standard di settore, includendo spese dirette (personale, sistemi, consulenze) e indirette (tempo, opportunità perse, inefficienze operative).}]
    {tcc}{TCC}{Total Compliance Cost}



    % ===== METODOLOGIE E FRAMEWORK =====

\newacronym[
    description={Metodologia per supportare decisioni complesse che coinvolgono criteri multipli e spesso conflittuali.}]
    {mcdm}{MCDM}{Multi-Criteria Decision Making}

% ===== TECNOLOGIE DI RETE =====

\newacronym[
    description={Architettura di rete che estende i principi della virtualizzazione alle reti geografiche, permettendo controllo centralizzato e ottimizzazione dinamica del traffico.}]
    {sd-wan}{SD-WAN}{Software-Defined Wide Area Network}

% ===== CLOUD E VIRTUALIZZAZIONE =====

\newacronym[
    description={Modello di cloud computing che fornisce risorse di calcolo virtualizzate attraverso Internet.}]
    {iaas}{IaaS}{Infrastructure as a Service}

\newacronym[
    description={Modello di cloud computing che fornisce una piattaforma di sviluppo e deployment completa attraverso Internet.}]
    {paas}{PaaS}{Platform as a Service}

\newacronym[
    description={Modello di distribuzione software in cui le applicazioni sono fornite attraverso Internet come servizio.}]
    {saas}{SaaS}{Software as a Service}

% ===== SICUREZZA INFORMATICA =====

\newacronym[
    description={Soluzione software che aggrega e analizza dati di sicurezza da diverse fonti per identificare minacce e incidenti.}]
    {siem}{SIEM}{Security Information and Event Management}

\newacronym[
    description={Centro operativo dedicato al monitoraggio, rilevamento e risposta agli incidenti di sicurezza informatica.}]
    {soc}{SOC}{Security Operations Center}

\newacronym[
    description={Framework di processi e tecnologie per gestire identità digitali e controlli di accesso.}]
    {iam}{IAM}{Identity and Access Management}

% ===== EDGE COMPUTING =====

\newglossaryentry{edge}{
    name={Edge Computing},
    description={Paradigma di elaborazione distribuita che porta computazione e storage vicino alle sorgenti di dati per ridurre latenza e migliorare performance.}
}
\newglossaryentry{on-premise}{
    name={On-Premise},
    description={Modello di deployment in cui le risorse IT sono gestite localmente all'interno dell'organizzazione, anziché essere ospitate nel cloud.}    
}

\newacronym[
    description={Rete di dispositivi fisici interconnessi attraverso Internet, dotati di sensori e capacità di comunicazione.}]
    {iot}{IoT}{Internet of Things}

% ===== DEVELOPMENT E DEVOPS =====

\newacronym[
    description={Pratiche di sviluppo software che enfatizzano integrazione frequente del codice e deployment automatizzato.}]
    {cicd}{CI/CD}{Continuous Integration/Continuous Deployment}

\newacronym[
    description={Pratica di gestione dell'infrastruttura IT attraverso codice versionato e automatizzato.}]
    {iac}{IaC}{Infrastructure as Code}

% ===== METODOLOGIE AGILI =====

\newacronym[
    description={Metodologia che integra sviluppo software (Dev) e operazioni IT (Ops) per accelerare il ciclo di vita dello sviluppo software.}]
    {devops}{DevOps}{Development Operations}

\newacronym[
    description={Estensione di DevOps che integra la sicurezza (Sec) nel processo di sviluppo e deployment software.}]
    {devsecops}{DevSecOps}{Development Security Operations}

% ===== BUSINESS CONTINUITY =====

\newacronym[
    description={Tempo massimo accettabile per il ripristino di un servizio dopo un'interruzione.}]
    {rto}{RTO}{Recovery Time Objective}

\newacronym[
    description={Quantità massima accettabile di perdita di dati in caso di interruzione del servizio.}]
    {rpo}{RPO}{Recovery Point Objective}

\newacronym[
    description={Piano strategico per garantire la continuità operativa dell'organizzazione durante e dopo eventi disruptivi.}]
    {bcp}{BCP}{Business Continuity Plan}

% ===== ANALYTICS E AI =====

\newacronym[
    description={Sottocampo dell'intelligenza artificiale che utilizza algoritmi per permettere ai sistemi di imparare automaticamente dai dati.}]
    {ml}{ML}{Machine Learning}

\newacronym[
    description={Simulazione di processi di intelligenza umana attraverso sistemi informatici.}]
    {ai}{AI}{Artificial Intelligence}

\newacronym[
    description={Insieme di strategie e tecnologie per l'analisi dei dati aziendali per supportare decisioni strategiche.}]
    {bi}{BI}{Business Intelligence}

% ===== PERFORMANCE E MONITORING =====

\newacronym[
    description={Contratto che definisce i livelli di servizio attesi tra fornitore e cliente.}]
    {sla}{SLA}{Service Level Agreement}

\newacronym[
    description={Metrica utilizzata per valutare l'efficacia nel raggiungimento di obiettivi strategici.}]
    {kpi}{KPI}{Key Performance Indicator}

\newacronym[
    description={Monitoraggio e gestione delle prestazioni e disponibilità delle applicazioni software.}]
    {apm}{APM}{Application Performance Monitoring}

% ===================================================================
% TERMINI AGGIUNTIVI - CAPITOLO 3
% ===================================================================

% ===== SISTEMI DI MONITORAGGIO E CONTROLLO =====

\newacronym[
    description={Sistema integrato per il controllo e monitoraggio automatico degli impianti edilizi (HVAC, illuminazione, sicurezza, energia).}]
    {bms}{BMS}{Building Management System}

\newacronym[
    description={Metodologia numerica per l'analisi e la simulazione del comportamento dei fluidi e del trasferimento termico attraverso modelli matematici.}]
    {cfd}{CFD}{Computational Fluid Dynamics}

\newacronym[
    description={Metrica di efficienza energetica dei data center definita come il rapporto tra energia totale consumata e energia utilizzata dall'equipaggiamento IT.}]
    {pue}{PUE}{Power Usage Effectiveness}

\newacronym[
    description={Dispositivo elettronico che controlla la velocità di rotazione dei motori elettrici variando la frequenza e la tensione di alimentazione per ottimizzare l'efficienza energetica.}]
    {vsd}{VSD}{Variable Speed Drive}

\newacronym[
    description={EDPB description.}]
    {edpb}{EDPB}{European Data Protection Bureau}

\newacronym[
    description={Data Protection Officer description.}]
    {dpo}{DPO}{Data Protection Officer}



% ===== TECNOLOGIE DI RETE AVANZATE =====

\newacronym[
    description={Tecnologia di analisi del traffico di rete che esamina il contenuto dei pacchetti dati oltre agli header per classificazione, security e quality of service.}]
    {dpi}{DPI}{Deep Packet Inspection}

\newacronym[
    description={Campo nell'header IP utilizzato per la classificazione e gestione della Quality of Service (QoS) nel traffico di rete.}]
    {dscp}{DSCP}{Differentiated Services Code Point}

\newacronym[
    description={Modello statistico per l'analisi e previsione di serie temporali che combina componenti autoregressivi, integrati e di media mobile.}]
    {arima}{ARIMA}{AutoRegressive Integrated Moving Average}

% ===== METRICHE ECONOMICHE E FINANZIARIE =====

\newacronym[
    description={Valore attuale netto, metrica finanziaria che calcola il valore presente di flussi di cassa futuri scontati al costo del capitale per valutare la redditività di investimenti.}]
    {npv}{NPV}{Net Present Value}

\newacronym[
    description={Tasso interno di rendimento, tasso di sconto che rende il NPV di un investimento uguale a zero, utilizzato per valutare l'attrattività economica di progetti.}]
    {irr}{IRR}{Internal Rate of Return}

\newacronym[
    description={Costo medio ponderato del capitale, rappresenta il tasso di rendimento minimo richiesto dagli investitori per finanziare un'azienda.}]
    {wacc}{WACC}{Weighted Average Cost of Capital}

% ===== CONTAINER E ORCHESTRAZIONE =====

\newglossaryentry{kubernetes}{
    name=Kubernetes,
    description={Piattaforma open-source per l'orchestrazione automatica di container che gestisce deployment, scaling, e operazioni di applicazioni containerizzate su cluster distribuiti.}
}

\newglossaryentry{microservizi}{
    name=Microservizi,
    description={Architettura applicativa che struttura un'applicazione come collezione di servizi loosely coupled, deployabili indipendentemente e organizzati attorno a specifiche funzionalità business.}
}

\newglossaryentry{container}{
    name=Container,
    description={Tecnologia di virtualizzazione leggera che incapsula applicazioni e le loro dipendenze in unità portabili ed eseguibili in modo consistente attraverso diversi ambienti.}
}

% ===== SERVICE MESH E NETWORKING =====

\newglossaryentry{servicemesh}{
    name=Service Mesh,
    description={Infrastruttura di comunicazione dedicata per gestire le interazioni tra microservizi, fornendo funzionalità di security, observability, e traffic management.}
}

\newglossaryentry{istio}{
    name=Istio,
    description={Piattaforma service mesh open-source che fornisce un modo uniforme per connettere, gestire e mettere in sicurezza microservizi attraverso proxy sidecar.}
}

% ===== STANDARD E PROTOCOLLI =====

\newacronym[
    description={Organizzazione che sviluppa standard per sistemi HVAC e gestione dell'ambiente negli edifici.}]
    {ashrae}{ASHRAE}{American Society of Heating, Refrigerating and Air-Conditioning Engineers}

\newglossaryentry{raft}{
    name=Raft,
    description={Algoritmo di consensus distribuito progettato per essere comprensibile, che garantisce la consistenza dei dati in sistemi distribuiti attraverso elezione di leader e replica dei log.}
}

% ===== DEPLOYMENT E AUTOMATION =====

\newglossaryentry{terraform}{
    name=Terraform,
    description={Tool open-source per Infrastructure as Code che permette di definire, provisioning e gestire infrastruttura cloud attraverso file di configurazione dichiarativi.}
}

\newglossaryentry{bluegreen}{
    name=Blue-Green Deployment,
    description={Strategia di deployment che mantiene due ambienti di produzione identici (blue e green) per permettere rilasci con zero downtime e rollback immediato.}
}

\newglossaryentry{selfhealing}{
    name=Self-Healing,
    description={Capacità di un sistema di rilevare automaticamente guasti o degradazioni delle prestazioni e intraprendere azioni correttive senza intervento umano.}
}

% ===== SICUREZZA E COMPLIANCE =====

\newglossaryentry{zerotrust}{
    name=Zero Trust,
    description={Modello di sicurezza che assume che nessun utente o dispositivo, interno o esterno alla rete, sia attendibile per default e richiede verifica continua per ogni accesso.}
}

\newglossaryentry{vectorclock}{
    name=Vector Clock,
    description={Algoritmo per determinare l'ordine causale degli eventi in sistemi distribuiti, dove ogni processo mantiene un vettore di timestamp logici.}
}

% ===== OTTIMIZZAZIONE ENERGETICA =====

\newglossaryentry{freecooling}{
    name=Free Cooling,
    description={Tecnologia di raffreddamento che sfrutta le condizioni climatiche esterne favorevoli per ridurre o eliminare l'uso di sistemi di refrigerazione meccanica.}
}

\newglossaryentry{containment}{
    name=Containment,
    description={Strategia di gestione del flusso d'aria nei data center che separa fisicamente i corridoi di aria calda e fredda per migliorare l'efficienza del cooling.}
}

% ===== CACHE E STORAGE =====

\newglossaryentry{cachecoherency}{
    name=Cache Coherency,
    description={Protocollo che garantisce la consistenza dei dati condivisi tra multiple cache in sistemi distribuiti, evitando inconsistenze dovute a modifiche concorrenti.}
}

\newacronym[
    description={Rete geograficamente distribuita di server che fornisce contenuti web agli utenti dalla località più vicina per ridurre latenza.}]
    {cdn}{CDN}{Content Delivery Network}

% ===== METODOLOGIE DI SVILUPPO =====

\newglossaryentry{apifirst}{
    name=API-First Design,
    description={Approccio di progettazione software che considera le API come cittadini di prima classe, progettandole prima dell'implementazione delle applicazioni che le utilizzano.}
}

\newglossaryentry{declarative}{
    name=Dichiarativo,
    description={Paradigma di programmazione che esprime la logica di computazione senza descrivere il flusso di controllo, specificando cosa deve essere fatto piuttosto che come.}
}

% ===================================================================
% TERMINI AGGIUNTIVI - CAPITOLO 2
% ===================================================================

% ===== SISTEMI E TECNOLOGIE PUNTO VENDITA =====

\newacronym[
    description={Sistema di elaborazione delle transazioni commerciali che gestisce pagamenti, inventario e dati di vendita nei punti vendita al dettaglio.}]
    {pos}{POS}{Point of Sale}

\newglossaryentry{memory-scraping}{
    name=Memory Scraping,
    description={Tecnica di attacco informatico che estrae dati sensibili dalla memoria volatile dei sistemi durante la finestra temporale in cui esistono in forma non cifrata.}
}

% ===== ENDPOINT DETECTION E RESPONSE =====

\newacronym[
    description={Soluzione di sicurezza che monitora continuamente endpoint e workstation per rilevare e rispondere a minacce informatiche avanzate.}]
    {edr}{EDR}{Endpoint Detection and Response}

\newglossaryentry{behavioral-analysis}{
    name=Behavioral Analysis,
    description={Analisi comportamentale - Tecnica di sicurezza che identifica attività anomale attraverso l'osservazione di pattern comportamentali che deviano dalla baseline normale.}
}

% ===== INTRUSION DETECTION E PREVENTION =====

\newacronym[
    description={Sistema di rilevamento delle intrusioni che monitora il traffico di rete e le attività di sistema per identificare comportamenti sospetti o malevoli.}]
    {ids}{IDS}{Intrusion Detection System}

\newacronym[
    description={Sistema di prevenzione delle intrusioni che oltre al rilevamento può bloccare attivamente traffico o attività identificate come dannose.}]
    {ips}{IPS}{Intrusion Prevention System}

\newglossaryentry{signature-based-detection}{
    name=Signature-Based Detection,
    description={Rilevamento basato su firme - Metodo di detection che utilizza pattern predefiniti di attacchi conosciuti per identificare attività malevole.}
}

\newglossaryentry{anomaly-based-detection}{
    name=Anomaly-Based Detection,
    description={Rilevamento basato su anomalie - Metodo di detection che identifica deviazioni dal comportamento normale utilizzando modelli statistici e machine learning.}
}

% ===== CLOUD SECURITY =====

\newacronym[
    description={Soluzione per la gestione continua della postura di sicurezza in ambienti cloud, identificando misconfigurazione e rischi.}]
    {cspm}{CSPM}{Cloud Security Posture Management}

\newglossaryentry{cloud-native}{
    name=Cloud-Native,
    description={Approccio di sviluppo e deployment che sfrutta pienamente le caratteristiche cloud, utilizzando microservizi, container e orchestrazione dinamica.}
}

% ===== ZERO TRUST E NETWORK SECURITY =====

\newglossaryentry{network-segmentation}{
    name=Network Segmentation,
    description={Segmentazione di rete - Pratica di dividere una rete in sottoreti separate per migliorare sicurezza e prestazioni, limitando la propagazione di minacce.}
}

\newglossaryentry{micro-segmentation}{
    name=Micro-Segmentation,
    description={Micro-segmentazione - Segmentazione granulare che applica controlli di sicurezza a livello di singolo workload o applicazione.}
}

\newglossaryentry{lateral-movement}{
    name=Lateral Movement,
    description={Movimento laterale - Tecnica utilizzata dagli attaccanti per spostarsi attraverso una rete compromessa alla ricerca di dati o sistemi di valore.}
}

% ===== PRIVACY E COMPLIANCE =====

\newglossaryentry{differential-privacy}{
    name=Differential Privacy,
    description={Privacy differenziale - Tecnica matematica che permette di estrarre informazioni utili da dataset contenenti dati personali garantendo protezione della privacy individuale.}
}

\newglossaryentry{privacy-by-design}{
    name=Privacy by Design,
    description={Privacy per progettazione - Approccio che integra la protezione della privacy direttamente nella progettazione di sistemi e processi business.}
}

\newglossaryentry{compliance-by-design}{
    name=Compliance by Design,
    description={Conformità per progettazione - Metodologia che integra requisiti normativi e di compliance direttamente nell'architettura dei sistemi informatici.}
}

\newglossaryentry{data-lifecycle-management}{
    name=Data Lifecycle Management,
    description={Gestione del ciclo di vita dei dati - Processo strutturato per gestire dati dalla creazione alla cancellazione sicura, includendo classificazione, uso e retention.}
}
\newglossaryentry{multi-tier}{
    name=Multi-Tier,
    description={Architettura multi-tier - Modello architetturale che separa le diverse componenti di un'applicazione in livelli distinti, migliorando scalabilità e manutenibilità.}
}

% ===== THREAT INTELLIGENCE E ANALISI RISCHI =====

\newglossaryentry{threat-landscape}{
    name=Threat Landscape,
    description={Panorama delle minacce - Visione complessiva delle minacce informatiche attive in un determinato periodo e settore, incluse tendenze e evoluzione.}
}

\newglossaryentry{threat-intelligence}{
    name=Threat Intelligence,
    description={Intelligence sulle minacce - Informazioni strutturate su minacce attuali e potenziali utilizzate per supportare decisioni di sicurezza informate.}
}

\newglossaryentry{attack-surface}{
    name=Attack Surface,
    description={Superficie di attacco - Insieme di tutti i punti di accesso possibili che un attaccante può utilizzare per entrare in un sistema o rete.}
}

\newglossaryentry{vulnerability-assessment}{
    name=Vulnerability Assessment,
    description={Valutazione delle vulnerabilità - Processo sistematico di identificazione, classificazione e prioritizzazione delle vulnerabilità di sicurezza in sistemi e reti.}
}

\newglossaryentry{penetration-testing}{
    name=Penetration Testing,
    description={Test di penetrazione - Attacco simulato autorizzato condotto per valutare la sicurezza di un sistema identificando vulnerabilità sfruttabili.}
}

% ===== TIPI DI ATTACCHI =====

\newglossaryentry{ransomware}{
    name=Ransomware,
    description={Tipo di malware che cifra i dati della vittima richiedendo un riscatto per la decifratura, spesso causando interruzioni operative significative.}
}

\newglossaryentry{supply-chain-attack}{
    name=Supply Chain Attack,
    description={Attacco alla catena di fornitura - Tipo di cyberattacco che compromette fornitori di software o hardware per raggiungere le organizzazioni target.}
}

\newglossaryentry{social-engineering}{
    name=Social Engineering,
    description={Ingegneria sociale - Tecnica di manipolazione psicologica utilizzata per indurre persone a rivelare informazioni confidenziali o compiere azioni compromettenti.}
}

\newglossaryentry{phishing}{
    name=Phishing,
    description={Tecnica di social engineering che utilizza comunicazioni fraudolente per indurre vittime a rivelare informazioni sensibili o installare malware.}
}

\newglossaryentry{malware}{
    name=Malware,
    description={Software malevolo progettato per danneggiare, disturbare o ottenere accesso non autorizzato a sistemi informatici.}
}

% ===== METRICHE E PERFORMANCE =====
\newacronym[
    description={Catena di Markov a tempo continuo - Modello matematico utilizzato per descrivere sistemi che evolvono nel tempo in modo continuo, spesso utilizzato in contesti di analisi delle prestazioni e dei rischi.}]
    {ctmc}{CTMC}{Continuous-Time Markov Chains}

\newacronym[
    description={Metrica finanziaria utilizzata per valutare l'efficienza di un investimento, calcolata come rapporto tra beneficio netto e costo dell'investimento.}]
    {roi}{ROI}{Return on Investment}

\newglossaryentry{false-positive}{
    name=False Positive,
    description={Falso positivo - Errore di classificazione in cui un sistema di sicurezza identifica erroneamente attività legittima come minaccia.}
}

\newglossaryentry{detection-rate}{
    name=Detection Rate,
    description={Tasso di rilevamento - Percentuale di minacce effettive correttamente identificate da un sistema di sicurezza rispetto al totale delle minacce presenti.}
}

% ===== RESILIENZA E CONTINUITÀ =====

\newglossaryentry{cyber-resilience}{
    name=Cyber Resilience,
    description={Resilienza informatica - Capacità di un'organizzazione di continuare a operare durante e dopo incidenti di sicurezza informatica.}
}

\newglossaryentry{incident-response}{
    name=Incident Response,
    description={Risposta agli incidenti - Processo strutturato per gestire e contenere le conseguenze di violazioni di sicurezza o cyberattacchi.}
}

\newglossaryentry{disaster-recovery}{
    name=Disaster Recovery,
    description={Ripristino di emergenza - Insieme di politiche e procedure per il recupero di sistemi IT critici dopo eventi disastrosi.}
}

\newglossaryentry{failover}{
    name=Failover,
    description={Processo automatico di commutazione verso un sistema di backup ridondante quando il sistema primario diventa non disponibile.}
}

% ===== ARCHITETTURE E FRAMEWORK =====

\newglossaryentry{defense-in-depth}{
    name=Defense in Depth,
    description={Difesa in profondità - Strategia di sicurezza che utilizza multiple layer di controlli di sicurezza per proteggere informazioni e sistemi.}
}

\newglossaryentry{security-framework}{
    name=Security Framework,
    description={Framework di sicurezza - Struttura concettuale che definisce approcci per la gestione della sicurezza informatica in un'organizzazione.}
}

\newglossaryentry{security-posture}{
    name=Security Posture,
    description={Postura di sicurezza - Stato complessivo della cybersecurity di un'organizzazione, inclusi controlli, policy e livello di rischio.}
}

\newglossaryentry{risk-assessment}{
    name=Risk Assessment,
    description={Valutazione del rischio - Processo di identificazione, analisi e valutazione dei rischi di sicurezza per supportare decisioni di gestione del rischio.}
}

% ===== AUTOMAZIONE E ORCHESTRAZIONE =====

\newglossaryentry{security-orchestration}{
    name=Security Orchestration,
    description={Orchestrazione di sicurezza - Coordinamento automatizzato di strumenti e processi di sicurezza per migliorare l'efficienza della risposta alle minacce.}
}

\newacronym[
    description={Piattaforma che combina orchestrazione, automazione e risposta per migliorare l'efficacia delle operazioni di sicurezza.}]
    {soar}{SOAR}{Security Orchestration, Automation and Response}

\newglossaryentry{playbook}{
    name=Playbook,
    description={Insieme di procedure standardizzate e automatizzate per rispondere a specifici tipi di incidenti di sicurezza o minacce.}
}

% ===== GOVERNANCE E POLICY =====

\newglossaryentry{policy-engine}{
    name=Policy Engine,
    description={Motore di policy - Sistema software che implementa, gestisce e applica automaticamente policy di sicurezza e compliance in ambienti distribuiti.}
}

\newglossaryentry{governance}{
    name=Governance,
    description={Insieme di processi, policy e controlli utilizzati per dirigere e controllare le attività IT di un'organizzazione.}
}

\newglossaryentry{audit-trail}{
    name=Audit Trail,
    description={Traccia di audit - Registro cronologico delle attività di sistema che fornisce evidenza documentale per verifiche di sicurezza e compliance.}
}
\newglossaryentry{compliance}{
    name=Compliance,
    description={Conformità e aderenza di un'organizzazione alle leggi, normative, regolamenti, standard di settore e politiche interne applicabili, attraverso l'implementazione di processi, controlli e procedure specifiche per mitigare rischi legali e reputazionali.}
}